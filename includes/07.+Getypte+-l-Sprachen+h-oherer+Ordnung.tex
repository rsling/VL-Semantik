\begin{frame}
  {Kernfragen in dieser Woche}
  \onslide<+->
  \onslide<+->
  \Large
  \centering
  Wie unterscheidet sich Montagues System von GB-Semantik?\\
  \onslide<+->
  \Halbzeile
  Welche Rolle spielen \alert{Typen}?\\
  \onslide<+->
  \Halbzeile
  Was sind \alert{$\lambda$-Sprachen?}\\
  Und woher kennen Sie den $\lambda$-Operator eigentlich schon?\\
  \onslide<+->
  \Halbzeile
  \grau{\footnotesize Texte für heute: \citet[Kapitel~4]{DowtyEa1981} | \citealt[Kapitel~7]{ChierchiaMcconnellginet2000}}
\end{frame}

\section{Einfachere Semantik}

\begin{frame}
  {Montague vs.\ Generativismus}
  \onslide<+->
  \onslide<+->
  Es geht wie immer auch ohne Bewegung.\\
  \Zeile 
  \begin{itemize}[<+->]
    \item Chierchia | auf Grundlage von GB-Syntax
      \begin{itemize}[<+->]
        \item Syntax und \alert{Semantik in Phrasenstrukturen}
        \item Sprache wird zu Logik durch unsichtbare Bewegung.
        \item Semantik als \alert{eigene Repräsentationsebene}
      \end{itemize}
      \Halbzeile
    \item Montague | \alert{Sprache ist Logik!}
      \begin{itemize}[<+->]
        \item Direkte Interpretation von Zeichen als logische Symbole
        \item Logische Form (lf) als \alert{Sichtbarmachen} logischer Eigenschaften
        \item \alert{Keine Überseztung}
      \end{itemize}
  \end{itemize}
\end{frame}

\begin{frame}
  {Vorbemerkung | Charakteristische Funktionen}
  \onslide<+->
  \onslide<+->
  Mengen, über Funktionen definiert\\
  \Halbzeile
  \begin{itemize}[<+->]
    \item Große Bedeutung von Mengen in formaler Semantik
      \Halbzeile
    \item Charakteristische Funktion von Mengen\\
      \alert{$\mathcal{S}(a) = 1\ iff\ a\in{}S,\ else\ 0$}
    \item CF als Einsortierung in ihre Menge
    \item CF in Mengendefinitionen\\
      \rule{0em}{1em}$S=\{x: x\ mod\ 2=0\}$\\
      \rule{0em}{1em}$\mathcal{S}=f(x)[x\ mod\ 2=0]$\\
      \Halbzeile
    \item \alert{Äquivalenz von Mengendenotation und CF-Dontation}
  \end{itemize}
\end{frame}

\begin{frame}
  {Vorbemerkung | T-Sätze und Funktionsapplikation}
  \onslide<+->
  \onslide<+->
  Funktionsapplikation als allgemeiner Kompositionsmechanismus\\
  \Halbzeile
  \begin{itemize}[<+->]
    \item Etwas umständliche Interpretation mit T-Sätzen\\
      \alert{$\Dem{\ekm{\Sub{S}\ NP\ VP}}{\mMm,g} = 1\ iff\ \Dem{NP}{\mMm,g}\in{}\Dem{VP}{\mMm,g}$}
      \Halbzeile
    \item CF statt Mengen | \alert{Funktion appliziert direkt!}\\
      \begin{itemize}[<+->]
        \item $\Dem{Mary}{\mMm,g}=Mary$ \emph{in \mM}
        \item $\Dem{sleeps}{\mMm,g}$ \emph{be the CF of the set of sleepers in \mM}
        \item \alert{$\Dem{S}{\mMm,g}=\Dem{sleeps}{\mMm,g}(\Dem{Mary}{\mMm,g})$}
        \item \orongsch{Kein Bedarf an T-Sätzen}
      \end{itemize}
  \end{itemize}
\end{frame}

\begin{frame}
  {Vorbemerkung | Funktionen von Mengen zu Mengen}
  \onslide<+->
  \onslide<+->
  Funktionen von Mengen von (Tupeln von) Individuen zu Ausdrücken usw.\\
  \Halbzeile
  \begin{itemize}[<+->]
    \item Funktionen \alert{Definitionsbereich $S_1$} $\rightarrow$ \gruen{Wertebereich $S_2$} | $\gruen{S_2}^{\alert{S_1}}$\\
      \grau{$S_2^{S_1}$ | Die Menge aller Funktionen von $S_1$ zu $S_2$}
      \Halbzeile
    \item Beispiel | Einstellige und Zweistellige Prädikate
      \begin{itemize}[<+->]
        \item $T=\{0,1\}$ | Wahrheitswerte
        \item $D$ | Diskursuniversum (Menge aller Individuen)
        \item $D\times D$ | Menge aller 2-Tupel von Individuen
          \Viertelzeile
        \item \alert{$T^D$} | Menge aller Funktionen von Individuen zu Wahrheitswerten\\
          \grau{Menge der CFs aller einstelligen Prädikate}
          \Viertelzeile
        \item \alert{$T^{D\times D}$} | Menge aller Funktionen von 2-Tupeln von Individuen zu Wahrheitswerten\\
          \grau{Menge der CFs aller zweistelligen Prädikate}
      \end{itemize}
  \end{itemize}
\end{frame}

\section{Getypte Sprachen}

\begin{frame}
  {Kein Bedarf an Phrasenkategorien}
  \onslide<+->
  \onslide<+->
  Logik hat bereits \alert{Typensysteme}, um Syntax zu strukturieren!\\
  \Halbzeile
  \begin{itemize}[<+->]
    \item \alert{$L_{Type}$} | Prädikatenlogik \alert{$L_1$ plus Typen}
    \item \alert{Typen} | Semantisch fundierte Klassen von Ausdrücken
      \Halbzeile
    \item Einfache Typen
      \begin{itemize}[<+->]
        \item Terme | $\ram{e}$
        \item Wffs\slash Formeln | $\ram{t}$ | \orongsch{Ersetzt Startsymbol \alert{S} der PSG!}
      \end{itemize}
      \Halbzeile
    \item Komplexe\slash \alert{funktionale} Typen
      \begin{itemize}[<+->]
        \item Einstellige Prädikate | $\ram{e,t}$
        \item Zweistellige Prädikate | $\ram{e,\ram{e,t}}$
      \end{itemize}
      \Halbzeile
    \item Allgemein | $\ram{\sigma,\tau}$-Ausdrücke denotieren Funktionen von Denotaten\\
      von $\ram{\sigma}$-Ausdrücken zu Denotaten von $\ram{\tau}$-Ausdrücken.
  \end{itemize}
\end{frame}

\begin{frame}
  {Modell | Denotate getypter Ausdrücke}
  \onslide<+->
  \onslide<+->
  Homogenes Diskursuniversum $D$ (auch $U$ und bei \citealt{DowtyEa1981} $A$)\\
  \Halbzeile
  \begin{itemize}[<+->]
    \item Allgemein $D_{\alpha}$ | Menge von Denotaten von Ausdrücken des Typs $\alpha$
      \Halbzeile
    \item Einfache Typen
      \begin{itemize}[<+->]
        \item \alert{$D_{\ram{e}} = U$}
        \item \alert{$D_{\ram{t}} = \{0,1\}$}
      \end{itemize}
      \Halbzeile
    \item Komplexe Typen | Rekursiv definierte Denotate
      \begin{itemize}[<+->]
        \item 
          \item Allgemein | \alert{$D_{\ram{\alpha,\beta}}=D_{\ram{\beta}}^{D_{\ram{\alpha}}}$}
          \item Einstelliuge Prädikate | $D_{\ram{e,t}}=D_{\ram{t}}^{D_{\ram{e}}}$
          \item Zweistellige Prädikate | $D_{\ram{e,\ram{e,t}}}=(D_{\ram{t}}^{D_{\ram{e}}})^{^{D_{\ram{e}}}}$
      \end{itemize}
      \Halbzeile
    \item Interpretation weiterhin durch $V,g$
  \end{itemize}
\end{frame}

\begin{frame}
  {Komplexe Typen für Funktionen und FA}
  \onslide<+->
  \onslide<+->
  $\ram{\sigma}$-Ausdrücke saturieren $\ram{\sigma,\tau}$-Ausdrücke zu $\ram{\tau}$-Ausdrücken.\\
  \Zeile
  \begin{itemize}[<+->]
    \item Beispiel für Saturierung durch \alert{Funktionsapplikation (FA)}
      \begin{itemize}[<+->]
        \item Wenn \alert{$P$} vom Typ \alert{$\ram{e,\ram{e,t}}$}, \gruen{$Q$} vom Typ \gruen{$\ram{e,t}$} und \orongsch{$x,y$} vom Typ \orongsch{$\ram{e}$}
        \item dann ist $\gruen{Q}(\orongsch{x})$ vom Typ \tuerkis{$\ram{t}$}
        \item und $\alert{P}(\orongsch{x})$ vom Typ \tuerkis{$\ram{e,t}$} sowie $\alert{P}(\orongsch{x})(\orongsch{y})$ vom Typ \tuerkis{$\ram{t}$}
      \end{itemize}
      \Zeile
    \item Funktionale Typen von \alert{Funktoren}
      \begin{itemize}[<+->]
        \item Negation $\neg$ | Typ \alert{$\ram{t,t}$}
        \item Andere Funktoren $\wedge,\vee,\rightarrow,\leftrightarrow$ | Typ \alert{$\ram{t,\ram{t,t}}$}
      \end{itemize}
  \end{itemize}
\end{frame}

\begin{frame}
  {Allgemeine Semantik für getypte Sprachen}
  \onslide<+->
  \onslide<+->
  Wirklich keine T-Sätze mehr!\\
  \Zeile
  \begin{itemize}[<+->]
    \item Semantik für $\ram{e}$-Typen (Terme)\\
      \Viertelzeile
      $\DEM{a_n} = \alert{V}(a_n)$ \\
      $\DEM{x_n} = \alert{g}(x_n)$
      \Halbzeile
    \item Ansonsten nur FA\\
      \bl{$\DEM{\delta(\alpha)} = \DEM{\delta}(\DEM{\alpha})$}
  \end{itemize}
\end{frame}

\begin{frame}
  {Verallgemeinerung und Sprachen höherer Ordnung}
  \onslide<+->
  \onslide<+->
  Sprache höherer Ordnung = Sprache mit \alert{Variablen über höhere Typen $\ram{\sigma,\tau}$}\\
  \Halbzeile
  \begin{itemize}[<+->]
    \item $Type$ ist die Menge aller Typen
      \begin{itemize}[<+->]
        \item \alert{$\ram{e},\ram{t}\in Type$}
        \item Wenn $\ram{\sigma},\ram{\tau}\in Type$, dann \alert{$\ram{\sigma,\tau}\in Type$}
        \item \grau{Nichts sonst ist in $Type$.}
      \end{itemize}
      \Viertelzeile 
    \item $ME$ ist die Menge aller bedeutungsvollen Ausdrücke
      \begin{itemize}[<+->]
        \item $ME_{\sigma}$ ist die Menge der Ausdrücke vom Typ $\sigma$ | $ME=\bigcup ME_{\sigma}$ mit $\sigma\in Type$
        \item $Ty$ ist eine Funktion von Ausdrücken zu ihren Typen | $Ty(a)=\sigma\ iff\ a\in ME_{\sigma}$
      \end{itemize}
      \Halbzeile
    \item Höhere Ordnung | \alert{Variablen über Ausdrücke von funktionalen Typen}
      \begin{itemize}[<+->]
        \item $P_{\ram{e,t}}$ und $Q_{\ram{e,\ram{e,t}}}$ | Bekannte \alert{Konstanten} höherer (=funktionaler) Typen
        \item Parallel \alert{$v_{n_{\ram{e,t}}}$} | Die \alert{n-te Variable über einstellige Prädikate}
          \Viertelzeile
        \item Damit möglich \gruen{$M=\{v_{1_{\ram{e,t}}}:\dem{v_{1_{\ram{e,t}}}(m)}=1\}$}\\
          Wenn $\dem{m}=Maria$, dann ist $M$ die Menge von Marias Eigenschaften!
      \end{itemize}
  \end{itemize}
\end{frame}

\begin{frame}
  {Systematische Interpretation zur systematischen Syntax}
  \onslide<+->
  \onslide<+->
  Zusammenfassung | Die Semantik reduziert sich auf FA und Variablenauswertung.\\
  \Halbzeile
  \begin{itemize}[<+->]
    \item Interpretation von Termen und Funktionsausdrücken
      \begin{itemize}[<+->]
        \item Nicht-logische Konstanten | \alert{$\alpha$: $\DEM{\alpha}=V(\alpha)$}
        \item Variablen | \alert{$\alpha$: $\DEM{\alpha}=V(\alpha)$}
        \item Wenn $\alpha\in\ram{a,b}$ und $\beta\in{}a$ dann $\DEM{\alpha(\beta)}=\DEM{\alpha}(\DEM{\beta})$
      \end{itemize}
      \Halbzeile
    \item Logische Konstanten (Typen $\ram{t,t}$ und $\ram{t,\ram{t,t}}$) denotieren Funktionen in $\{0,1\}$.
      \Halbzeile
    \item Quantoren
      \begin{itemize}[<+->]
        \item Für Variable $v_{1_{\ram{\orongsch{\alpha}}}}$ und Wff $\phi\in{}ME_t$ ist $\DEM{\alert{(\forall{}v_1)\phi}}=1$ gdw
        \item[ ] für alle $a\in{}D_{\orongsch{\alpha}}$ $\Dem{\phi}{\mMm,g[a/v_1]}=1$
       \Viertelzeile 
     \item Für Variable $v_{1_{\ram{\orongsch{\alpha}}}}$ und Wff $\phi\in{}ME_t$ ist $\DEM{\alert{(\exists{}v_1)\phi}}=1$ gdw
        \item[ ] für mindestens ein $a\in{}D_{\orongsch{\alpha}}$ $\Dem{\phi}{\mMm,g[a/v_1]}=1$
      \end{itemize}
  \end{itemize}
\end{frame}

\begin{frame}
  {Beispiel | Quantifikation über Prädikate}
  \onslide<+->
  \onslide<+->
  \alert{$\forall{}v_{0_{\ram{e,t}}}\ekm{v_{0_{\ram{e,t}}}(j)\rightarrow v_{0_{\ram{e,t}}}(d)}$}
  \Halbzeile
  \begin{itemize}[<+->]
    \item Eine quantifizierbare Variable vom Typ $\ram{e,t}$ | \alert{$v_{0_{\ram{e,t}}}$}
    \item Zwei Individuenkonstanten | $j,d\in ME_{\ram{e}}$ \grau{\zB John und Dorothy}
    \item \textit{Für alle einstelligen Prädikate gilt: Wenn $j$ die vom Prädikat\\
      beschriebene Eigenschaft hat, hat $d$ auch diese Eigenschaft.}
      \Halbzeile
    \item Wann ist diese Wff wahr?
      \begin{itemize}[<+->]
        \item Wenn $j$ und $d$ \alert{alle benennbaren Eigenschaften teilen}?
        \item Eine Eigenschaft jedes Objekts |\\
          \alert{CF der Menge $\{x:x\ is\ the\ sole\ member\ of\ this\ set\}$} (\textit{union set})
        \item Einzige Möglichkeit für Wahrheit der Wff also \gruen{j=d}
      \end{itemize}
  \end{itemize}
\end{frame}

\begin{frame}
  {Beispiel | Wortbildung mit Präfix \textit{non}}
  \onslide<+->
  \onslide<+->
  \textit{non} in Sätzen wie \textit{This function is \alert{non-continuous}.}\\
  \Halbzeile
  \begin{itemize}[<+->]
    \item Produktives Suffix im Englischen, wie \textit{nicht-} im Deutschen
    \item Bedeutungsbeitrag | \alert{Invertiert die CF} eines Adjektivs
    \item \alert{Komplementbildung} der Ursprungsmenge in $D_{\ram{e,t}}$
      \Halbzeile
    \item Syntax und Semantik von \textit{non}
      \Viertelzeile
      \begin{itemize}[<+->]
        \item Adjektiv \textit{continuous} | Typ $\ram{e,t}$
        \item Typ von \textit{non} | In: Adjektiv \slash Out: Adjektiv | \alert{$\ram{\ram{e,t},\ram{e,t}}$}
          \Viertelzeile
        \item $\DEM{non}=h$ s.\,t.\ \alert{$h\in D_{\ram{\ram{e,t},\ram{e,t}}}$} and for every $k\in{}D_{\ram{e,t}}$ and every $d\in D_{\ram{e}}$\\
          \gruen{$(h(k))(d) = 1$ iff $k(d)=0$ and $(h(k))(d) = 0$ iff $k(d)=1$}

      \end{itemize}
  \end{itemize}
\end{frame}

\begin{frame}
  {Beispiel | Argumentunterdrückung}
  \onslide<+->
  \onslide<+->
  Optionale Argumente wie in \alert{\textit{I eat.}} oder \alert{\textit{Vanity kills.}}\\
  \Halbzeile
  \begin{itemize}[<+->]
    \item Zweistellige Verben wie \textit{eat} in \alert{$ME_{\ram{e,\ram{e,t}}}$}
      \Halbzeile
    \item Aus einem zweistelligen Verb ein einstelliges machen
      \Viertelzeile
      \begin{itemize}[<+->]
        \item Phonologisch leere lexikalische Konstante | \alert{$R_O\in ME_{\ram{\ram{e,\ram{e,t}},\ram{e,t}}}$}\\
          \grau{\footnotesize Ähnlich wie lexikalische Regeln in HPSG}
          \Viertelzeile
        \item Semantik | $\DEM{R_0}=h$ s.\,t.\ \alert{$h\in D_{\ram{\ram{e,\ram{e,t}},\ram{e,t}}}$} and for all $k\in D_{\ram{e,\ram{e,t}}}$ and all $d\in D_{\ram{e}}$\\
          \gruen{$(h(k))(d)=1$ iff there is some $d\Prm\in D_{\ram{e}}$ s.\,t.\ $k(d\Prm)(d)=1$}
      \end{itemize}
  \end{itemize}
\end{frame}

\section{$\lambda$-Sprachen}

\begin{frame}
  {Sie kennen bereits $\lambda$-Abstraktionen!}
  \onslide<+->
  \onslide<+->
  \centering 
  \Large
  Was bedeutet \alert{$f(x)=3x^2+5x+8$} ?\\
  \onslide<+->
  \Zeile
  \raggedright
  \normalsize
  \begin{itemize}[<+->]
    \item \alert{$3x^2+5x+8$} ist eine Wff mit einer ungebundenen Variable.
    \item Die Variable wird gebunden und die Wff \alert{wird damit zur Funktion}\\
      \grau{\footnotesize $x$ wird zur Eingabevariable und muss bei Anwendung durch Eingabewert ersetzt werden.}
    \item Außerdem wird die Funktion \alert{f genannt}.
  \end{itemize}
  \onslide<+->
  \Halbzeile
  \centering 
  \Large
  In $\lambda$-Notation: \gruen{$f\stackrel{def}{=}\lambda x\ekm{3x^2+5x+8}$}
\end{frame}

\begin{frame}
  {Nur ein neuer Variablenbinder}
  \onslide<+->
  \onslide<+->
  \gruen{Mit $\lambda$ bildet man ad hoc anonyme Funktionen.}\\
  \Halbzeile
  \begin{itemize}[<+->]
    \item Abstraktion über Wffs beliebiger Komplexität
    \item $\lambda$-Bindung der Variable | Gebundene Variable als \alert{Eingabevariable der Funktion}
    \item Sehr ähnlich wie Listendefinition
      \begin{itemize}[<+->]
        \item Menge | \alert{$\{x: x\ mod\ 2=0\}$} \grau{| allgemein $\{x:\phi\}$}
        \item CF dieser Menge | \alert{$\lambda x\ekm{x\ mod\ 2=0}$} \grau{| allgemein $\lambda x\ekm{\phi}$}
      \end{itemize}
  \end{itemize}
\end{frame}

\begin{frame}
  {Formale Erweiterung von $L_{Type}$}
  \onslide<+->
  \onslide<+->
  Nur wenige Erweiterungen in $L_{Type}$\\
  \Halbzeile
  \begin{itemize}[<+->]
    \item Für jede Wff $\phi$ mit \alert{$Ty(\phi)=\ram{t}$} und jede \gruen{$x\in Var$} und jede \orongsch{$a\in Con$}
      \Viertelzeile
      \begin{itemize}[<+->]
        \item Abstraktion | $\phi\ \Longrightarrow\ \gruen{\lambda x}\ekm{\alert{\phi}^{[\orongsch{a}/\gruen{x}]}}$\\
          \grau{\footnotesize Definition $\phi^{[a/x]}$ | Wff $\phi$ in der alle $a$ durch $x$ getauscht wurden}
        \item Anwendung der Funktion ($\lambda$-Konversion) | $\gruen{\lambda x}\ekm{\alert{\phi}^{[\orongsch{a}/\gruen{x}]}}(\orongsch{a})=\alert{\phi}$
      \end{itemize}
      \Halbzeile
    \item Es gilt \alert{$\lambda{}x\ekm{\phi^{a/x}}(a)\equiv{}\phi$} für jede Wff $\phi$, jede $a\in Con$ und jede $x\in{}Var$
    \item $x$ kann von einem \alert{beliebigen Typ $\sigma$} sein.
    \item Es gilt für \alert{$\lambda x\ekm{\phi}$} mit $x\in ME_{\ram{\sigma}}$ stets \alert{\alert{$\phi\in ME_{\ram{t}}$}} sowie \alert{$\lambda x\ekm{\phi}\in ME_{\ram{\sigma,t}}$}
  \end{itemize}
\end{frame}

\begin{frame}
  {Zwei Beispiele}
  \onslide<+->
  \onslide<+->
  Abstraktion über \alert{Individuenvariable} und \alert{Prädikatsvariable}\\
  \Halbzeile
  \begin{itemize}[<+->]
    \item Individuenvariable \alert{$x_{\ram{e}}$} \grau{alternativ $v_{1_{\ram{e}}}$}
      \begin{itemize}[<+->]
        \item \alert{$\lambda{}x_{\ram{e}}\ekm{L(x)}$}
        \item Mit $L$ \zB für \textit{laughs}
        \item Die CF der Menge von Individuen $d\in{}D_{\ram{e}}$ mit Eigenschaft $L$ (alle Lachenden)
        \item Mengendefinition dazu \alert{$\{x: L(x)\}$}
      \end{itemize}
      \Halbzeile
    \item Prädikatsvariable \gruen{$P_{\ram{e,t}}$} \grau{alternativ $v_{1_{\ram{e,t}}}$}
      \begin{itemize}[<+->]
        \item \gruen{$\lambda{}P_{\ram{e,t}}\ekm{P(l)}$}
        \item Mit $l$ \zB für \textit{Horst Lichter}
        \item Die CF aller Eigenschaften $k\in{}D_{\ram{e,t}}$ von $l$ (alle Eigenschaften Horst Lichters)
        \item Mengendefinition dazu \gruen{$\{P: P(l)\}$}
      \end{itemize}
  \end{itemize}
\end{frame}

\begin{frame}
  {Als wäre das jetzt nicht schon klar \ldots}
  \onslide<+->
  \onslide<+->
  Die vollen Regeln aus \citet[102]{DowtyEa1981} (Syn C.10 and Sem 10)\\
  \Halbzeile
    \begin{itemize}[<+->]
      \item If $\alpha\in{}ME_{\alpha}$ and $u\in{}Var_b$, then $\lambda{}u\ekm{\alpha}\in{}ME_{\ram{b,a}}$.
      \item If $\alpha\in{}ME_a$ and $u\in{}Var_b$ then $\DEM{\lambda{}u\ekm{\alpha}}$ is that function $h$ from $D_b$ into $D_a$\\
        ($h\in D_a^{D_b}$) s.\,t.\ for all objects $k$ in $D_b$, $h(k)$ is equal to $\Dem{\alpha}{\mMm,g\ekm{k/u}}$.
    \end{itemize}
  \Halbzeile
\end{frame}

\begin{frame}
  {Konversionen}
  \onslide<+->
  \onslide<+->
  \alert{Konversionen\slash Reduktionen} | Arten, $\lambda$-Ausdrücke umzuschreiben\\
  \Halbzeile
  \begin{itemize}[<+->]
    \item \alert{$\alpha$-Konversion} | Umbenennung von Variablen
      \begin{itemize}[<+->]
        \item \alert{$\lambda x\ekm{\phi}\stackrel{\alpha}{\equiv}\lambda y\ekm{\phi^{\ekm{x/y}}}$} gdw $y$ in $\phi$ nicht vorkommt
      \end{itemize}
      \Halbzeile
    \item \gruen{$\beta$-Reduktion} | Funktionsapplikation
      \begin{itemize}[<+->]
        \item \gruen{$\lambda x\ekm{\phi}(a)\stackrel{\beta}{\equiv}\phi^{\ekm{x/a}}$}
        \item Ausdrucke mit nicht realisierten, aber möglichen $\beta$-Reduktionen: \gruen{$\beta$-Redex}
      \end{itemize}
      \Halbzeile
    \item \tuerkis{$\eta$-Reduktion} | Entfernen von leeren Abstraktionen
      \begin{itemize}[<+->]
        \item \tuerkis{$\lambda x\ekm{F(x)}\stackrel{\eta}{\equiv}F$} gdw $Ty(F)=Ty(\lambda x\ekm{F(x)})$ \grau{(und $x$ nicht frei in F ist)}
        \item Ausdruck mit nicht realisierten, aber möglichen $\eta$-Reduktionen: \gruen{$\eta$-Redex}
        \item Mäßige Semantiker | $\eta$-Redex-Fetisch mit \orongsch{$\lambda x\lambda y\lambda z\ekm{gibt\Prm(x,y,z)}$} usw.
      \end{itemize}
  \end{itemize}
\end{frame}

\begin{frame}
  {The \emph{non} example revised \citep[104]{DowtyEa1981}}
  \onslide<+->
  \onslide<+->
  Das können Sie jetzt nachvollziehen!\\
  \Zeile
  \begin{itemize}[<+->]
    \item $\forall{}x\forall{}v_{0^{\ram{e,t}}}\ekm{(\mathbf{non}(v_{0_{\ram{e,t}}}))(x)\leftrightarrow\neg{}(v_{0_{\ram{e,t}}}(x))}$
    \item $\forall{}v_{0_{\ram{e,t}}}\ekm{\lambda{}x\ekm{(\mathbf{non}(v_{0_{\ram{e,t}}}))(x)}=\lambda{}x\ekm{\neg{}(v_{0_{\ram{e,t}}}(x))}}$
    %{\footnotesize (since $\lambda{x}\ekm{\mathbf{non}(v)(x)}$ is unnecessarily abstract/$\eta$ reduction)}
    \item $\lambda{}v_{0_{\ram{e,t}}}\ekm{\mathbf{non}(v_{0_{\ram{e,t}}})=\lambda{}v_{0_{\ram{e,t}}}\ekm{\lambda{}x\ekm{\neg{}(v_{0_{\ram{e,t}}}(x))}}}$
    \item %{\footnotesize and since that is about all assignments for $\lambda{}v_{0_{\ram{e,t}}}$:}\\
    $\mathbf{non}=\lambda{}v_{0_{\ram{e,t}}}\ekm{\lambda{x}\ekm{\neg{}v_{0_{\ram{e,t}}}(x)}}$
  \end{itemize}
\end{frame}

\begin{frame}
  {Example with \textit{non}}
  \onslide<+->
  \onslide<+->
  \textit{Mary is non-belligerent.}\\
  \Viertelzeile
  \grau{Translate `belligerent' as $c_{0_{\ram{e,t}}}$, `Mary' as $c_{0_{\ram{e}}}$, ignore the copula.}\\
  \onslide<+->
  \Zeile
  \centering 
  \scalebox{0.8}{\begin{forest}
    [$\neg{}c_{0_{\ram{e,t}}}(c_{0_{\ram{e}}})$ \bl{(by $\lambda$ conv.)}
      [$\lambda{v_{0_{\ram{e}}}}\ekm{\neg{}c_{0_{\ram{e,t}}}(v_{0_{\ram{e}}})}(c_{0_{\ram{e}}})$ \bl{(by FA)}
        [$c_{0_{\ram{e}}}$]
        [$\lambda{v_{0_{\ram{e}}}}\ekm{\neg{}c_{0_{\ram{e,t}}}(v_{0_{\ram{e}}})}$ \bl{(by $\lambda$ conv.)}
          [$\lambda{}v_{0_{\ram{e,t}}}\ekm{\lambda{v_{0_{\ram{e}}}}\ekm{\neg{}v_{0_{\ram{e,t}}}(v_{0_{\ram{e}}})}}(c_{0_{\ram{e,t}}})$ \bl{(by FA)}
            [$\lambda{}v_{0_{\ram{e,t}}}\ekm{\lambda{v_{0_{\ram{e}}}}\ekm{\neg{}v_{0_{\ram{e,t}}}(v_{0_{\ram{e}}})}}$]
            [$c_{0_{\ram{e,t}}}$]
          ]
        ]
      ]
    ]
  \end{forest}}
\end{frame}


\section{Ausblick auf Quantifikation bei Montague}

\begin{frame}
  {Quantifizierte NPs bei Montague}
  \onslide<+->
  \onslide<+->
  Können referentielle und quantifizierte NPs denselben Typ haben?\\
  \Halbzeile
  \begin{itemize}[<+->]
    \item Quantoren-NP-Syntax | Wie die referentieller NPs
    \item Quantoren-NP-Semantik | Wie die von prädikatenlogischen Quantoren
      \Halbzeile
    \item Erstmal nicht aufregend bzw.\ erwartbar in $L_{Type}$
      \Viertelzeile
      \begin{itemize}[<+->]
        \item \textit{Every student walks.}: \alert{$\forall{v_{0_{\ram{e}}}}\ekm{c_{0_{\ram{e,t}}}(v_{0_{\ram{e}}})\rightarrow{}c_{1_{\ram{e,t}}}(v_{0_{\ram{e}}})}$}
        \item \textit{Some student walks.}: \alert{$\forall{v_{0_{\ram{e}}}}\ekm{c_{0_{\ram{e,t}}}(v_{0_{\ram{e}}})\wedge{}c_{1_{\ram{e,t}}}(v_{0_{\ram{e}}})}$}
      \end{itemize}
  \end{itemize}
\end{frame}

\begin{frame}
  {Ein höherer Typ}
  \onslide<+->
  \onslide<+->
  Die Macht höherstufiger $\lambda$-Sprachen\\
  \Zeile 
  \begin{itemize}[<+->]
    \item Versuchen Sie, diese Ausdrücke zu verstehen
      \begin{itemize}[<+->]
        \item \bl{$\lambda{}v_{0_{\ram{e,t}}}\forall{v_{0_{\ram{e}}}}\ekm{c_{0_{\ram{e,t}}}(v_{0_{\ram{e}}})\rightarrow{}v_{0_{\ram{e,t}}}(v_{0_{\ram{e}}})}$}
        \item \bl{$\lambda{}v_{0_{\ram{e,t}}}\exists{v_{0_{\ram{e}}}}\ekm{c_{0_{\ram{e,t}}}(v_{0_{\ram{e}}})\wedge{}v_{0_{\ram{e,t}}}(v_{0_{\ram{e}}})}$}
      \end{itemize}
    \item Denken Sie daran:
      \begin{itemize}[<+->]
        \item $c_{0_{\ram{e,t}}}$ | Das Prädikat für \textit{students}
        \item $\lambda{}v_{0_{\ram{e,t}}}$ | Variable über einstellige Prädikate
        \item $v_{0_{\ram{e}}}$ | Variable über Individuen
      \end{itemize}
      \Halbzeile
    \item \alert{Funktionen zweiter Ordnung} (Prädikate als Eingabewerte) 
      \Halbzeile
      \onslide<+->
    \item \gruen{CFs der Mengen von Prädikaten die auf alle\slash einige Studierende zutreffen}
  \end{itemize}
\end{frame}

\begin{frame}
  {Kombination mit Prädikat}
  \onslide<+->
  \onslide<+->
  \centering 
  \scalebox{1}{\begin{forest}
    [$\exists{v_{0_{\ram{e}}}}\ekm{c_{0_{\ram{e,t}}}(v_{0_{\ram{e}}})\wedge{}c_{1_{\ram{e,t}}}(v_{0_{\ram{e}}})}$ \bl{(by $\lambda$ conv.)}
      [$\lambda{}v_{0_{\ram{e,t}}}\exists{v_{0_{\ram{e}}}}\ekm{c_{0_{\ram{e,t}}}(v_{0_{\ram{e}}})\wedge{}v_{0_{\ram{e,t}}}(v_{0_{\ram{e}}})}(c_{1_{\ram{e,t}}})$ \bl{(by FA)}
        [$\lambda{}v_{0_{\ram{e,t}}}\exists{v_{0_{\ram{e}}}}\ekm{c_{0_{\ram{e,t}}}(v_{0_{\ram{e}}})\wedge{}v_{0_{\ram{e,t}}}(v_{0_{\ram{e}}})}$]
        [$c_{1_{\ram{e,t}}}$]
      ]
    ]
  \end{forest}}
\end{frame}

\section{Aufgaben}

\begin{frame}
  {Aufgaben I}
  Überlegen Sie, wie die Semantik folgender Sätze in einer $\lambda$-Sprache kompositional modelliert werden kann. Sie können ein vollständiges Fragment entwickeln, müssen es aber nicht. Übersetzen Sie gerne auch einfach einzelne relevante Ausdrücke "`plausibel"' in Prädikatenlogik höherer Ordnung mit $\lambda$-Abstraktion. Die relevanten Konstituenten, bei denen Sie über die Vorteile einer $\lambda$-Sprache nachdenken sollten, sind jeweils farblich hervorgehoben.\\
  \Halbzeile
  \begin{enumerate}
    \item \textit{\alert{Martin und Maria} laufen.}
    \item \textit{Maria \alert{schwimmt oder taucht}.}
    \item \textit{Eine Linguistin \alert{schwimmt und läuft}.}
    \item \textit{Martin \alert{macht irgendwas}.}
    \item \textit{Das Buch \gruen{brennt} \alert{auf dem Tisch}}.
    \item \textit{Das Buch \gruen{liegt} \alert{auf dem Tisch}}.
    \item \textit{Herr Webelhuth legt das Buch \alert{auf oder neben den Tisch}.}
  \end{enumerate}
\end{frame}


\begin{frame}
  {Aufgaben II}
  Versuchen Sie, die Affixe bzw. den Kompositionsvorgang in folgenden Wortpaaren in einer $\lambda$-Prädikatenlogik höherer Ordnung zu modellieren. (Das gleiche wie auf der letzten Folie, nur für Wortbildung statt für Syntax.) Das ist längst nicht alles trivial, und einiges wird nicht funktionieren, je nachdem wie genau Sie es nehmen.\\
  \Halbzeile
  \begin{enumerate}
    \item \textit{Linguist -- Linguist\alert{in}}\\
      \grau{\footnotesize mit und ohne "`generische"' Form}
    \item \textit{streichen -- \alert{rot}streichen}\\
      \grau{\footnotesize Versuchen Sie, die temporalen\slash aspektuellen Besonderheiten irgendwie zu umschiffen.}
    \item \textit{gehen -- \alert{be}gehen}
    \item \textit{schreiben -- \alert{ver}schreiben}
    \item \textit{lesen -- Les\alert{er}}
    \item \textit{Leser -- \alert{Karten}leser}
  \end{enumerate}
\end{frame}

\begin{frame}
  {Aufgaben III}
  Wie kann man Passiv in $L_{Type}$ modellieren?
  \Halbzeile
  \begin{enumerate}
    \item Modellieren Sie zunächst die Semantik des passivierten Verbs auf Basis einer Semantik des Aktivverbs.
    \item Versuchen Sie, für das Deutsche ein minimales Fragment im Stil von $L_{Type}$ zu bauen, das die folgenden beiden Sätze modelliert:
      \begin{itemize}[<+->]
        \item Maria grüßt Martin.
        \item Martin wird gegrüßt.
      \end{itemize}
    \item Geben Sie ein minimales Modell an, in dem beide Sätze wahr sind.
    \item Geben Sie ein minimales Modell an, in dem nur der Aktivsatz, nicht aber der Passivsatz wahr ist.
  \end{enumerate}
\end{frame}

