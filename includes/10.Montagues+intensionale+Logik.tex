%\documentclass{beamer}
\documentclass[handout]{beamer}

\usepackage{beamerthemesplit,
            graphicx,
            color,
            colortbl,
            rotate,
            setspace,
            pifont,
            amsmath,
            amsfonts,
            amssymb,
            rotate,
            stmaryrd,
            xyling}

\title{Semantics{ }\\(9) Montague's Intensional Logic}
\author{Roland Sch\"afer (University of G\"ottingen)}
\date{Summer Term 2005 (July 06)}

\newcommand{\den}[1]{{$\llbracket${}#1$\rrbracket$}}
\newcommand{\dem}[1]{{\llbracket{}#1\rrbracket}}
\newcommand{\Dem}[2]{\llbracket{}#1\rrbracket^{#2}}
\newcommand{\DEM}[1]{\llbracket{}#1\rrbracket^{\mathcal{M},g}}
\newcommand{\DEMM}[1]{\llbracket{}#1\rrbracket^{\mathcal{M},w,i,g}}
\newcommand{\DEMI}[1]{\llbracket{}#1\rrbracket_{\not c}^{\mathcal{M},g}}
%\newcommand{\ram}[1]{{\langle{}#1\rangle}}
\newcommand{\ram}[1]{{\langle{}#1\rangle}}
\newcommand{\ek}[1]{$\left[$#1$\right]$}
\newcommand{\lk}[0]{$\left[$}
\newcommand{\rk}[0]{$\right]$}
\newcommand{\ekm}[1]{\left[{}#1\right]}
\newcommand{\bl}[1]{\textcolor{blue}{#1}}
\newcommand{\rot}[1]{\textcolor{red}{#1}}
\newcommand{\gr}[1]{\textcolor{gray}{#1}}
\newcommand{\gn}[1]{\textcolor{green}{#1}}
\newcommand{\lgr}[1]{\textcolor[gray]{0.9}{#1}}
\newcommand{\mM}[0]{$\mathcal{M}$}
\newcommand{\mMm}[0]{\mathcal{M}}
\newcommand{\Prn}[0]{$^{\prime}$}
\newcommand{\Prm}[0]{^{\prime}}
\newcommand{\up}[0]{$\ \hat{ }\ $}
\newcommand{\down}[0]{$\ \check{ }\ $}
\newcommand{\upm}[0]{\ \hat{ }\ }
\newcommand{\downm}[0]{\ \check{ }\ }
\newcommand{\PRI}[0]{^{\prime}}
\newfont{\dslash}{dsrom12}

\begin{document}

\frame{\titlepage}

\section{New types and up/down}
\subsection{Denoting intensions}
\frame{\frametitle{Beyond truth functionality}
\begin{itemize}
  \item<1-> $\DEMM{\phi}$ and $\DEMM{\mathbf{P}}$ don't truth conditionally determine $\DEMM{\mathbf{P}\phi}$
  \item<2-> \textit{Iceland was once covered with a glacier.}
  \item<3-> \textbf{F}, \textbf{B}, $\mathbf{\lozenge}$, $\mathbf{\Box}$ are not fully truth functional
  \item<4-> Leibnitz Law of identity of individuals for logics containing `=' failing in opaque contexts
  \item<5-> `former', `alleged', etc. are not intersective adjectives like `red'
  \item<6-> Frege: sometimes expressions \textbf{denote a sense}
  \item<7-> again: individual concepts (variable function on indices) vs. names (constant)
\end{itemize}
}

\frame{\frametitle{$\DEMI{\alpha}$}
\begin{itemize}
  \item<1-> intension relative to models
  \item<2-> for a name $d$: $\DEMI{d}=\ekm{\begin{array}{lll}
                                              \ram{w_1,t_1} & \rightarrow & b\\
                                              \ram{w_2,t_1} & \rightarrow & b\\
                                              \ram{w_1,t_2} & \rightarrow & b\\
                                              \ram{w_2,t_2} & \rightarrow & b\\
                                              \ram{w_1,t_3} & \rightarrow & b\\
                                              \ram{w_2,t_3} & \rightarrow & b\\
                                           \end{array}}$
\end{itemize}
}

\frame{\frametitle{$\DEMI{\alpha}$}
\begin{itemize}
  \item<1-> for an individual concept denoting expression $m$:
  \item<2->  $\DEMI{m}=\ekm{\begin{array}{lll}
                                              \ram{w_1,t_1} & \rightarrow & a\\
                                              \ram{w_2,t_1} & \rightarrow & c\\
                                              \ram{w_1,t_2} & \rightarrow & b\\
                                              \ram{w_2,t_2} & \rightarrow & c\\
                                              \ram{w_1,t_3} & \rightarrow & c\\
                                              \ram{w_2,t_3} & \rightarrow & b\\
                                           \end{array}}$
\end{itemize}
}

\frame{\frametitle{$\DEMI{\alpha}$}
\begin{itemize}
  \item<1-> for a one place predicate $B$:
  \item<2->  $\DEMI{B}=\ekm{\begin{array}{lll}
                                              \ram{w_1,t_1} & \rightarrow & \{a,b\}\\
                                              \ram{w_2,t_1} & \rightarrow & \{b,c\}\\
                                              \ram{w_1,t_2} & \rightarrow & \{a,c\}\\
                                              \ram{w_2,t_2} & \rightarrow & \{a\}\\
                                              \ram{w_1,t_3} & \rightarrow & \{b,c\}\\
                                              \ram{w_2,t_3} & \rightarrow & \{a,b,c\}\\
                                           \end{array}}$
\end{itemize}
}

\frame{\frametitle{Intensions of formulas}
\begin{itemize}
  \item<1-> formula $\phi$: $\DEMI{\phi}$ is a function from indices to truth values
  \item<2-> {\footnotesize $\DEMI{B(m)}=\ekm{\begin{array}{lll}
                                              \ram{w_1,t_1} & \rightarrow & 1\\
                                              \ram{w_2,t_1} & \rightarrow & 1\\
                                              \ram{w_1,t_2} & \rightarrow & 0\\
                                              \ram{w_2,t_2} & \rightarrow & 0\\
                                              \ram{w_1,t_3} & \rightarrow & 1\\
                                              \ram{w_2,t_3} & \rightarrow & 1\\
                                            \end{array}}$ }
  \item<3-> {\footnotesize $\DEMI{B(n)}=\ekm{\begin{array}{lll}
                                              \ram{w_1,t_1} & \rightarrow & 0\\
                                              \ram{w_2,t_1} & \rightarrow & 1\\
                                              \ram{w_1,t_2} & \rightarrow & 1\\
                                              \ram{w_2,t_2} & \rightarrow & 0\\
                                              \ram{w_1,t_3} & \rightarrow & 1\\
                                              \ram{w_2,t_3} & \rightarrow & 1\\
                                           \end{array}}$ }
\end{itemize}
}

\frame{\frametitle{Intensions of formulas}
\begin{itemize}
  \item<1-> again, the proposition $\DEMI{Bm}$ is a set of indices ($\ram{w_i,t_j}$)
  \item<2-> from the extension at all indices, compute the intension
  \item<3-> $\DEMI{\alpha}(\ram{w_i,t_j})=\Dem{\alpha}{\mMm,w_i,t_j,g}$
\end{itemize}
}

\frame{\frametitle{Intensions of variables}
\begin{itemize}
  \item<1-> constant function on indices
  \item<2-> will play a great role, so remember!
  \item<3-> $\DEMI{u}(\ram{w_i,t_j})=g(u)$
\end{itemize}
}

\subsection{Technical devices}

\frame{\frametitle{What expressions denote}
\begin{itemize}
  \item<1-> sometimes expressions denote individuals, sets of individuals, truth values\ldots
  \item<2-> and sometimes \bl{they denote intensions} (functions)
  \item<3-> \gr{alternatively: introduce rules which access an expression's extension/intension as appropriate}
\end{itemize}
}

\frame{\frametitle{Up and down}
\begin{itemize}
  \item<1-> Church/Montague: \bl{for an extension-denoting expression $\alpha$, $\hat{ }\alpha$ denotes $\alpha$'s intension}
  \item<2-> $\DEMM{\ \hat{ }Bm}=\DEMI{Bm}$
  \item<3-> $\alpha$ and $\ \hat{ }\alpha$ are just \bl{denoting expressions}
  \item<4-> for an intension-denoting expression $\alpha$: $\DEMM{\ \check{ }\alpha}=\Dem{\alpha}{\mMm,g}(\ram{w,t})$
\end{itemize}
}

\frame{\frametitle{Down-up and up-down}
\begin{itemize}
  \item<1-> observe: $\DEMM{\ \check{ }\ \hat{ }\alpha}=\DEMM{\alpha}$ for any $\ram{w,t}$
  \item<2-> but not always: $\DEMM{\ \hat{ }\ \check{ }\alpha}=\DEMM{\alpha}$ for any $\ram{w,t}$
  \item<3-> can easily be the case for intension-denoting expressions
\end{itemize}
}

\frame{\frametitle{Non-equality}
\begin{itemize}
  \item<1-> $k$' intension: {\tiny $\DEMI{k}=\ekm{\begin{array}{lll}
                                           \ram{w_1,t_1} & \rightarrow & \ekm{\begin{array}{lll}
                                                                                \ram{w_1,t_1} & \rightarrow & a \\
                                                                                \ram{w_1,t_2} & \rightarrow & a \\
                                                                                \ram{w_2,t_1} & \rightarrow & a \\
                                                                                \ram{w_2,t_2} & \rightarrow & a \\
                                                                              \end{array}} \\
                                           \ram{w_1,t_2} & \rightarrow & \ekm{\begin{array}{lll}
                                                                                \ram{w_1,t_1} & \rightarrow & a \\
                                                                                \ram{w_1,t_2} & \rightarrow & b \\
                                                                                \ram{w_2,t_1} & \rightarrow & c \\
                                                                                \ram{w_2,t_2} & \rightarrow & d \\
                                                                              \end{array}} \\
                                           \ram{w_2,t_1} & \rightarrow & \ekm{\begin{array}{lll}
                                                                                \ram{w_1,t_1} & \rightarrow & c \\
                                                                                \ram{w_1,t_2} & \rightarrow & b \\
                                                                                \ram{w_2,t_1} & \rightarrow & d \\
                                                                                \ram{w_2,t_2} & \rightarrow & a \\
                                                                              \end{array}} \\
                                           \ram{w_2,t_2} & \rightarrow & \ekm{\begin{array}{lll}
                                                                                \ram{w_1,t_1} & \rightarrow & c \\
                                                                                \ram{w_1,t_2} & \rightarrow & d \\
                                                                                \ram{w_2,t_1} & \rightarrow & a \\
                                                                                \ram{w_2,t_2} & \rightarrow & b \\
                                                                              \end{array}} \\
                                        \end{array}}$}
\end{itemize}
}

\frame{\frametitle{Non-equality}
\begin{itemize}
  \item<1-> $k$' extension (e.g., at $\ram{w_1,t_2}$): $\DEMI{k}(\ram{w_1,t_2})=$
  \item<2-> {\tiny $\Dem{k}{\mMm,w_1,t_2,g}=\ekm{\begin{array}{lll}
                                            \ram{w_1,t_1} & \rightarrow & a \\
                                            \ram{w_1,t_2} & \rightarrow & b \\
                                            \ram{w_2,t_1} & \rightarrow & c \\
                                            \ram{w_2,t_2} & \rightarrow & d \\
                                          \end{array}}$}
  \item<3-> {\tiny however: $\Dem{\ \hat{ }\ \check{ }\ k}{\mMm,w_1,t_2,g}=\ekm{\begin{array}{lll}
                                            \ram{w_1,t_1} & \rightarrow & a \\
                                            \ram{w_1,t_2} & \rightarrow & b \\
                                            \ram{w_2,t_1} & \rightarrow & d \\
                                            \ram{w_2,t_2} & \rightarrow & b \\
                                          \end{array}}$ }
  \item<3-> {\tiny since: $\Dem{\ \check{ }\ k}{\mMm,w_1,t_1,g}=a$\\
                          $\Dem{\ \check{ }\ k}{\mMm,w_1,t_2,g}=b$\\
                          $\Dem{\ \check{ }\ k}{\mMm,w_2,t_1,g}=d$\\
                          $\Dem{\ \check{ }\ k}{\mMm,w_2,t_2,g}=b$\\}
\end{itemize}
}

\section{The IL of PTQ}
\subsection{Syntax}
\frame{\frametitle{A typed higher order $\lambda$ language with = and $\ \hat{ }\ $/$\ \check{ }\ $}
\begin{itemize}
  \item<1-> $\neg,\ \wedge,\ \vee,\ \rightarrow,\ \leftrightarrow,\ \mathbf{F},\ \mathbf{P},\ \Box,\ =$ (syncategorematically)
  \item<2-> $t,e\in Type$ ($Con_{type}$, $Var_{type}$)
  \item<3-> if $a,b\in Type$, then $\ram{a,b}\in Type$
  \item<4-> if $a\in Type$, then $\ram{s,a}\in Type$
  \item<5-> $s\not\in Type$
\end{itemize}
}

\frame{\frametitle{Meaningful expressions}
  \begin{itemize}
    \item<1-> $ME_{type}$
    \item<2-> abstraction: if $\alpha\in ME_a$, $\beta\in Var_b$, $\lambda\beta\alpha\in ME_{\ram{b,a}}$
    \item<3-> FA: if $\alpha\in ME_{\ram{a,b}}$, $\beta\in ME_{a}$ then $\alpha(\beta)\in ME_b$
    \item<4-> if $\alpha,\beta\in ME_a$ then $\alpha=\beta\in ME_t$
  \end{itemize}
}

\frame{\frametitle{Interpretations of $\ \hat{ }\ $ and $\ \check{ }\ $}
  \begin{itemize}
    \item<1-> if $\alpha\in ME_a$ then $\upm\alpha\in ME_{s,a}$
    \item<2-> if $\alpha\in ME_{\ram{s,a}}$ then $\downm\alpha\in ME_{a}$
    \item<3-> \begin{tabular}{l|l|l}
                type & variables & constants \\
                \hline
                $e$ & $x,y,z$ & $a,b,c$ \\
                $\ram{s,e}$ & $x,y,z$ & $-$ \\
                $\ram{e,t}$ & $X,Y$ & $walk',A,B$ \\
                $\ram{\ram{s,e},t}$ & $Q$ & $rise',change'$ \\
                $\ram{s,\ram{e,t}}$ & $P$ & $-$ \\
                $\ram{e,e}$ & $P$ & Sq \\
                $\ram{e,\ram{e,t}}$ & $R$ & $Gr,K$ \\
                $\ram{e,\ram{e,e}}$ & $-$ & $Plus$ \\
              \end{tabular}
  \end{itemize}
}

\subsection{Semantics}
\frame{\frametitle{The model}
  \begin{itemize}
    \item<1-> $\ram{A,W,T,<,F}$
    \item<2-> $D_{\ram{a,b}}={D_b}^{D_a}$
    \item<3-> $D_{\ram{s,a}}={D_a}^{W\times T}$
    \item<4-> `senses' = \textbf{possible} denotations
    \item<5-> actual intensions chosen from the set of senses
    \item<6-> now: F(expression)=intenstion (itself a function)
    \item<7-> s.t. intension(index)=extention
    \item<8-> \gr{instead of: F(expression)(index)=extemsion}
  \end{itemize}
}

\frame{\frametitle{Some interpretations}
  \begin{itemize}
    \item<1-> $\DEMM{\lambda u\alpha},\ u\in Var_b,\ \alpha\in ME_a$ is a function $h$ with domain $D_b$ s.t. $x\in D_b$, $h(x)=\Dem{\alpha}{\mMm,w,t,g^{\prime}}$ with $g^{\prime}$ exactly like $g$ except $g^{\prime}(u)=x$
    \item<2-> $\DEMM{\upm\alpha}$ is a function $h$ from $W\times T$ to denotations of $\alpha$'s type s.t. at every $\ram{w\PRI,t\PRI}\in W\times T$ $\Dem{\alpha}{\mMm,w\PRI,t\PRI,g}=h(\ram{w\PRI,t\PRI})=\DEMM{\upm\alpha}(\ram{w\PRI,t\PRI})$
  \end{itemize}
}

\subsection{Technical refinements}
\frame{\frametitle{Some examples}
  \begin{itemize}
    \item<1-> $\alpha=\beta$ at $\ram{w,t}$ might be true, but $\upm\alpha=\upm\beta$ need not be 1 at that same index
    \item<2-> on types:
                  \begin{itemize}
                    \item<2->$e$ - individuals
                    \item<3->$\ram{s,e}$ - individual concepts (`present Queen of England')
                    \item<4->$\ram{s,\ram{e,t}}$ - properties of inidviduals
                    \item<5->$\ram{e,t}$ - sets of individuals
                    \item<6->$\ram{\ram{s,e},t}$- sets of individual concepts
                  \end{itemize}  
  \end{itemize}
}

\frame{\frametitle{Some examples}
  \begin{itemize}
    \item<1-> on properties:
                  \begin{itemize}
                    \item<1->$\ram{s,\ram{a,t}}$ - properties of denotations of $a$-type expressions
                    \item<2->$\ram{s,\ram{e,t}}$ - properties of individuals
                    \item<3->$\ram{s,\ram{\ram{s,t},t}}$ - properties of propositions
                  \end{itemize}
    \item<4-> from relations $\ram{e,\ram{e,t}}$ to relations-in-intensions $\ram{s,\ram{e,\ram{e,t}}}$
  \end{itemize}
}

\frame{\frametitle{On indices}
\begin{itemize}
  \item<1-> In IL indices are never denoted by expressions!
  \item<2-> Expressions denote functions in the domain of indices.
  \item<3-> hence: $\ram{s,a}$ never applied to some typed argument ($s$ is not a type!)
  \item<4-> useful thing: We never talk about indices!
  \item<5-> since often $\downm\alpha(\beta)$ is needed for $\alpha\in ME_{\ram{s,\ram{e,t}}}$ and $\beta\in ME_e$, abbr. $\alpha\{\beta\}$
\end{itemize}
}

\section{Examples}
\frame{\frametitle{\textbf{Nec}}
\begin{itemize}
  \item<1-> former problem with \textbf{Nec} as $\ram{t,t}$: non-compositional extensional interpretation
  \item<2-> $\mathbf{Nec}\in ME_{\ram{\ram{s,t},t}}$ - ${\{0,1\}}^{(\{0,1\}^{W\times T})}$
  \item<3-> from (from indices to truth values = propositions) to truth values
  \item<4-> we could give $\Box\phi$ as $\mathbf{Nec}(\upm\phi)$
\end{itemize}
}

\frame{\frametitle{\textbf{For}}
\begin{itemize}
  \item<1-> `former' as in `a former member of this club'
  \item<2-> instead of $\ram{\ram{e,t},\ram{e,t}}$
  \item<3-> intensionally: $\ram{\ram{s,\ram{e,t}},\ram{e,t}}$
  \item<4-> extensions at all indices accessible via intension: those individuals bearing property $\ram{e,t}$ not at current but at some past index qualify
  \item<5->formally: $\DEMI{\mathbf{For}}$ is a func. $h$ s.t. for any property $k$, $h(\ram{w,t})(k)$ is the set $k(\ram{w,t\PRI})$ for all $t\PRI<t$.
  \item<6-> So, for any individual $x$ $h(\ram{w,t})(k)(x)=1$ iff $k(\ram{w,t\PRI})(x)=1$ for some $t\PRI<t$.
\end{itemize}
}

\frame{\frametitle{\textbf{Bel}}
\begin{itemize}
  \item<1-> relations between individuals and propositions
  \item<2-> $\ram{\ram{s,t},\ram{e,t}}$
  \item<3-> $\mathbf{Bel}(\upm(B(m))(j))$  \textit{John believes that Miss America is bald.}
  \item<4-> take the model from page 134 (Dowty et al.):
  \item<5-> $\Dem{B(m)}{M,w_2,t_1,g}=1$ since $\Dem{m}{M,w_2,t_1,g}=\Dem{n}{M,w_2,t_1,g}$
  \item<6-> however: $\Dem{\upm(B(m))}{M,w_2,t_1,g}\not=\Dem{\upm(B(n))}{M,w_2,t_1,g}$
\end{itemize}
}

\frame{\frametitle{de dicto}
\begin{itemize}
  \item<1-> $\mathbf{Bel}(\upm(B(m))(j))$ `John believes that Miss America is bald.'
  \item<2-> $\mathbf{Bel}(\upm(B(n))(j))$ `John believes that Norma is bald.'
  \item<3-> needn't be equal: John can take worlds other than $\ram{w_2,t_1}$ into account where $\dem{n}\not=\dem{m}$
  \item<4-> $\alpha=\beta\rightarrow\ekm{\phi\leftrightarrow\phi^{\ekm{\alpha/\beta}}}$ is true iff $\alpha$ is not in the scope of $\upm, \mathbf{F},\mathbf{P},\Box$ (oblique contexts)
  \item<5-> however: $\upm\alpha=\upm\beta\rightarrow\ekm{\phi\leftrightarrow\phi^{\ekm{\alpha/\beta}}}$
\end{itemize}
}

\frame{\frametitle{de re}
\begin{itemize}
  \item<1-> like so: $\lambda x\ekm{\mathbf{Bel}(\upm\ekm{B(x)})(j)}(m)$
  \item<2-> the above is true at an index $\ram{w,t}$ iff $\Dem{\mathbf{Bel}(\upm\ekm{B(x)})(j)}{w,t}=1$\\
     if $\Dem{m}{w,t}=x$, i.e. if John is in a believe-rel with $\upm(B(x))$ \\s.t. $g(x)=m$ (by semantics of $\lambda$)
  \item<3-> Why is $\upm(B(x))$ not equal to $\upm(B(m))$?
  \item<4-> constant $m$: non-rigid designator relativized to indices
  \item<5-> variable $x$: a rigid designator by def. of $g$ (for the relevant checking case with $g(x)=Miss America$
  \item<6-> the above: a belief about `whoever $m$ is'
  \item<7-> \rot{$\lambda$ conversion is restricted in IL!}
\end{itemize}
}

\frame{\frametitle{Once again}
\begin{itemize}
  \item<1-> \textit{John believes that a republican will win.}
  \item<2-> $\exists x\ekm{Rx\wedge\mathbf{Bel}(j,\upm\ekm{\mathbf{F}W(x)})}$
  \item<3-> $\mathbf{Bel}(j,\mathbf{F}\exists x\ekm{R(x)\wedge W(x)})$
\end{itemize}
}


\end{document}
