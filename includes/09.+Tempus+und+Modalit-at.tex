\begin{frame}
  {Kernfragen dieser Woche}
  \onslide<+->
  \onslide<+->
  \Large
  \centering 
  Wie kann man Tempuslogik durch Verschieben\\ von $i$-Indexen modellieren?\\
  \Halbzeile
  \onslide<+->
  Warum braucht man eine ausgeklügeltere Semantik\\ von Tempus und Modalität?\\
  \Halbzeile
  \onslide<+->
  Wie kann\slash muss man den Auswertungshintergrund\\ von Propositionen einschränken?\\
  \Halbzeile
  \onslide<+->
  Wie beeinflussen opake Kontexte (\textit{glauben} usw.)\\ die Satzbedeutung?\\
  \onslide<+->
  \Halbzeile
  \grau{\footnotesize Texte für heute: \citealt[Kapitel~5]{ChierchiaMcconnellginet2000}}
\end{frame}

\section{Tempus}

\begin{frame}
  {Tempus und Zeit}
  \onslide<+->
  \onslide<+->
  Priors Tempusoperatoren | \textit{Was war, was wird sein, was heißt \alert{immer}?}\\
  \Zeile
  \begin{itemize}[<+->]
    \item \alert{$\phi$} | \alert{Präsens}: Es ist jetzt ($i_{now}$) der Fall, dass $\phi$.
    \item \alert{$\mathbf{P}\phi$} | \alert{Präteritum}: Es war (zu einem $i<i_{now}$) der Fall, dass $\phi$.
    \item \alert{$\mathbf{F}\phi$} | \alert{Futur}: Es wird (zu einem $i>i_{now}$) der Fall sein, dass $\phi$.
    \item $\alert{\mathbf{G}\phi} = \neg \mathbf{F}\neg \phi$ | Es wird immer der Fall sein, dass $\phi$.
    \item $\alert{\mathbf{H}\phi} = \neg \mathbf{P}\neg \phi$ | Es war immer der Fall, dass $\phi$.
  \end{itemize}
\end{frame}

\begin{frame}
  {Auswertung}
  \onslide<+->
  \onslide<+->
  \textit{Arno Schmidt ist gestorben.} bzw.\ \alert{\textit{Arno Schmidt starb.}}\\
  \Zeile
  \begin{itemize}[<+->]
    \item Priorsche Formalisierung | $\mathbf{P}S(a)$
      \Halbzeile
    \item Realtiv zu $\ram{w,i}$ (reale Welt jetzt) ist \alert{$\DEMM{\mathbf{P}D(a)}=1$}
      \Halbzeile
      \begin{itemize}[<+->]
         \item wenn es ein $i\Prm$ gibt sodass $i\Prm<i$\\
           \grau{\footnotesize äquivalent $\ram{i\Prm,i}\in\ <$}
           \Viertelzeile
         \item sodass $\Dem{D(a)}{\mMm,w,i\Prm,g}=1$
           \end{itemize}
  \end{itemize}
\end{frame}

\begin{frame}
  {Natürliche Sprache und TP\slash IP}
  \onslide<+->
  \onslide<+->
  Wie immer: Marker von Tempus stehen nicht am Satzanfang\\
  \Halbzeile
  \begin{itemize}[<+->]
    \item Priorsche Tempusoperatoren als Modifikation von Wffs (Logik), also \alert{Sätzen} (NL)
    \item GB-Ansätze mit \alert{Tempusanhebung} in Position mit \alert{Satzskopus}
      \Halbzeile
    \item TP\slash IP-Position motiviert durch Kopulas und Hilfsverben (im Englischen)
      \begin{itemize}[<+->]
        \item \textit{He \alert{is} stupid.} -- aber -- \textit{Kare-wa bakarashi-\alert{i}.}
        \item \textit{He \alert{was} stupid.} -- aber -- \textit{Kare-wa bakarashi-kat-\alert{ta}.}
        \item \textit{What\Sub{i} \alert{did} you expect t\Sub{i}.} -- aber -- \textit{Nani-o yokishi-\alert{ta}-ka.}
      \end{itemize}
  \end{itemize}
\end{frame}

\begin{frame}
  {Syntax für ein Chierchia-Fragment mit Tempus}
  \onslide<+->
  \begin{itemize}[<+->]
    \item \alert{T$\Prm \rightarrow$ T VP} | Tempusmarkierung der VP über T-Kopf (T bzw.\ T\Up{0})
    \item TP $\rightarrow$ NP T$\Prm$ | Subjekt in TP\slash IP
    \item TP $\rightarrow$ TP conj TP | Satzverbindungen = TP-Verbindungen
    \item TP $\rightarrow$ neg TP | Satznegation
    \item \alert{[\Sub{TP} NP T VP] $\Rightarrow$ [\Sub{TP} T NP VP]} | Tempusanhebung (Linkssadjunktion!)
  \end{itemize}
\end{frame}

\begin{frame}
  {Semantik und Valuation à la Chierchia}
  \onslide<+->
  \begin{itemize}[<+->]
    \item Semantische Auswertung einer TP
      \Halbzeile
      \begin{itemize}[<+->]
        \item Konkrete T\Up{0} | Hilfsverben mit Bedeutung $\mathbf{P}$, $\mathbf{F}$
        \item $\DEMM{\mathbf{P} TP} =1$
        \item wenn es mindestens ein $i\Prm$ gibt, für das $i\Prm<i$,
        \item und sodass $\Dem{TP}{\mMm,w,i\Prm,g}=1$
      \end{itemize}
      \Zeile
    \item Valuation
      \Halbzeile
      \begin{itemize}[<+->]
        \item \alert{$U$} | Diksursuniversum, Quantifikationsdomäne
        \item \alert{$V(\beta)$} | Nicht-modal-temporale Auswertungsfunktion für alle $\beta$ außer Eigennamen
        \item \alert{$V(\beta)(\ram{w,i})$} | Modal-Temporale Auswertungsfunktion: Für jedes Prädikat eine Funktion von Welt-Zeit-Paaren zur Prädikatsmenge (Individuen, Tupel)
      \end{itemize}
  \end{itemize}
\end{frame}


% %\subsection{Some problems}
% \frame{\frametitle{Natural tenses}
% \begin{itemize}
%   \item<1-> NL tenses beyond TOp's:
%   \item<2-> \emph{Arno Schmidt had already read Poe when he started writing `Zettels Traum'.}
%   \item<3-> \emph{Gosh, I forgot to feed the cat.}
%   \item<4-> \bl{shifts of evaluation time}
% \end{itemize}
% }
% 
% \frame{\frametitle{Reichenbach}
% {\footnotesize\begin{tabular}{|l|l|l|l|}
%   \hline 
%    & past (R$<$S) & present (R,S) & future (S$<$R) \\
%    \hline
%    anterior(E$<$R) & E$<$R$<$S & E$<$R,S & S$<$E$<$R \\
%                  & \emph{er war gegangen} & \emph{er ist gegangen} & S,E$<$R \\
%                  & & & E$<$S$<$R\\
%                  & & & \emph{er wird gegangen sein} \\
%    \hline
%    simple(E,R) & E,R$<$S & \bl{E,R,S} & S$<$E,R \\
%                  & \emph{er ging} & \emph{er geht} & \emph{er wird gehen} \\
%    \hline
%    posterior(R$<$E) & R$<$E$<$S & R,S$<$E & S$<$R$<$E \\
%                   & R$<$S,E & \emph{er wird gehen} & \emph{$^{\ast}$er wird gehen werden} \\
%                   & R$<$S,E & & \\
%                   & R$<$S$<$E & & \\
%                   & \emph{$^{\ast}$er w\"urde gehen} & & \\
%    \hline
% \end{tabular}}
% }
% 
% \frame{\frametitle{Embedded tenses and adverbials}
% \begin{itemize}
%   \item<1-> \emph{A man was born who will be king.}
%   \item<2-> \textbf{P}(a man is born \textbf{F}(who be king)) ?
%   \item<3-> \emph{Yesterday, Maria woke up happy.}
%   \item<4-> \textbf{Y}(\textbf{P}(Maria wake up happy)) ?
% \end{itemize}
% }

\section{Modalität}

\begin{frame}
  {Sprachliche Realisierung von Modalität}
  \onslide<+->
  \onslide<+->
  Modalität in sehr verschiedenen Erscheinungsformen\\
  \Halbzeile
  \begin{itemize}[<+->]
    \item \textit{I \alert{eat up to} 100 nachos a minute.} | \alert{Tempusformen}
    \item \textit{\alert{Responderet} alius minus sapienter.} | \alert{Modus}
    \item \textit{Herr Webelhuth \alert{can} look like Michael Moore.} | \alert{Modalverben}
    \item \textit{\alert{Maybe} Herr Keydana will show up.} | \alert{Adverben}
    \item \textit{Frau Klenk is recogniz\alert{able}.} | \alert{Affixe}
  \end{itemize}
\end{frame}

\begin{frame}
  {Arten von Modalitäten}
  \onslide<+->
  \onslide<+->
  Auswertung von Modalität vor einem \alert{Hintergrund von Welten}
  \Halbzeile
  \begin{itemize}[<+->]
    \item Modallogik | Auswertung von $\Box$ und $\Diamond$ relativ zu \alert{allen Welten}\\
      \grau{\footnotesize Zumindest in einer einfachen Modallogik für Einsteiger}
      \Halbzeile
    \item Natürliche Sprache | \textit{Wir müssen gehen.} usw.\ als \alert{ambige Sätze}
    \item \alert{Mehreren Lesarten} je nach \gruen{spezifischem Hintergrund von Welten}
  \end{itemize}
\end{frame}

\begin{frame}
  {Logische Modalität\slash Root Modality}
  \onslide<+->
  \onslide<+->
  \textit{Agent Cooper \alert{cannot} solve the mystery.}\\
  \Halbzeile
  \begin{itemize}[<+->]
    \item Logische Form | $\neg\lozenge S(c,m)$
      \Halbzeile
    \item Falsche Interpretation | \orongsch{Er könnte unter keinen Umständen das Rätsel lösen.}
    \item Korrekt | In den \alert{kontextuell salienten Hintergrundwelten}\\
      verhindern Umstände die Lösung.
      \Viertelzeile
      \begin{itemize}[<+->]
        \item Cooper fehlen Informationen, sonst könnte er.
        \item Cooper liegt angeschossen im Great Northern, sonst könnte er.
        \item Usw.
      \end{itemize}
  \end{itemize}
\end{frame}


\begin{frame}
  {Epistemische Modalität}
  \onslide<+->
  \onslide<+->
  \textit{Leo Johnson \alert{must} be the murderer of Laura Palmer.}\\
  \Halbzeile
  \begin{itemize}[<+->]
    \item \alert{Bekannte Fakten}\slash \alert{Wissenshintergrund} legen den Schluss zwingend nah.\\
      \grau{\footnotesize Hier: Twin Peaks, Staffel 1, Folge 7}
      \begin{itemize}[<+->]
        \item Leo ist eine gewalttätige Person.
        \item Leo schmuggelt Kokain nach TP, Laura war abhängig von K.
        \item Leo hat Verbindungen zu Jacques Renault, dem Barkeeper aus One Eyed Jack's,\\
          und Laura hat bei One Eyed Jack's gearbeitet.
      \end{itemize}
      \Halbzeile
    \item Bekannte Fakten\slash der \alert{epistemische Hintergrund} zur \alert{Reduktion}\\
      \alert{des Hintergrunds möglicher Welten}
    \item Bei \orongsch{Irrtum} | Ein paar Welten zu viel entfernt
  \end{itemize}
\end{frame}


\begin{frame}
  {Deontische Modalität}
  \onslide<+->
  \onslide<+->
  \textit{Agent Cooper \alert{must} solve the mystery.}\\
  \Halbzeile
  \begin{itemize}[<+->]
    \item \alert{Juristische\slash moralische Postulate} fordern von Cooper eine Lösung.\\
      \grau{\footnotesize Hier: Twin Peaks, Staffel 1--2}
      \begin{itemize}[<+->]
        \item Cooper hat als FBI-Agent einen Eid geschworen und eine Dienstpflicht.
        \item Ohne Lösung könnte es weitere Opfer geben.
        \item Es geht um Gut und Böse an sich, wir sind auf der Seite des Guten.
      \end{itemize}
      \Halbzeile
    \item Der \alert{deontische Hintergrund} zur \alert{Reduktion der Welten\\
      auf die moralisch\slash juristisch erwünschten}
    \item Oft kodifiziert | Zehn Gebote, BGB, StGB usw.
  \end{itemize}
\end{frame}

\begin{frame}
  {Funktion zur Reduktion der relevanten Welten}
  \onslide<+->
  \onslide<+->
  Welche Welten brachen wir gerade?\\
  \Halbzeile
  \begin{itemize}[<+->]
    \item Der jeweils relevante logische\slash epistemische\slash deontische Weltenhintergrund
    \item Gegeben durch \alert{eine Funktion in $\wp W^{\wp W}$} \grau{bzw.\ $(\wp W\times I)^{(\wp W\times I)}$}
      \Halbzeile
    \item Bei Chierchia $g$ | Warum?
      \Halbzeile
    \item Interessant wäre die Frage, wie die Welten ausgewählt werden.\\
      Eine Funktion zu postulieren löst hier erstmal noch nicht viel.
  \end{itemize}
\end{frame}


\section{Eingebettete Propositionen}

\begin{frame}
  {Syntax und Semantik der Einbettung}
  \onslide<+->
  \onslide<+->
  \textit{Moreau \orongsch{glaubt}, \alert{dass} \gruen{Ästhetizismus toll ist}.}\\
  \Halbzeile
  \begin{itemize}[<+->]
    \item In GB-artiger Syntax
      \begin{itemize}[<+->]
        \item \alert{CP $\rightarrow$ C IP}
        \item Theta-Rolle für die CP vom Matrixverb
        \item \grau{Einbettung von Infinitiven etwas komplizierter wegen PRO o.\,ä.}
      \end{itemize}
      \Halbzeile
    \item Semantik von \alert{Propositionalen Einstellungsverben} wie \textit{glauben}
      \begin{itemize}[<+->]
        \item Inhalt der Einstellung | Eine vom Subjekt für wahr gehaltene Proposition
        \item Formal \alert{eine Menge von $\ram{w_n,i_n}$ aus dem Hintergrund des Sprechers}
      \end{itemize}
  \end{itemize}
\end{frame}

\begin{frame}
  {Der Up-Operator\ \ $\hat{ }$}
  \onslide<+->
  \onslide<+->
  Propositionen \grau{(Mengen von $\ram{w_j,i_j}$)} als First-Class Citizens der Logik\\
  \Halbzeile
  \begin{itemize}[<+->]
    \item Für Tupel von \alert{Individuen $u_n\in U$} und \alert{Propositionen $p_m\in\wp W\times I$}
    \item $\DEMM{glauben}$ als \alert{Funktion in $(U\times(\wp W\times I))^{(W\times I)}$}
    \item Konkret von der Sprecher-Welt-Zeit-Koordinate $\ram{w,i}$\\
      zu einem Tupel aus Glaubendem $u_n$ und dem Inhalt des Glaubens $p_m$
      \Halbzeile
    \item Der Up-Operator |\ \ \alert{$\hat{ }\phi$} sei die \alert{Intension des Ausdrucks $\phi$}.
    \item \alert{$G(m, \hat{\ }\ T(\text{\textit{ä}}))$} oder lesbarer \alert{Glaubt(moreau, $\hat{\ }\ $Toll(ästhetizismus))}
    \item Wahr, wenn es jetzt ein Tupel aus Moreau und einer Menge Welten gibt,\\
      in denen Ästhetizismus toll ist, sodass diese Welten Teil des Weltenhintergrunds\\
      von Moreau sind.
    \item \gruen{Gelöstes Problem | "`Wahrheitswert"' des Glaubensinhalts}
    \item Verben wie \textit{glauben} fordern eine Proposition als Argument!
  \end{itemize}
\end{frame}

\begin{frame}
  {Quines Ortcutt-Geschichte I}
  \onslide<+->
  \onslide<+->
  \alert{\textit{Ralph believes that the guy from the beach is a spy.}}\\
  \Halbzeile
  \begin{itemize}[<+->]
    \item Ralph kennt B.\,J.\ Ortcutt als netten Typen vom Strand.
    \item Abends sieht er einen dubiosen Typen mit Hut im Dunkeln in einer Seitenstraße.
    \item Der Typ ist Ortcutt, der in der Kneipe in Verkleidung eine Show abziehen will.
    \item Aber Ralph erkennt ihn nicht.
  \end{itemize}
\end{frame}

\begin{frame}
  {Quines Ortcutt-Geschichte II}
  \onslide<+->
  \onslide<+->
  \alert{\textit{Ralph believes that the guy from the beach is a spy.}}\\
  \Halbzeile
  \begin{itemize}[<+->]
    \item Ist der obige Satz wahr oder falsch?
    \item \gruen{Wahr!} Ortcutt und der dubiose Typ sind dasselbe Individuum.
    \item \orongsch{Falsch!} Ralph weiß das nicht und glaubt auch nicht daran.
      \Halbzeile
    \item Ist Ralph wahnsinnig oder nicht ganz normal?
    \item Oder können Sätze gleichzeitig wahr und falsch sein?
  \end{itemize}
\end{frame}

\begin{frame}
  {De dicto und de re}
  \onslide<+->
  \onslide<+->
  Russells Interpretation definiter Singular-NPs\\
  \onslide<+->
  \alert{$the\stackrel{def}{=}\lambda{Q}\lambda{}P\ekm{\exists{x}\ekm{Q(x)\wedge{}P(x)}\wedge\forall{}y\ekm{Q(y)\leftrightarrow y=x}}$}\\
  \grau{\footnotesize Beispiel | Q für \textit{Queen of England} und P für \textit{is bald}}
  \Halbzeile
  \begin{itemize}[<+->]
    \item In einem Bewegungsansatz 
      \begin{itemize}[<+->]
        \item Quantorenbewegung an \gruen{einbettende} oder \orongsch{eingebettete} IP
        \item \gruen{[$_{IP}$ the guy from the beach$_i$ [$_{IP}$ Ralph believes [$_{CP}$ that x$_i$ is a spy]]]}
        \item \orongsch{Ralph believes [$_{CP}$ that [$_{IP}$ the guy from the beach$_i$ [$_{IP}$ x$_i$ is a spy]]]}
      \end{itemize}
      \Halbzeile
    \item Zwei Lesarten automatisch verfügbar
      \begin{itemize}[<+->]
        \item \gruen{De re-Lesart} | \gruen{Wahr!} Denn für den Typen vom Strand gilt \ldots
        \item \orongsch{De dicto-Lesart} | \orongsch{Falsch!} Denn Ralph glaubt, dass \ldots
      \end{itemize}
  \end{itemize}
\end{frame}

\begin{frame}
  {Rigide Designatoren}
  \onslide<+->
  \onslide<+->
  \textit{Yuri Gagarin \alert{might not} have been the first man in space.}\\
  \grau{\footnotesize Erinnerung | In einem naiven Ansatz: \textit{YG könnte auch nicht YG gewesen sein.}}
  \Halbzeile
  \begin{itemize}[<+->]
    \item Namen sind \alert{rigide} und bezeichnen immer dasselbe Individuum! (Kripke)
      \Halbzeile
    \item $\Diamond$ THE(first-man-in-space)(not-be-Gagarin)
      \begin{itemize}[<+->]
        \item \gruen{In irgendeiner Welt ist YG (rigide) nicht der erste Mensch auf dem Mond (nicht-rigide).}
      \end{itemize}
      \Halbzeile
    \item THE(first-man-in-space)($\Diamond$[not-be-Gagarin])
      \begin{itemize}[<+->]
        \item \orongsch{Der erste Mensch auf dem Mond (= YG) war in einer zugänglichen Welt nicht YG.}
        \item Diese Lesart ist auszuschließen. S. Chierchia, Dowty usw.
      \end{itemize}
  \end{itemize}
\end{frame}

