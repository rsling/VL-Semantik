

\frame{\frametitle{}
\begin{center}
\textcolor{blue}{The book (PMW:87-246) deals with logic far more in-depth than we do. Only what is mentioned on the slides is relevant for the test. Reading the whole chapter from PMW will do you no harm, though.}
\end{center}
}

\section{What logic is about}

\subsection{On reasoning}
\frame{\frametitle{Theories}
\begin{itemize}
  \item<1-> a collection of statements (propositions)
  \item<2-> axioms (statements accepted to be true)
  \item<3-> maybe based on observations (induction)
  \item<4-> statements that follow from the axioms (deduction)
  \item<5-> predictions beyond the axioms
  \item<6-> rechecking for usability: e.g., Russell's paradox
\end{itemize}
}

\frame{\frametitle{Proofs}
\begin{itemize}
  \item<1-> \textcolor{blue}{axioms}: atomic truths of your theory
  \item<2-> \textcolor{blue}{theorem}: a proposition you want to prove
  \item<3-> \textcolor{blue}{lemma}: subsidiary propositions (used to prove the theorem)
  \item<4-> \textcolor{blue}{corollary}: propositions proved while proving some axiom
\end{itemize}
}

\subsection{Where we need logic}
\frame{\frametitle{A method of reasoning}
\begin{itemize}
  \item<1-> logic does not generate truths
  \item<2-> \textcolor{blue}{formalizing statements, predications} etc.
  \item<3-> \textcolor{blue}{rules of deduction from axioms to theorems}
  \item<4-> empirical (induction) and exact (deduction) science
  \item<5-> aiming at an adequate \textcolor{blue}{model} of the world (e.g., heliocentric universe)
\end{itemize}
}

\frame{\frametitle{Why logic for semantics?}
\begin{itemize}
  \item<1-> truth-conditional
  \item<2-> compositional behavior of propositions and connectives
  \item<3-> a logic for entailments
  \item<4-> why, e.g.: \emph{It is not the case that someone is happy.} $\rightarrow$ \emph{Nobody is happy.}
\end{itemize}
}

\section{Statement calculus}
\subsection{Formalization: Recursive Syntax}
\frame{\frametitle{Atomic formulas: statements}
\begin{itemize}
  \item<1-> statements/propositions = the \bl{atoms}
  \item<2-> a propositional symbol \emph{p}: a well-formed formula (\bl{wff})
  \item<3-> ex.: \emph{Herr \underline{K}eydana is a passionate cyclist.}: \emph{k}
  \item<4-> \den{k}=1 or 0 (depending on corresponding \textbf{model})
\end{itemize}
}

\frame{\frametitle{Complex (molecular) formulas}
\begin{itemize}
  \item<1-> \textcolor{blue}{syntax}: restricts the forms of wff's to make them interpretable
  \item<2-> define functors: functions in $\{0,1\}$
  \item<3-> If \emph{p} and \emph{q} are wff's, then
      \begin{itemize}
         \item<4-> $\textcolor{blue}{\neg} p$
         \item<5-> $p\textcolor{blue}{\vee} q$
         \item<6-> $p\textcolor{blue}{\wedge} q$
         \item<7-> $p\textcolor{blue}{\rightarrow} q$
         \item<8-> $p\textcolor{blue}{\leftrightarrow} q$
      \end{itemize}
     is also a wff (a \bl{molecular term}).
\end{itemize}
}

\frame{\frametitle{Complex (molecular) formulas}
\begin{itemize}
  \item<1-> \textcolor{blue}{syntax}: restricts forms of wff's to make them interpretable
  \item<1-> define functors: functions in $\{\ra{0,1},\ra{1,0},0,1\}$
  \item<1-> If \emph{p} and \emph{q} are wff's, then
      \begin{itemize}
         \item<1-> $\textcolor{blue}{\neg} p$ \textcolor{blue}{(negation)}
         \item<1-> $p\textcolor{blue}{\vee} q$ \textcolor{blue}{(disjunction)}
         \item<1-> $p\textcolor{blue}{\wedge} q$ \textcolor{blue}{(conjunction)}
         \item<1-> $p\textcolor{blue}{\rightarrow} q$ \textcolor{blue}{(conditional)}
         \item<1-> $p\textcolor{blue}{\leftrightarrow} q$ \textcolor{blue}{(biconditional)}
      \end{itemize}
     is also a wff.
\end{itemize}
}

\frame{\frametitle{Complex (molecular) formulas}
\begin{itemize}
  \item<1-> \textcolor{blue}{syntax}: restricts forms of wff's to make them interpretable
  \item<1-> define functors: functions in $\{\ra{0,1},\ra{1,0},0,1\}$
  \item<1-> If \emph{p} and \emph{q} are wff's, then
      \begin{itemize}
         \item<1-> $\textcolor{blue}{\neg} p$ \textcolor{blue}{(negation - `not')}
         \item<1-> $p\textcolor{blue}{\vee} q$ \textcolor{blue}{(disjunction - `or')}
         \item<1-> $p\textcolor{blue}{\wedge} q$ \textcolor{blue}{(conjunction - `and')}
         \item<1-> $p\textcolor{blue}{\rightarrow} q$ \textcolor{blue}{(conditional - `if')}
         \item<1-> $p\textcolor{blue}{\leftrightarrow} q$ \textcolor{blue}{(biconditional - `iff')}
      \end{itemize}
     is also a wff.
\end{itemize}
}

\subsection{Interpretation}

\frame{\frametitle{Functions and truth tables}
\begin{itemize}
  \item<1-> standard defintion: \\
  									\medskip    
                    \begin{center}
                    $\dem{\neg} = \left[
                         \begin{array}{l}
	                          1 \rightarrow 0\\
	                          0 \rightarrow 1
                         \end{array}
                         \right]$
                       \end{center}
                         
  \item<2-> but most widely used: \textcolor{blue}{truth tables}\\
              \begin{center}
              \begin{tabular}{c|c}
                    $\neg$ & $p$\\
                    \hline
                    \textcolor{blue}{0} & 1 \\
                    \textcolor{blue}{1} & 0 \\
                  \end{tabular}
              \end{center}
\end{itemize}
}

\frame{\frametitle{Disjunction}
     \begin{center}
            \begin{tabular}{c|c|c}
                  $p$ & $\vee$ & $q$\\
                  \hline
                  1 & \textcolor{blue}{1} & 1 \\
                  1 & \textcolor{blue}{1} & 0 \\
                  0 & \textcolor{blue}{1} & 1 \\
                  0 & \textcolor{blue}{0} & 0 \\
                \end{tabular}
            \end{center}
\begin{itemize}
  \item<2-> \emph{Herr \underline{K}eydana is a passionate cyclist \textbf{or} we all \underline{l}ove logic.}
  \item<3-> \textcolor{blue}{$K\vee$L}
\end{itemize}
}

\frame{\frametitle{Conjunction}
     \begin{center}
            \begin{tabular}{c|c|c}
                  $p$ & $\wedge$ & $q$\\
                  \hline
                  1 & \textcolor{blue}{1} & 1 \\
                  1 & \textcolor{blue}{0} & 0 \\
                  0 & \textcolor{blue}{0} & 1 \\
                  0 & \textcolor{blue}{0} & 0 \\
                \end{tabular}
            \end{center}
\begin{itemize}
  \item<2-> \emph{Herr \underline{K}eydana is a passionate cyclist \textbf{and} we all \underline{l}ove logic.}
  \item<3-> \textcolor{blue}{$K\wedge$L}
\end{itemize}
}

\frame{\frametitle{Conditional}
     \begin{center}
            \begin{tabular}{c|c|c}
                  $p$ & $\rightarrow$ & $q$\\
                  \hline
                  1 & \textcolor{blue}{1} & 1 \\
                  1 & \textcolor{blue}{0} & 0 \\
                  0 & \textcolor{blue}{1} & 1 \\
                  0 & \textcolor{blue}{1} & 0 \\
                \end{tabular}
            \end{center}
\begin{itemize}
  \item<2-> \emph{\textbf{If} it \underline{r}ains, \textbf{then} the \underline{s}treets get wet.}
  \item<3-> \textcolor{blue}{$R\rightarrow$S}
\end{itemize}
}

\frame{\frametitle{Any problems with that?}
\textbf{\emph{If it rains, the streets get wet.}}
\begin{itemize}
  \item<1-> it is raining (\textcolor{blue}{1}) , the streets are wet \textcolor{blue}{1} : \textcolor{blue}{\textbf{1}}
  \item<2-> it is raining (\textcolor{blue}{1}) , the streets are dry \textcolor{blue}{0} : \textcolor{blue}{\textbf{0}}
  \item<3-> it is not raining (\textcolor{blue}{0}) , the streets are wet \textcolor{blue}{1} : \textcolor{blue}{\textbf{1}}
  \item<4-> it is not raining (\textcolor{blue}{0}) , the streets are dry \textcolor{blue}{0} : \textcolor{blue}{\textbf{1}}
  \item<5-> \textcolor{blue}{ex vero non sequitur falsum}
\end{itemize}
}

\frame{\frametitle{Biconditional}
     \begin{center}
            \begin{tabular}{c|c|c}
                  $p$ & $\leftrightarrow$ & $q$\\
                  \hline
                  1 & \textcolor{blue}{1} & 1 \\
                  1 & \textcolor{blue}{0} & 0 \\
                  0 & \textcolor{blue}{0} & 1 \\
                  0 & \textcolor{blue}{1} & 0 \\
                \end{tabular}
            \end{center}
\begin{itemize}
  \item<2-> \emph{\textbf{If and only if} your \underline{s}core is above 50, \textbf{then} you \underline{p}ass the semantics exam.}
  \item<3-> \textcolor{blue}{$S\leftrightarrow$P}
\end{itemize}
}

\frame{\frametitle{Scope of functors}
\begin{itemize}
  \item<1-> brackets are facultative
  \item<2-> or set non-default functor scope
  \item<3-> default scope\\
  
  \medskip
     \begin{center}
        \textcolor{blue}{scope} $\left\downarrow \begin{array}{c}
           \neg \\
           \wedge \\
           \vee \\
           \rightarrow \\
           \leftrightarrow
         \end{array} \right\uparrow
        $ \textcolor{blue}{binding strength}
     \end{center}
\end{itemize}
}

\frame{\frametitle{An example}
\begin{itemize}
  \item<1-> $p\wedge\neg q\vee r\rightarrow \neg s$
  \item<2-> $p\wedge\textcolor{blue}{(\neg} q\textcolor{blue}{)}\vee r\rightarrow \textcolor{blue}{(\neg} s\textcolor{blue}{)}$
  \item<3-> $\textcolor{blue}{(}p\textcolor{blue}{\wedge}(\neg q)\textcolor{blue}{)}\vee r\rightarrow (\neg s)$
  \item<4-> $\textcolor{blue}{(}(p\wedge(\neg q))\textcolor{blue}{\vee} r\textcolor{blue}{)}\rightarrow (\neg s)$
  \item<5-> $\textcolor{blue}{(}((p\wedge(\neg q))\vee r)\textcolor{blue}{\rightarrow} (\neg s)\textcolor{blue}{)}$
\end{itemize}
}

% \frame{\frametitle{An example: Polish notation}
% \begin{itemize}
%   \item<1-> uses letters for functors and \bl{prefix style}: 
%      \begin{itemize}
%        \item<2-> $\neg$ \bl{N} (\emph{negatio})
%        \item<3-> $\vee$ \bl{A} (\emph{alternatio)}
%        \item<4-> $\wedge$ \bl{K} (\emph{koniunktio})
%        \item<5-> $\rightarrow$ \bl{C} (\emph{conditionalis})
%        \item<6-> $\leftrightarrow$ \bl{E} (\emph{equivalentia})
%      \end{itemize}
%   \item<7-> $p\wedge\neg q\vee r\rightarrow \neg s$ : CAKpNqrNs
%   \item<8-> (C(A(Kp(Nq))r)(Ns))
% \end{itemize}
% }
% 
% \frame{\frametitle{An example}
% Draw trees!\\
%   \begin{center}
%     \Tree[0]{
%     p\B{ddr} & \wedge & \neg\B{d} & q\B{dl} & \vee & r\B{dddl} & \rightarrow & \neg\B{d} & s\B{dl} \\
%     &        & \neg\B{dl} &   &      &   &            & \neg\B{dddl} &   \\
%     & \wedge\B{drrr} \\
%     &        &      &   & \vee\B{drr} \\
%        &        &      &   &      &   & \rightarrow \\
%     }
%     \end{center}
% }

\frame{\frametitle{Large truth tables}
\begin{itemize}
  \item<1-> for \emph{n} atoms in the term: $2^n$ lines
  \item<2-> alternating blocks of 1's and 0's under every atom
  \item<3-> $2^{(m-1)}$ times `1' followed by $2^{(m-1)}$ times `0' for the $m$-th atom from the right
  \item<4-> until $2^n$ lines are reached
\end{itemize}
}

\frame{\frametitle{An example}
 {\scriptsize
    \begin{center}
    \begin{tabular}{c|c|c|c|c|c|c|c|c}
     $p$ & $\wedge$ & $\neg$ & $q$ & $\vee$ & $r$ & $\rightarrow$ & $\neg$ & $s$ \\
     \hline
      1 &  &  & 1 &  & 1 &  &  & 1 \\
      1 &  &  & 1 &  & 1 &  &  & 0 \\
      1 &  &  & 1 &  & 0 &  &  & 1 \\
      1 &  &  & 1 &  & 0 &  &  & 0 \\
      1 &  &  & 0 &  & 1 &  &  & 1 \\
      1 &  &  & 0 &  & 1 &  &  & 0 \\
      1 &  &  & 0 &  & 0 &  &  & 1 \\
      1 &  &  & 0 &  & 0 &  &  & 0 \\
      0 &  &  & 1 &  & 1 &  &  & 1 \\
      0 &  &  & 1 &  & 1 &  &  & 0 \\
      0 &  &  & 1 &  & 0 &  &  & 1 \\
      0 &  &  & 1 &  & 0 &  &  & 0 \\
      0 &  &  & 0 &  & 1 &  &  & 1 \\
      0 &  &  & 0 &  & 1 &  &  & 0 \\
      0 &  &  & 0 &  & 0 &  &  & 1 \\
      0 &  &  & 0 &  & 0 &  &  & 0 \\
     \end{tabular}
   \end{center}
 }
}

\frame{\frametitle{An example}
 {\scriptsize
    \begin{center}
    \begin{tabular}{c|c|c|c|c|c|c|c|c}
     $p$ & $\wedge$ & $\neg$ & $q$ & $\vee$ & $r$ & $\rightarrow$ & $\neg$ & $s$ \\
     \hline
      \gr{1} &  & \bl{0} & 1 &  & \gr{1} &  & \bl{0} & 1 \\
      \gr{1} &  & \bl{0} & 1 &  & \gr{1} &  & \bl{1} & 0 \\
      \gr{1} &  & \bl{0} & 1 &  & \gr{0} &  & \bl{0} & 1 \\
      \gr{1} &  & \bl{0} & 1 &  & \gr{0} &  & \bl{1} & 0 \\
      \gr{1} &  & \bl{1} & 0 &  & \gr{1} &  & \bl{0} & 1 \\
      \gr{1} &  & \bl{1} & 0 &  & \gr{1} &  & \bl{1} & 0 \\
      \gr{1} &  & \bl{1} & 0 &  & \gr{0} &  & \bl{0} & 1 \\
      \gr{1} &  & \bl{1} & 0 &  & \gr{0} &  & \bl{1} & 0 \\
      \gr{0} &  & \bl{0} & 1 &  & \gr{1} &  & \bl{0} & 1 \\
      \gr{0} &  & \bl{0} & 1 &  & \gr{1} &  & \bl{1} & 0 \\
      \gr{0} &  & \bl{0} & 1 &  & \gr{0} &  & \bl{0} & 1 \\
      \gr{0} &  & \bl{0} & 1 &  & \gr{0} &  & \bl{1} & 0 \\
      \gr{0} &  & \bl{1} & 0 &  & \gr{1} &  & \bl{0} & 1 \\
      \gr{0} &  & \bl{1} & 0 &  & \gr{1} &  & \bl{1} & 0 \\
      \gr{0} &  & \bl{1} & 0 &  & \gr{0} &  & \bl{0} & 1 \\
      \gr{0} &  & \bl{1} & 0 &  & \gr{0} &  & \bl{1} & 0 \\
     \end{tabular}
   \end{center}
 }
}

\frame{\frametitle{An example}
 {\scriptsize
    \begin{center}
    \begin{tabular}{c|c|c|c|c|c|c|c|c}
     $p$ & $\wedge$ & $\neg$ & $q$ & $\vee$ & $r$ & $\rightarrow$ & $\neg$ & $s$ \\
     \hline
      1 & \bl{0} & 0 & \lgr{1} &  & \gr{1} &  & \gr{0} & \lgr{1} \\
      1 & \bl{0} & 0 & \lgr{1} &  & \gr{1} &  & \gr{1} & \lgr{0} \\
      1 & \bl{0} & 0 & \lgr{1} &  & \gr{0} &  & \gr{0} & \lgr{1} \\
      1 & \bl{0} & 0 & \lgr{1} &  & \gr{0} &  & \gr{1} & \lgr{0} \\
      1 & \bl{1} & 1 & \lgr{0} &  & \gr{1} &  & \gr{0} & \lgr{1} \\
      1 & \bl{1} & 1 & \lgr{0} &  & \gr{1} &  & \gr{1} & \lgr{0} \\
      1 & \bl{1} & 1 & \lgr{0} &  & \gr{0} &  & \gr{0} & \lgr{1} \\
      1 & \bl{1} & 1 & \lgr{0} &  & \gr{0} &  & \gr{1} & \lgr{0} \\
      0 & \bl{0} & 0 & \lgr{1} &  & \gr{1} &  & \gr{0} & \lgr{1} \\
      0 & \bl{0} & 0 & \lgr{1} &  & \gr{1} &  & \gr{1} & \lgr{0} \\
      0 & \bl{0} & 0 & \lgr{1} &  & \gr{0} &  & \gr{0} & \lgr{1} \\
      0 & \bl{0} & 0 & \lgr{1} &  & \gr{0} &  & \gr{1} & \lgr{0} \\
      0 & \bl{0} & 1 & \lgr{0} &  & \gr{1} &  & \gr{0} & \lgr{1} \\
      0 & \bl{0} & 1 & \lgr{0} &  & \gr{1} &  & \gr{1} & \lgr{0} \\
      0 & \bl{0} & 1 & \lgr{0} &  & \gr{0} &  & \gr{0} & \lgr{1} \\
      0 & \bl{0} & 1 & \lgr{0} &  & \gr{0} &  & \gr{1} & \lgr{0} \\
     \end{tabular}
   \end{center}
 }
}

\frame{\frametitle{An example}
 {\scriptsize
    \begin{center}
    \begin{tabular}{c|c|c|c|c|c|c|c|c}
     $p$ & $\wedge$ & $\neg$ & $q$ & $\vee$ & $r$ & $\rightarrow$ & $\neg$ & $s$ \\
     \hline
      \lgr{1} & 0 & \lgr{0} & \lgr{1} & \bl{1} & 1 &  & \gr{0} & \lgr{1} \\
      \lgr{1} & 0 & \lgr{0} & \lgr{1} & \bl{1} & 1 &  & \gr{1} & \lgr{0} \\
      \lgr{1} & 0 & \lgr{0} & \lgr{1} & \bl{0} & 0 &  & \gr{0} & \lgr{1} \\
      \lgr{1} & 0 & \lgr{0} & \lgr{1} & \bl{0} & 0 &  & \gr{1} & \lgr{0} \\
      \lgr{1} & 1 & \lgr{1} & \lgr{0} & \bl{1} & 1 &  & \gr{0} & \lgr{1} \\
      \lgr{1} & 1 & \lgr{1} & \lgr{0} & \bl{1} & 1 &  & \gr{1} & \lgr{0} \\
      \lgr{1} & 1 & \lgr{1} & \lgr{0} & \bl{1} & 0 &  & \gr{0} & \lgr{1} \\
      \lgr{1} & 1 & \lgr{1} & \lgr{0} & \bl{1} & 0 &  & \gr{1} & \lgr{0} \\
      \lgr{0} & 0 & \lgr{0} & \lgr{1} & \bl{1} & 1 &  & \gr{0} & \lgr{1} \\
      \lgr{0} & 0 & \lgr{0} & \lgr{1} & \bl{1} & 1 &  & \gr{1} & \lgr{0} \\
      \lgr{0} & 0 & \lgr{0} & \lgr{1} & \bl{0} & 0 &  & \gr{0} & \lgr{1} \\
      \lgr{0} & 0 & \lgr{0} & \lgr{1} & \bl{0} & 0 &  & \gr{1} & \lgr{0} \\
      \lgr{0} & 0 & \lgr{1} & \lgr{0} & \bl{1} & 1 &  & \gr{0} & \lgr{1} \\
      \lgr{0} & 0 & \lgr{1} & \lgr{0} & \bl{1} & 1 &  & \gr{1} & \lgr{0} \\
      \lgr{0} & 0 & \lgr{1} & \lgr{0} & \bl{0} & 0 &  & \gr{0} & \lgr{1} \\
      \lgr{0} & 0 & \lgr{1} & \lgr{0} & \bl{0} & 0 &  & \gr{1} & \lgr{0} \\
     \end{tabular}
   \end{center}
 }
}

\frame{\frametitle{An example}
 {\scriptsize
    \begin{center}
    \begin{tabular}{c|c|c|c|c|c|c|c|c}
     $p$ & $\wedge$ & $\neg$ & $q$ & $\vee$ & $r$ & $\rightarrow$ & $\neg$ & $s$ \\
     \hline
      \lgr{1} & \lgr{0} & \lgr{0} & \lgr{1} & 1 & \lgr{1} & \bl{0} & 0 & \lgr{1} \\
      \lgr{1} & \lgr{0} & \lgr{0} & \lgr{1} & 1 & \lgr{1} & \bl{1} & 1 & \lgr{0} \\
      \lgr{1} & \lgr{0} & \lgr{0} & \lgr{1} & 0 & \lgr{0} & \bl{1} & 0 & \lgr{1} \\
      \lgr{1} & \lgr{0} & \lgr{0} & \lgr{1} & 0 & \lgr{0} & \bl{1} & 1 & \lgr{0} \\
      \lgr{1} & \lgr{1} & \lgr{1} & \lgr{0} & 1 & \lgr{1} & \bl{0} & 0 & \lgr{1} \\
      \lgr{1} & \lgr{1} & \lgr{1} & \lgr{0} & 1 & \lgr{1} & \bl{1} & 1 & \lgr{0} \\
      \lgr{1} & \lgr{1} & \lgr{1} & \lgr{0} & 1 & \lgr{0} & \bl{0} & 0 & \lgr{1} \\
      \lgr{1} & \lgr{1} & \lgr{1} & \lgr{0} & 1 & \lgr{0} & \bl{1} & 1 & \lgr{0} \\
      \lgr{0} & \lgr{0} & \lgr{0} & \lgr{1} & 1 & \lgr{1} & \bl{0} & 0 & \lgr{1} \\
      \lgr{0} & \lgr{0} & \lgr{0} & \lgr{1} & 1 & \lgr{1} & \bl{1} & 1 & \lgr{0} \\
      \lgr{0} & \lgr{0} & \lgr{0} & \lgr{1} & 0 & \lgr{0} & \bl{1} & 0 & \lgr{1} \\
      \lgr{0} & \lgr{0} & \lgr{0} & \lgr{1} & 0 & \lgr{0} & \bl{1} & 1 & \lgr{0} \\
      \lgr{0} & \lgr{0} & \lgr{1} & \lgr{0} & 1 & \lgr{1} & \bl{0} & 0 & \lgr{1} \\
      \lgr{0} & \lgr{0} & \lgr{1} & \lgr{0} & 1 & \lgr{1} & \bl{1} & 1 & \lgr{0} \\
      \lgr{0} & \lgr{0} & \lgr{1} & \lgr{0} & 0 & \lgr{0} & \bl{1} & 0 & \lgr{1} \\
      \lgr{0} & \lgr{0} & \lgr{1} & \lgr{0} & 0 & \lgr{0} & \bl{1} & 1 & \lgr{0} \\
     \end{tabular}
   \end{center}
 }
}

\frame{\frametitle{An example}
 {\scriptsize
    \begin{center}
    \begin{tabular}{c|c|c|c|c|c|c|c|c}
     $p$ & $\wedge$ & $\neg$ & $q$ & $\vee$ & $r$ & $\rightarrow$ & $\neg$ & $s$ \\
     \hline
      \lgr{1} & \lgr{0} & \lgr{0} & \lgr{1} & \lgr{1} & \lgr{1} & \bl{0} & \lgr{0} & \lgr{1} \\
      \lgr{1} & \lgr{0} & \lgr{0} & \lgr{1} & \lgr{1} & \lgr{1} & \bl{1} & \lgr{1} & \lgr{0} \\
      \lgr{1} & \lgr{0} & \lgr{0} & \lgr{1} & \lgr{0} & \lgr{0} & \bl{1} & \lgr{0} & \lgr{1} \\
      \lgr{1} & \lgr{0} & \lgr{0} & \lgr{1} & \lgr{0} & \lgr{0} & \bl{1} & \lgr{1} & \lgr{0} \\
      \lgr{1} & \lgr{1} & \lgr{1} & \lgr{0} & \lgr{1} & \lgr{1} & \bl{0} & \lgr{0} & \lgr{1} \\
      \lgr{1} & \lgr{1} & \lgr{1} & \lgr{0} & \lgr{1} & \lgr{1} & \bl{1} & \lgr{1} & \lgr{0} \\
      \lgr{1} & \lgr{1} & \lgr{1} & \lgr{0} & \lgr{1} & \lgr{0} & \bl{0} & \lgr{0} & \lgr{1} \\
      \lgr{1} & \lgr{1} & \lgr{1} & \lgr{0} & \lgr{1} & \lgr{0} & \bl{1} & \lgr{1} & \lgr{0} \\
      \lgr{0} & \lgr{0} & \lgr{0} & \lgr{1} & \lgr{1} & \lgr{1} & \bl{0} & \lgr{0} & \lgr{1} \\
      \lgr{0} & \lgr{0} & \lgr{0} & \lgr{1} & \lgr{1} & \lgr{1} & \bl{1} & \lgr{1} & \lgr{0} \\
      \lgr{0} & \lgr{0} & \lgr{0} & \lgr{1} & \lgr{0} & \lgr{0} & \bl{1} & \lgr{0} & \lgr{1} \\
      \lgr{0} & \lgr{0} & \lgr{0} & \lgr{1} & \lgr{0} & \lgr{0} & \bl{1} & \lgr{1} & \lgr{0} \\
      \lgr{0} & \lgr{0} & \lgr{1} & \lgr{0} & \lgr{1} & \lgr{1} & \bl{0} & \lgr{0} & \lgr{1} \\
      \lgr{0} & \lgr{0} & \lgr{1} & \lgr{0} & \lgr{1} & \lgr{1} & \bl{1} & \lgr{1} & \lgr{0} \\
      \lgr{0} & \lgr{0} & \lgr{1} & \lgr{0} & \lgr{0} & \lgr{0} & \bl{1} & \lgr{0} & \lgr{1} \\
      \lgr{0} & \lgr{0} & \lgr{1} & \lgr{0} & \lgr{0} & \lgr{0} & \bl{1} & \lgr{1} & \lgr{0} \\
     \end{tabular}
   \end{center}
 }
}

\frame{\frametitle{Assignments: a contingent example}
 {\scriptsize
    \begin{center}
    \begin{tabular}{c|c|c|c|c|c|c|c|c}
     $p$ & $\wedge$ & $\neg$ & $q$ & $\vee$ & $r$ & $\rightarrow$ & $\neg$ & $s$ \\
     \hline
      \gr{1} & \lgr{0} & \lgr{0} & \gr{1} & \lgr{1} & \gr{1} & \bl{0} & \lgr{0} & \gr{1} \\
      \gr{1} & \lgr{0} & \lgr{0} & \gr{1} & \lgr{1} & \gr{1} & \bl{1} & \lgr{1} & \gr{0} \\
      \gr{1} & \lgr{0} & \lgr{0} & \gr{1} & \lgr{0} & \gr{0} & \bl{1} & \lgr{0} & \gr{1} \\
      \gr{1} & \lgr{0} & \lgr{0} & \gr{1} & \lgr{0} & \gr{0} & \bl{1} & \lgr{1} & \gr{0} \\
      \gr{1} & \lgr{1} & \lgr{1} & \gr{0} & \lgr{1} & \gr{1} & \bl{0} & \lgr{0} & \gr{1} \\
      \gr{1} & \lgr{1} & \lgr{1} & \gr{0} & \lgr{1} & \gr{1} & \bl{1} & \lgr{1} & \gr{0} \\
      \gr{1} & \lgr{1} & \lgr{1} & \gr{0} & \lgr{1} & \gr{0} & \bl{0} & \lgr{0} & \gr{1} \\
      \gr{1} & \lgr{1} & \lgr{1} & \gr{0} & \lgr{1} & \gr{0} & \bl{1} & \lgr{1} & \gr{0} \\
      \gr{0} & \lgr{0} & \lgr{0} & \gr{1} & \lgr{1} & \gr{1} & \bl{0} & \lgr{0} & \gr{1} \\
      \gr{0} & \lgr{0} & \lgr{0} & \gr{1} & \lgr{1} & \gr{1} & \bl{1} & \lgr{1} & \gr{0} \\
      \gr{0} & \lgr{0} & \lgr{0} & \gr{1} & \lgr{0} & \gr{0} & \bl{1} & \lgr{0} & \gr{1} \\
      \gr{0} & \lgr{0} & \lgr{0} & \gr{1} & \lgr{0} & \gr{0} & \bl{1} & \lgr{1} & \gr{0} \\
      \gr{0} & \lgr{0} & \lgr{1} & \gr{0} & \lgr{1} & \gr{1} & \bl{0} & \lgr{0} & \gr{1} \\
      \gr{0} & \lgr{0} & \lgr{1} & \gr{0} & \lgr{1} & \gr{1} & \bl{1} & \lgr{1} & \gr{0} \\
      \gr{0} & \lgr{0} & \lgr{1} & \gr{0} & \lgr{0} & \gr{0} & \bl{1} & \lgr{0} & \gr{1} \\
      \gr{0} & \lgr{0} & \lgr{1} & \gr{0} & \lgr{0} & \gr{0} & \bl{1} & \lgr{1} & \gr{0} \\
     \end{tabular}
   \end{center}
 }
}

\frame{\frametitle{Tautology}
\begin{itemize}
  \item<1-> take $p\vee\neg p$\\
  \medskip
  \item<2-> truth-table: \begin{tabular}{c|c|c|c}
                           $p$ & $\vee$ & $\neg$ & $p$\\
                           \hline
                            1 & \bl{1} & \gr{0} & 1 \\
                            0 & \bl{1} & \gr{1} & 0 \\
                         \end{tabular}
  \item<3-> true under every assignment, it is \textbf{valid}
  \item<4-> by \emph{law of excluded middle}: for every $P$, $P\vee\neg P$ is true
\end{itemize}
}

\frame{\frametitle{Contradiction}
\begin{itemize}
  \item<1-> take $p\wedge\neg p$\\
  \medskip
  \item<2-> truth-table: \begin{tabular}{c|c|c|c}
                           $p$ & $\wedge$ & $\neg$ & $p$\\
                           \hline
                            1 & \bl{0} & \gr{0} & 1 \\
                            0 & \bl{0} & \gr{1} & 0 \\
                         \end{tabular}
  \item<3-> false under every assignment, called \textbf{contradictory}
\end{itemize}
}

\frame{\frametitle{Contingency}
\begin{itemize}
  \item<1-> take $p\wedge p$\\
  \medskip
  \item<2-> truth-table: \begin{tabular}{c|c|c}
                           $p$ & $\wedge$ & $p$\\
                           \hline
                            1 & \bl{1} & 1 \\
                            0 & \bl{0} & 0 \\
                         \end{tabular}
  \item<3-> the truth value depends on the assignemt	
\end{itemize}
}

\subsection{Laws of the PropC}
\frame{\frametitle{What are laws?}
\begin{itemize}
  \item<1-> notice: similarities of set theory and logic
  \item<2-> non-trivial exact nature of their equivalence
  \item<3-> laws state equivalences of (types of) wff
  \item<4-> truth-conservative rewriting of wff's
  \item<5-> any subformula which is a tautology (\bl{T}) or contradiction (\bl{F}):\\
  
  \medskip
  ignore by \bl{Identity} Laws (Id.):\\
     \begin{itemize}
       \item<6-> $(P\vee F)\Leftrightarrow P$, $(P\vee T)\Leftrightarrow T$
       \item<7-> $(P\wedge F)\Leftrightarrow F$, $(P\wedge T)\Leftrightarrow P$
     \end{itemize}
\end{itemize}
}

\frame{\frametitle{Equivalences: $\Leftrightarrow$}
\begin{itemize}
  \item<1-> \textcolor{blue}{X $\Leftrightarrow$ Y}: X has the same truth-conditions as Y
  \item<2-> derivability of laws and rules (convenient redundancies)\\
  
  \medskip
  \item<3-> \textcolor{blue}{Idempotency (Idemp.)}: 
      \begin{itemize}
        \item<4-> $(P\vee P)\Leftrightarrow P$
        \item<5-> $(P\wedge P)\Leftrightarrow P$
        \item<6-> \emph{Peter walks and Peter walks.} $\Leftrightarrow$ \emph{Peter walks.}
      \end{itemize}
\end{itemize}
}

\frame{\frametitle{Simple laws}
\begin{itemize}
  \item<1-> \bl{Associative Laws for $\vee$ and $\wedge$ (Assoc.)}:
    \begin{itemize}
      \item<2-> \bl{$((P\vee Q)\vee R) \Leftrightarrow (P\vee (Q\vee R))$}
      \item<3-> \emph{((He walks or she talks) or we walk.) $\Leftrightarrow$ \\(He walks or (she talks or we walk.))}
    \end{itemize}
  \item<4-> \bl{Commutative Laws for $\vee$ and $\wedge$ (Comm.)}:
    \begin{itemize}
      \item<5-> \bl{$(P\vee Q) \Leftrightarrow (Q\vee P)$}
      \item<6-> \emph{Peter walks or Sue snores.} $\Leftrightarrow$ \emph{Sue snores or Peter walks.}
    \end{itemize}
  \item<7-> \bl{Distributive Laws for $\vee\wedge$ and $\wedge\vee$ (Distr.)}:
    \begin{itemize}
      \item<8-> \bl{$(P\vee(Q\wedge R)) \Leftrightarrow ((P\vee Q)\wedge(P\vee R))$}
      \item<9-> {\footnotesize \emph{(Sue snores) and (Peter walks or we talk).}\\ $\Leftrightarrow$ \emph{(Sue snores and Peter walks) or (Sue snores and we talk).}}
    \end{itemize}
\end{itemize}
}

\frame{\frametitle{Laws dealing with tautology and contradiction}
\begin{itemize}
  \item<1-> \bl{Complement Laws}:
    \begin{itemize}
      \item<2-> \gr{Tautology (T): $(P\vee\neg P)\Leftrightarrow \textbf{T}$}
      \item<3-> \gr{Contradiction (F): $(P\wedge\neg P)\Leftrightarrow \textbf{F}$}
      \item<4-> Double Negation (DN): \bl{$(\neg\neg P)\Leftrightarrow P$}
      \item<5-> \emph{It is not the case that Sandy is not walking.} \\ $\Leftrightarrow$ \emph{Sandy is walking.}
    \end{itemize}
\end{itemize}
}

\frame{\frametitle{Conditionals Laws}
\begin{itemize}
      \item<1-> \bl{\textbf{Implication} (Impl.)}:\\
          {\scriptsize\begin{tabular}{ccc|c|cccc}
          $P$ & $\rightarrow$ & $Q$ & $\Leftrightarrow$ & $\neg$ & $P$ & $\vee$ & $Q$ \\
          \hline
          1 & \bl{1} & 1 &   & 0 & \gr{1} & \bl{1} & 1 \\
          1 & \bl{0} & 0 &   & 0 & \gr{1} & \bl{0} & 0 \\
          0 & \bl{1} & 1 &   & 1 & \gr{0} & \bl{1} & 1 \\
          0 & \bl{1} & 0 &   & 1 & \gr{0} & \bl{1} & 0\\
        \end{tabular}}
        
        \bigskip
      \item<2-> \bl{\textbf{Contraposition} (Contr.)}:\\
          {\scriptsize\begin{tabular}{ccc|c|ccccc}
          $P$ & $\rightarrow$ & $Q$ & $\Leftrightarrow$ & $\neg$ & $Q$ & $\rightarrow$ & $\neg$ & $P$ \\
          \hline
          1 & \bl{1} & 1 &   & 0 & \gr{1} & \bl{1} & 0 & \gr{1} \\
          1 & \bl{0} & 0 &   & 1 & \gr{0} & \bl{0} & 0 & \gr{1} \\
          0 & \bl{1} & 1 &   & 0 & \gr{1} & \bl{1} & 1 & \gr{0} \\
          0 & \bl{1} & 0 &   & 1 & \gr{0} & \bl{1} & 1 & \gr{0} \\
        \end{tabular}}
\end{itemize}
}

\frame{\frametitle{DeMorgan (DeM)}
\begin{itemize}
  \item<1-> \bl{DeMorgan's Laws}:
    \begin{itemize}
      \item<2-> \bl{$\neg(P\vee Q)\Leftrightarrow (\neg P\wedge\neg Q)$}
      \item<3-> alternatively: $\overline{P\vee Q}\Leftrightarrow\overline{P}\wedge\overline{Q}$
      \item<4-> \bl{$\neg(P\wedge Q)\Leftrightarrow (\neg P\vee\neg Q)$}
      \item<5-> consequently: $\overline{\overline{P}\vee\overline{Q}}\Leftrightarrow\overline{\overline{P}}\wedge\overline{\overline{Q}}\Leftrightarrow P\wedge Q$
    \end{itemize}
\end{itemize}
}

\subsection{Rules of Inference}

\frame{\frametitle{The Modus Ponens (MP)}
    \begin{itemize}
      \item Definition:\\{\footnotesize\begin{tabular}{|ccc|c|}
              \hline
              $P$ & $\rightarrow$ & $Q$ & \gr{premise 1} \\
              P & & & \gr{premise 2}\\
              \hline
              & & Q & \gr{conclusion}\\
              \hline
            \end{tabular}}\\
      
      \medskip
      \item<2-> or: $(P\rightarrow Q) \wedge (P) \rightarrow (Q)$
      \item<3-> \emph{(1) If It rains, the streets get wet. (2) It is raining.\\ $\rightarrow$ The streets are getting wet.}
    \end{itemize}
}
\frame{\frametitle{MP: a truth table illustration}
\begin{itemize}
  \item<1-> Premises are always set to be true!
  \item<2-> the table:
  
  \medskip
  \begin{tabular}{lll}
   $P$ & $\rightarrow$ & $Q$ \\
   1 & 1 & 1 \\
   1 & 0 & 0 \\
   0 & 1 & 1 \\
   0 & 1 & 0 \\
  \end{tabular}
\end{itemize}
}

\frame{\frametitle{MP: a truth table illustration}
\begin{itemize}
  \item<1-> The conditional must be true.
  \item<1> cancel the `false' row\\
  
  \medskip
  \begin{tabular}{lll}
   $P$ & $\rightarrow$ & $Q$ \\
   1 & 1 & 1 \\
   \gr{1} & \gr{0} & \gr{0} \\
   0 & 1 & 1 \\
   0 & 1 & 0 \\
  \end{tabular}
\end{itemize}
}

\frame{\frametitle{MP: a truth table illustration}
\begin{itemize}
  \item<1-> $P$ must be true.
  \item<1-> cancel the `false' rows, Q can only be true:\\
  
  \medskip
  \begin{tabular}{lll}
   $P$ & $\rightarrow$ & $Q$ \\
   1 & 1 & \textcolor{blue}{1} \\
   \gr{1} & \gr{0} & \gr{0} \\
   \gr{0} & \gr{1} & \gr{1} \\
   \gr{0} & \gr{1} & \gr{0} \\
  \end{tabular}
\end{itemize}
}

\frame{\frametitle{The Modus Tollens (MT)}
\begin{itemize}
      \item<1-> Definition:\\{\footnotesize\begin{tabular}{|lll|}
              \hline
              $P$ & $\rightarrow$ & $Q$ \\
               & & $\neg Q$\\
              \hline
              $\neg P$ & & \\
              \hline
            \end{tabular}}\\
      
      \medskip
      \item<2-> the table illustration:\\
      
       \medskip
       {\footnotesize\begin{tabular}{cccc}
       $P$ & $\rightarrow$ & $Q$ &  \\
        \gr{1} & \gr{1} & \gr{1} & (by premise 2) \\
        \gr{1} & \gr{0} & \gr{0} & (by premise 1) \\
        \gr{0} & \gr{1} & \gr{1} & (by premise 2) \\
        \textcolor{blue}{0} & 1 & 0 & \\
       \end{tabular}}
\end{itemize}
}

\frame{\frametitle{The Syllogisms}
\begin{itemize}
  \item<1-> \bl{Hypothetical Syllogism (HS)}:
    \begin{itemize}
      \item \bl{$((P\rightarrow Q) \wedge (Q\rightarrow R)) \rightarrow (P\rightarrow R)$}
      \item \emph{(1) If it rains, the streets get wet. (2) If the streets get wet,\\it smells nice.
      	$\rightarrow$ If it rains, it smells nice.}
    \end{itemize}
  \item<2-> \bl{Disjunctive Syllogism (DS)}:
    \begin{itemize}
      \item \bl{$((P\vee Q) \wedge (\neg P)) \rightarrow (Q)$}
      \item \emph{(1) Either Peter sleeps or Peter is awake. (2) Peter isn't awake.\\
      $\rightarrow$ Peter sleeps.}
    \end{itemize}
\end{itemize}
}

\frame{\frametitle{Trivial rules}
\begin{itemize}
  \item<1-> \bl{Simplification (Simp.)}:
    \begin{itemize}
      \item \bl{$(P\wedge Q)\rightarrow P$}
      \item \emph{(1) It is raining and the sun is shining. $\rightarrow$ It is raining.}
    \end{itemize}
  \item<2-> \bl{Conjunction (Conj.)}:
    \begin{itemize}
      \item \bl{$(P) \wedge (Q) \rightarrow (P\wedge Q)$}
      \item \emph{(1) It is raining. (2) The sun is shining. $\rightarrow$ It is raining and the sun is shining.}
    \end{itemize}
  \item<3-> \bl{Addition (Add.)}:
    \begin{itemize}
      \item \bl{$(P) \rightarrow (P\wedge Q)$}
      \item \emph{(1) It is raining. $\rightarrow$ It is raining or the sun is shining.}
      \item What if Q is instantiated as true or false by another premise?
    \end{itemize}
\end{itemize}
}

\subsection{Proof}
\frame{\frametitle{A sample proof}
\begin{itemize}
  \item<1-> Prove \bl{$p\vee q$} from \bl{$(p \vee q) \rightarrow \neg (r \wedge \neg s)$} and \bl{$r \wedge \neg s$}
  \item<2-> The proof:\\
  
  \medskip
    \begin{tabular}{llll}
      &  &  & $p\vee q$ \\
     1 & $(p\vee q)\rightarrow\neg (r\wedge \neg s)$ & & \\
     2 & $r\wedge\neg s$ & &  \\
     \hline
     & $p\vee q$    & & 1,2,MT \\
    \end{tabular}
\end{itemize}
}

