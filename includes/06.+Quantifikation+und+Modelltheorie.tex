\section{Von Prädikatenlogik zu natürlicher Sprache}

\begin{frame}
  {Zur Erinnerung}
  \onslide<+->
  \onslide<+->
  Semantik von Fragment F1\\
  \Halbzeile
  \begin{itemize}[<+->]
    \item Namen referieren auf \alert{spezifische Individuen}
    \item intransitive Verben referieren auf \alert{Mengen von Individuen}
    \item mehrstellige Verben referieren auf Mengen von \alert{Tupeln von Individuen}
    \item Sätze referieren auf \alert{Wahrheitswerte}!
      \Halbzeile
    \item F2 | Integration von Erkenntnissen aus Prädikatenlogik
  \end{itemize}
  \onslide<+->
  \Zeile
  \centering 
  Alles Wesentliche dieser Sitzung in \citet[Kapitel~3]{ChierchiaMcconnellginet2000}
\end{frame}

\begin{frame}
  {Das Problem mit Pronomina}
  \onslide<+->
  \onslide<+->
  Wie situationsabhängige Namen\\
  \Halbzeile
  \begin{itemize}[<+->]
    \item[ ] \textit{\alert{This} is red.}
    \item Pronomen \alert{\textit{this}} | syntaktisch eine NP
    \item \ldots\ und referiert auf \alert{ein spezifisches Objekt} (wie Namen)\\
      \grau{\footnotesize keine Quantifikation bzw. Mengenreferenz}
      \Halbzeile
    \item Aber \orongsch{nur in gegebener Situation interpretierbar}\\
      \grau{\footnotesize Deixis, im Text auch Anaphorik}
    \item Kein Äquivalent in klassischer Logik
  \end{itemize}
\end{frame}

\begin{frame}
  {Pronomina und Variablen}
  \onslide<+->
  \onslide<+->
  Ähnlichkeit von Variablen und Pronominalausdrücken\\
  \Halbzeile
  \begin{itemize}[<+->]
    \item Rumpf einer quantifizierten Wff | Wff $P(x)$ aus Wff $(\forall x)Px$
    \item Ungebundenes $x$ in $P(x)$ \alert{ähnlich wie Pronominalbedeutung}\\
      \grau{\footnotesize Externe Interpretationsvorschrift erforderlich}
    \Halbzeile
  \item Quantoren | Auswertungsalgorithmus\\
    \grau{\footnotesize Für alle möglichen belegungen von $x$, $P(x)$}
  \item Pronomina | Kontextuelle Auswertung\\
    \grau{\footnotesize Belegung für $x$ im gegebenen Kontext}
  \end{itemize}
\end{frame}

\begin{frame}
  {Prädikatenlogik | Syntax}
  \onslide<+->
  \onslide<+->
  Als Vorüberlegung | Prädikatenlogik als \alert{Phrasenstrukturgrammatik}\\
  \Halbzeile
  \begin{itemize}[<+->]
    \item[ ] $a\ \rightarrow\ const, var$ \grau{| Individuenausdrücke}
    \item[ ] $conn\ \rightarrow\ \wedge,\vee,\rightarrow,\leftrightarrow$ \grau{| Funktoren}
    \item[ ] $neg\ \rightarrow\ \neg$ \grau{| Negation}
    \item[ ] $Q\ \rightarrow\ \exists,\forall$ \grau{| nur zwei Quantoren}
    \item[ ] $pred^1\ \rightarrow\ P, Q$ \grau{| einstellige Prädikate}
    \item[ ] $pred^2\ \rightarrow\ R$ \grau{| zweistellige Prädikate}
    \item[ ] $pred^3\ \rightarrow\ S$ \grau{| dreistellige Prädikate}
    \item[ ] $const\ \rightarrow\ b, c$ \grau{| nur zwei Individenkonstanten}
    \item[ ] $var\ \rightarrow\ x_1,x_2,\cdots x_n$ \grau{| beliebig viele Variablen}
      \Halbzeile
    \item \grau{Die Formalisierung ist äquivalent zur mengenbasierten von letzter Woche!}
  \end{itemize}
\end{frame}

\begin{frame}
  {Prädikatenlogik | PS-Regeln}
  \onslide<+->
  \onslide<+->
  Wir nehmen eine \alert{Prädikatsnotation ohne Klammern} | $Px$ statt $P(x)$ usw.\\
  \Halbzeile
  \begin{itemize}[<+->]
    \item $wff\rightarrow pred^1\ a_1\ldots\ a_n$ \grau{| n-stellige Prädikate und ihre Argumente}
    \item $wff\rightarrow neg\ wff$ \grau{| Applikation von Negation auf Wffs}
    \item $wff\rightarrow wff\ conn\ wff$ \grau{| Applikation von anderen Funktoren auf Wffs}
    \item $wff\rightarrow Q\ var\ wff$ \grau{| Quantifikation}
  \end{itemize}
\end{frame}

\begin{frame}
  {Eine Wff ohne Quantoren}
  \onslide<+->
  \onslide<+->
  Zum Beispiel: \textit{Ben ($b$) paddelt ($P$) und ($\wedge$) Ben rudert ($R$) nicht ($\neg$) mit Chris ($c$).}\\
  In PL: \alert{$Pb\wedge\neg Rbc$}\\
  \onslide<+->
  \Zeile
  \centering
  \scalebox{0.8}{\begin{forest}
    [$wff$, calign=child, calign child=2
      [$wff$
        [$pred^1$
          [$P$]
        ]
        [$a$
          [$const$
            [$b$]
          ]
        ]
      ]
      [$conn$
        [$\wedge$]
      ]
      [$wff$
        [$\neg$]
        [$wff$, calign=child, calign child=2
          [$pred^2$
            [$R$]
          ]
          [$a$
            [$const$
              [$b$]
            ]
          ]
          [$a$
            [$const$
              [$c$]
            ]
          ]
        ]
      ]
    ]
  \end{forest}}
\end{frame}

\begin{frame}
  {Eine Wff mit Quantoren}
  \onslide<+->
  \onslide<+->
  Zum Beispiel: \textit{Als Paddler hat man immer jemanden, mit dem man nicht rudert.}\\
  In PL: \alert{$\forall x_1[Px_1\rightarrow\exists x_2\neg Px_1x_2]$}\\
  \onslide<+->
  \Halbzeile
  \centering
  \scalebox{0.7}{\begin{forest}
    [$wff$
      [$Q\ var$
        [$\forall x_1$]
      ]
      [$wff$, calign=child, calign child=2
        [$wff$
          [$pred^1$
            [$P$]
          ]
          [$a$
            [$var$
              [$x_1$]
            ]
          ]
        ]
        [$conn$
          [$\rightarrow$]
        ]
        [$wff$
          [$Q\ var$
            [$\exists x_2$]
          ]
          [$wff$
            [$\neg$]
            [$wff$, calign=child, calign child=2
              [$pred^2$
                [$R$]
              ]
              [$a$
                [$const$
                  [$x_1$]
                ]
              ]
              [$a$
                [$const$
                  [$x_2$]
                ]
              ]
            ]
          ]
        ]
      ]
    ]
  \end{forest}}
\end{frame}

\begin{frame}
  {Skopus und c-Kommando}
  Skopus in konfigurationaler Logik-Syntax: \alert{c-Kommando}\\
  Variablen als \alert{gebunden vom nächsten c-kommandierenden koindizierten Quantor}\\
  \Halbzeile
  \centering
  \scalebox{0.6}{\begin{forest}
    [$wff$
      [$Q\ var$
        [$\forall x_1$]
      ]
      [$wff$, calign=child, calign child=2, gruentree
        [$wff$
          [$pred^1$
            [$P$]
          ]
          [$a$
            [$var$
              [$x_1$]
            ]
          ]
        ]
        [$conn$
          [$\rightarrow$]
        ]
        [$wff$
          [$Q\ var$
            [$\exists x_2$]
          ]
          [$wff$, bluetree
            [$\neg$]
            [$wff$, calign=child, calign child=2
              [$pred^2$
                [$R$]
              ]
              [$a$
                [$const$
                  [$x_1$]
                ]
              ]
              [$a$
                [$const$
                  [$x_2$]
                ]
              ]
            ]
          ]
        ]
      ]
    ]
  \end{forest}}\\
  \visible<2->{\footnotesize\alert{Skopus\slash c-Kommando-Domäne von $\exists x_2$}}\visible<3->{ | \footnotesize\gruen{Skopus\slash c-Kommando-Domäne von $\forall x_1$} (zgl.\ \alert{derer von $\exists x_2$})}\\
\end{frame}

\section{Modelltheorie}

\begin{frame}
  {Semantik für PL in Vorbereitung auf natürliche Sprache}
  \onslide<+->
  \onslide<+->
  Ziel (zur Erinnerung) | T-Sätze der Form \textit{S aus L ist wahr in v gdw \ldots}\\
  \Halbzeile
  \begin{itemize}[<+->]
    \item \alert{Modell $\Model$} | zugängliches Diskursuniversum (bzw.\ dessen Beschreibung)
    \item \alert{Menge $D_n$} | Zugängliche Individuen (\textit{domain}) in $\Model_n$
    \item \alert{Funktion $V_n$} | Valuation -- Zuweisung von
      \begin{itemize}[<+->]
        \item Namen zu Individuen in $\Model_n$
        \item Predikaten zu Tupeln von Individuen
      \end{itemize}
    \item \alert{$\Model_n=\tuple{D_n,V_n}$}
      \Halbzeile
    \item \alert{Funktion $g_n$} | Zuweisung von Variablen zu Individuen in $\Model_n$ 
      \Halbzeile
    \item Allgemeine Evaluation in $\Model_n$ | $\dem{\alpha}^{\Model_n,g_n}$\\
      \grau{Lies: \textit{Die Extension von Ausdruck $\alpha$ relativ zu $\Model_n$ und $g_n$}}
  \end{itemize}
\end{frame}

\begin{frame}
  {Unterschied zwischen $V_n$ und $g_n$}
  \onslide<+->
  \onslide<+->
  Feste und variable Denotation\\
  \Halbzeile
  \begin{itemize}[<+->]
    \item $V_n$ evaluiert \alert{statisch} im Modell.\\
      \grau{\footnotesize Wenn das Modell einmal feststeht, evaluiert $V_n$ jede Konstante stets gleich.}
      \Halbzeile
    \item Variablen (gebunden durch Quantoren) werden \alert{volatil interpretiert}.\\
    \item \alert{Iteration} durch Universum $D_n$ durch $g_n$
    \item Eine Modifikation der Belegung pro Iteration
      \begin{itemize}[<+->]
        \item Modifizierte \textit{assignment function} \alert{$g_n[d_i/x_m]$}\\
          Lies: \textit{relativ zu $g_n$, wobei die Referenz von Variable $x_m$ auf Individuum $d_i$ gesetzt wird}
      \end{itemize}
  \end{itemize}
\end{frame}

\begin{frame}
  {Evaluation von Variablen}
  \onslide<+->
  \onslide<+->\scriptsize
  $\alert{D_1}=\{Herr\ Webelhuth, Frau\ Klenk, Turm-Mensa\}$ \grau{| Individuen in $\Model_1$}\\
  \onslide<+->
  $\alert{V_1(P)}=\{Herr\ Webelhuth, Frau\ Klenk, Turm-Mensa\}$ \grau{| Prädikat $P$ (\zB \textit{ist ein physikalisches Objekt}) in $\Model_1$}\\
  \onslide<+->
  Evaluiere \alert{$\dem{\forall x_1Px_1}^{\Model_1,g_1}$}\visible<8->{$=\gruen{1}$ weil keiner Belegung $\dem{Px_1}^{\Model_1,g_1}=\orongsch{0}$}\\
  \Halbzeile
  \begin{itemize}[<+->]
    \item Initiale Belegung \alert{$\dem{x_1}^{\Model_1,g_1}=Herr\ Webelhuth$}\\
      \scalebox{0.7}{$g_1 = \left[\begin{array}{l}
          \gruen<5>{x_1 \rightarrow Herr\ Webelhuth}\\
            x_2 \rightarrow Herr\ Webelhuth\\
            x_3 \rightarrow Herr\ Webelhuth
        \end{array}\right]$}\\
        \Viertelzeile
          $\dem{Px_1}^{\Model_1,g_1}=\gruen{1}$
          \Halbzeile
        \item \alert{$\dem{x_1}^{\Model_1,g_1[Klenk/x_1]}=Frau\ Klenk$}\\
         \scalebox{0.7}{$g_1 = \left[\begin{array}{l}
             \gruen<6>{x_1 \rightarrow Frau\ Klenk}\\
            x_2 \rightarrow Herr\ Webelhuth\\
            x_3 \rightarrow Herr\ Webelhuth
        \end{array}\right]$}\\
        \Viertelzeile
          $\dem{Px_1}^{\mMm_1,g_1\ekm{Klenk/x_1}}=\gruen{1}$
          \Halbzeile
        \item \alert{$\dem{x_1}^{\Model_1,g_1[Turm-Mensa/X_1]}=Turm-Mensa$}\\
         \scalebox{0.7}{$g_1 = \left[\begin{array}{l}
             \gruen<7>{x_1 \rightarrow Turm-Mensa}\\
            x_2 \rightarrow Herr\ Webelhuth\\
            x_3 \rightarrow Herr\ Webelhuth
        \end{array}\right]$}\\
        \Viertelzeile
          $\dem{Px_1}^{\mMm_1,g_1\ekm{Mensa/x_1}}=\gruen{1}$
  \end{itemize}
\end{frame}

\begin{frame}
  {Evaluation mit zwei Variablen}
  \onslide<+->
  \onslide<+->\scriptsize
  $\alert{D_1}=\{Herr\ Webelhuth, Frau\ Klenk, Turm-Mensa\}$ \grau{| Individuen in $\Model_1$}\\
  \onslide<+->
  $\alert{V_1(Q)}=\{\tuple{Webelhuth,Klenk},\tuple{Webelhuth,Mensa},\tuple{Klenk,Webelhuth}\}$ \grau{| Prädikat $Q$ (\zB \textit{x besucht y}) in $\Model_1$}\\
  \onslide<+->
  Evaluiere \alert{$\dem{\forall x_1\exists x_2 Qx_1x_2}^{\Model_1,g_1}\visible<18->{=\orongsch{0}}$} \visible<18->{weil nicht für jede Belegung von $x_1$ mindestens einmal \gruen{1}}\\
  \onslide<+->
  \Zeile
  \begin{minipage}{0.5\textwidth}\begin{itemize}[<+->]
        \item Initiale Belegung $\dem{x_1}^{\Model_1,g_1}=Frau\ Klenk$
          \begin{itemize}[<+->]\scriptsize
            \item $\dem{Qx_1x_2}^{\mMm_1,g_1}=\orongsch{0}$
            \item $\dem{Qx_1x_2}^{\mMm_1,g_1[\gruen<8>{Klenk/x_2}]}=\orongsch{0}$
            \item $\dem{Qx_1x_2}^{\mMm_1,g_1[\gruen<9>{Webelhuth/x_2}]}=\gruen{1}$
          \end{itemize}
        \item $\dem{x_1}^{\Model_1,g_1[Turm-Mensa/x_1]}=Turm-Mensa$
          \begin{itemize}[<+->]\scriptsize
            \item $\dem{Qx_1x_2}^{\mMm_1,g_1[\tuerkis<11-13>{Turm-Mensa/x_1}]}=\orongsch{0}$
            \item $\dem{Qx_1x_2}^{\mMm_1,g_1[\tuerkis<11-13>{Turm-Mensa/x_1},\gruen<12>{Klenk/x_2}]}=\orongsch{0}$
            \item $\dem{Qx_1x_2}^{\mMm_1,g_1[\tuerkis<11-13>{Turm-Mensa/x_1},\gruen<13>{Webelhuth/x_2}]}=\orongsch{0}$ \rot{Abbruch!}
          \end{itemize}
        \item $\dem{x_1}^{\Model_1,g_1[Webelhuth/x_1]}=Herr\ Webelhuth$
          \begin{itemize}[<+->]\scriptsize
            \item $\dem{Qx_1x_2}^{\mMm_1,g_1[\tuerkis<15-17>{Webelhuth/x_1}]}=\gruen{1}$
            \item $\dem{Qx_1x_2}^{\mMm_1,g_1[\tuerkis<15-17>{Webelhuth/x_1},\gruen<16>{Klenk/x_2}]}=\gruen{1}$
            \item $\dem{Qx_1x_2}^{\mMm_1,g_1[\tuerkis<15-17>{Webelhuth/x_1},\gruen<17>{Webelhuth/x_2}]}=\orongsch{0}$
          \end{itemize}
    \end{itemize}\end{minipage}%
    \begin{minipage}{0.5\textwidth}%
    \centering 
    \scalebox{1.2}{$g_1 = \left[\begin{array}{l}
        x_1 \rightarrow \tuerkis<11-13,15-17>{%
          \only<5-10,14,18->{Frau\ Klenk}%
          \only<11-13>{Turm-Mensa}%
          \only<15-17>{Herr\ Webelhuth}%
        }\\
        x_2 \rightarrow \gruen<8-9,12-13,16-17>{%
          \only<5-7,10-11,14-15,18->{Turm-Mensa}%
          \only<8,12,16>{Frau\ Klenk}%
          \only<9,13,17>{Herr\ Webelhuth}
        }\\
      x_3 \rightarrow Herr\ Webelhuth\\
    \end{array}\right]$}\\
  \end{minipage}
\end{frame}

\section{Quantifikation in natürlicher Sprache}

\begin{frame}
  {Seltsame Quantoren}
  \onslide<+->
  \onslide<+->
  Wie quantifiziert \textit{meist}?\\
  \Halbzeile
  \begin{itemize}[<+->]
    \item Kleineres Problem | $\exists$ sowohl \textit{mindestens ein} als auch \textit{einige}
      \Halbzeile
    \item Grundsätzliches Problem | \textit{meist} (und andere)
      \begin{itemize}[<+->]
        \item[ ] \textit{Die meisten Patienten sind zufrieden.}
        \item Potentieller Quantor \alert{$\rotatebox[origin=c]{180}{M}$} | \alert{$\rotatebox[origin=c]{180}{M}xPx\rightarrow Zx$}\\\grau{\footnotesize Für die meisten Objekte gilt, dass sie zufrieden sind, wenn sie Patienten sind.}
        \item \orongsch{Falsche Interpretation}| Domäne = $\dem{P}^{\Model_1}\{x:x\ ist\ Patient\}$, nicht $D_1$
      \end{itemize}
      \Halbzeile
    \item Korrekte Lösung | \alert{Generalisierte Quantoren} (am Ende des Seminars)
  \end{itemize}
\end{frame}

\begin{frame}
  {Natürliche Sprache | Ambiger Skopus}
  \onslide<+->
  \onslide<+->
  In PL ist Skopus klar geregelt, in natürlicher Sprache nicht.\\
  \Halbzeile
  \begin{itemize}[<+->]
    \item c-Kommando für Skopus nicht adäquat
    \item Natürliche Sprache ohne \alert{pränexe Normalform} (PNF), Quantor in situ
    \item Außerdem \alert{Ambiguität = mehrere Lesarten}
      \begin{itemize}[<+->]
        \item \emph{Everybody loves somebody.} (\emph{ELS})
        \item $\forall{}x_1\exists{}x_2Lx_1x_2$
        \item $\exists{}x_2\forall{}x_1Lx_1x_2$
      \end{itemize}
      \Halbzeile
    \item Für eine strukturelle Modellierung (c-Kommando) | \alert{LF-Bewegung}
    \item Beispiele für andere Lösungen, mehr in Montagues lf-Tradition
      \begin{itemize}[<+->]
        \item \alert{Cooper Storage} (implementiert in HPSG)
        \item \alert{Unterspezifikation} (implementiert in HPSG; kognitiv recht plausibel)
        \item \alert{Hypothetische Beweise} (implementiert in Kategorialgrammatik)
      \end{itemize}
  \end{itemize}
\end{frame}


\begin{frame}
  {Für eine strukturelle Lösung | LF-Bewegung}
  \onslide<+->
  \onslide<+->
  Relevante syntaktische Erweiterung zu $F_1$ | \alert{Quantifier Raising (QR) Rule}\\
  \Halbzeile
  \onslide<+->
  \centering 
  {\Large \alert{$[_S\ X\ NP\ Y\ ]\ \Longrightarrow\ [_{S^{\prime}}\ NP_i\ [_S\ X\ t_i\ Y\ ]]$}}\\
  \Halbzeile
  \begin{itemize}[<+->]
    \item Phrasenstruktur als Input und Output (= Skopus in Syntax, LF als Syntax)
    \item Koindizierung und Linksadjunktion an S beide Teil einer Regel
    \item \grau{Kein wesentlicher Unterschied, falls CP oder IP statt S}
      \Halbzeile
    \item Außerdem | \alert{$Det \rightarrow every,\ some$} and \alert{$NP \rightarrow Det\ N^{count}$}
      \Halbzeile
    \item Syntax-Problem | Völlig unnötig eine \orongsch{kontextsensitive Regel}
    \item Semantik-Probleme bei Chierchia
      \begin{itemize}[<+->]
        \item Einführung syntaktischer Typen wird skizzenhaft (s.\ Montague)
        \item Definition zulässiger Modelle unterschlagen (s.\ Montague)
      \end{itemize}
  \end{itemize}
\end{frame}

\begin{frame}
  {Semantik für QR mit \textit{every}}
  \onslide<+->
  \onslide<+->
  \centering
  \alert{\Large $\Dem{\ekm{\ekm{every\ \beta}_i\ S}}{\mMm,g}=1\ iff\ for\ all\ d\in D:$\\
  $if\ d\in\Dem{\beta}{\mMm,g}\ then\ \Dem{S}{\mMm,g\ekm{u/t_i}}$}\\
  \onslide<+->
  \Zeile
  A sentence containing the trace $t_i$ with an adjoined $NP_i$ (which consists of \emph{every} plus the common noun $\beta$) extend to 1 iff for each individual $d$ in the universe $D$ which is in the set referred to by the common noun $\beta$, $S$ denotes 1 with $d$ assigned to the pronominal trace $t_i$. $g$ is modified iteratively to check that.
\end{frame}


\begin{frame}
  {Semantik für QR-Regel mit \textit{some}}
  \onslide<+->
  \onslide<+->
  \centering
  \alert{\Large $\Dem{\ekm{\ekm{a\ \beta}_i\ S}}{\mMm,g}=1\ iff\ for\ some\ u\in U:$\\
    $u\in\Dem{\beta}{\mMm,g}\ and\ \Dem{S}{\mMm,g\ekm{u/t_i}}$}\\
  \onslide<+->
  \Zeile
  Die Interpretation erfolgt nach ähnlichem Schema.
\end{frame}

\begin{frame}
  {Bäume}
  \onslide<+->
  \onslide<+->
  \textit{Martin sends \gruen{all colleagues} \orongsch{some paper}.} in the $\exists\forall$ reading:\\
  \centering
  \onslide<+->
  \Zeile 
  $\vcenter{\hbox{\scalebox{0.5}{\begin{forest}
    [$S$
      [$NP$
        [$N$
          [\textit{Martin}]
        ]
      ]
      [$VP$, calign=child, calign child=2
        [$V_{tr}$
          [\textit{sends}]
        ]
        [$NP$, gruentree
          [$Det$
            [\textit{all}]
          ]
          [$N$
            [\textit{colleagues}]
          ]
        ]
        [$NP$, orongschtree
          [$Det$
            [\textit{some}]
          ]
          [$N$
            [\textit{paper}]
          ]
        ]
      ]
    ]
  \end{forest}}}}$~\only<4->{$\Longrightarrow$}\only<5->{$\Longrightarrow$}~$\vcenter{\hbox{\scalebox{0.5}{%
    \only<4>{\begin{forest}
    [$S\Prm$
      [$NP_i$, gruentree
        [$Det$
          [\textit{all}]
        ]
        [$N$
          [\textit{colleagues}]
        ]
      ]
      [$S$
        [$NP$
          [$N$
            [\textit{Martin}]
          ]
        ]
        [$VP$, calign=child, calign child=2
          [$V_{tr}$
            [\textit{sends}]
          ]
          [$t_1$, gruentree]
          [$NP$, orongschtree
            [$Det$
              [\textit{some}]
            ]
            [$N$
              [\textit{paper}]
            ]
          ]
        ]
      ]
    ]
  \end{forest}}%
  \only<5->{\begin{forest}
    [$S\Prm$
      [$NP$, orongschtree
        [$Det$
          [\textit{some}]
        ]
        [$N$
          [\textit{paper}]
        ]
      ]
      [$S\Prm$
        [$NP_i$, gruentree
          [$Det$
            [\textit{all}]
          ]
          [$N$
            [\textit{colleagues}]
          ]
        ]
        [$S$
          [$NP$
            [$N$
              [\textit{Martin}]
            ]
          ]
          [$VP$, calign=child, calign child=2
            [$V_{tr}$
              [\textit{sends}]
            ]
            [$t_1$, gruentree]
            [$t_2$, orongschtree]
          ]
        ]
      ]
    ]
  \end{forest}}}}}$%
\end{frame}

\section{Aufgaben}

\begin{frame}
  {Aufgaben I}
\end{frame}

\end{document}
