%\documentclass{beamer}
\documentclass[handout]{beamer}

\usepackage{beamerthemesplit,
            graphicx,
			      pstricks,
			      color,
			      colortbl,
            fancyhdr,
            rotate,
            url,
            setspace,
            pifont,
            cite,
            amsmath,
            amsfonts,
            amssymb,
            gb4e,
            cgloss4e,
            tree-dvips,
            avm+,
            ecltree+,
            rotate,
            stmaryrd}

\title{Semantics{ }\\(5) Quantification in English}
\author{Roland Sch\"afer (University of G\"ottingen)}
\date{Summer Term 2005 (May 25)}

\newcommand{\den}[1]{{$\llbracket${}#1$\rrbracket$}}
\newcommand{\dem}[1]{{\llbracket{}#1\rrbracket}}
\newcommand{\Dem}[2]{\llbracket{}#1\rrbracket^{#2}}
\newcommand{\ra}[1]{{\langle{}#1\rangle}}
\newcommand{\ek}[1]{$\left[$#1$\right]$}
\newcommand{\lk}[0]{$\left[$}
\newcommand{\rk}[0]{$\right]$}
\newcommand{\ekm}[1]{\left[{}#1\right]}
\newcommand{\bl}[1]{\textcolor{blue}{#1}}
\newcommand{\rot}[1]{\textcolor{red}{#1}}
\newcommand{\gr}[1]{\textcolor{gray}{#1}}
\newcommand{\gn}[1]{\textcolor{green}{#1}}
\newcommand{\lgr}[1]{\textcolor[gray]{0.9}{#1}}
\newcommand{\mM}[0]{$\mathcal{M}$}
\newcommand{\mMm}[0]{\mathcal{M}}
\newfont{\dslash}{dsrom12}

\begin{document}

\frame{\titlepage}

\section{From PC to F1}
\subsection{Taking stock}

\frame{\frametitle{Back to semantics: F1}
\begin{itemize}
  \item<1-> before we turn to quantification in F1/F2 English:
  \item<2-> names refer to individuals
  \item<3-> itr. verbs refer to sets of individuals
  \item<4-> tr. verbs refer to sets of ordered pairs of individuals
  \item<5-> sentences refer to truth values
\end{itemize}
}

\subsection{Pronouns and context}
\frame{\frametitle{Reference of pronouns}
\begin{itemize}
  \item<1-> \emph{\textbf{This} drives a Golf.}
  \item<2-> \emph{this} = a pronominal NP
  \item<3-> denotes an \bl{individual}
  \item<4-> but \bl{not rigidly}
  \item<5-> fixed only within a specific context (SOA)
\end{itemize}
}

\frame{\frametitle{Pronouns and variables}
\begin{itemize}
  \item<1-> quantified expression: $(\forall x)Px$
  \item<2-> \emph{for all assignments of `this', `this' has property P}
  \item<3-> Q evaluation in PC is algorithmic
  \item<4-> \bl{variables interpreted like definite pronominal NPs} (within a fixed context)
\end{itemize}
}

\subsection{Phrase structure version of PC}
\frame{\frametitle{Categories and lexicon}
\begin{itemize}
  \item<1-> \emph{a} $\rightarrow$ const, var
  \item<2-> conn $\rightarrow$ $\wedge,\vee,\rightarrow,\leftrightarrow$
  \item<3-> neg $\rightarrow$ $\neg$
  \item<4-> Q  $\rightarrow$ $\exists, \forall$
\end{itemize}
}

\frame{\frametitle{Categories and lexicon}
\begin{itemize}
  \item<1-> pred_1 $\rightarrow$ P, Q
  \item<2-> pred_2 $\rightarrow$ R
  \item<3-> pred_3 $\rightarrow$ S
  \item<4-> const $\rightarrow$ b, c
  \item<5-> var $\rightarrow$ x_1, x_2, \ldots, x_n
\end{itemize}
}

\frame{\frametitle{Phrase structure}
\begin{itemize}
  \item<1-> wff $\rightarrow$ pred_n a_1 a_2 \ldots a_n
  \item<2-> wff $\rightarrow$ neg wff
  \item<3-> wff $\rightarrow$ wff con wff
  \item<4-> wff $\rightarrow$ (Q var) wff
\end{itemize}
}

\subsection{Trees}
\frame{\frametitle{A wff without Q}
 
\begin{center}
{\small
   \begin{bundle}
    {wff}\setlength\GapDepth{2ex}\setlength\GapWidth{3em}
    \chunk{\begin{bundle}
             {wff}
             \chunk{\begin{bundle}
                      {pred_1}
                      \chunk{P}
                    \end{bundle}
                   }
             \chunk{\begin{bundle}
                     {a}
                     \chunk{
                            \begin{bundle}
                              {var}
                              \chunk{b}
                            \end{bundle}
                           }
                    \end{bundle}
                   }
           \end{bundle}
          }
    \chunk{\begin{bundle}
            {conn}
            \chunk{$\wedge$}
           \end{bundle}
          }
    \chunk{\begin{bundle}
             {wff}
             \chunk{\begin{bundle}
                     {neg}
                     \chunk{$\neg$}
                    \end{bundle}
                   }
            \chunk{\begin{bundle}
                    {wff}
                    \chunk{\begin{bundle}
                             {wff}
                             \chunk{\begin{bundle}
                                     {pred_2}
                                     \chunk{R}
                                    \end{bundle} 
                                   }
                             \chunk{\begin{bundle}
                                     {a}
                                     \chunk{\begin{bundle}
                                              {const}
                                              \chunk{b}
                                            \end{bundle}
                                           }
                                    \end{bundle} 
                                   }
                             \chunk{\begin{bundle}
                                     {a}
                                     \chunk{\begin{bundle}
                                              {const}
                                              \chunk{c}
                                            \end{bundle}
                                           }
                                    \end{bundle} 
                                   }
                           \end{bundle}
                          }
                   \end{bundle}
                  }      
           \end{bundle}
          }
  \end{bundle}
 }
\end{center}
}

\frame{\frametitle{A wff with Q's}
\begin{center}
{\small
 \begin{bundle}
  {wff}\setlength\GapDepth{1.4ex}\setlength\GapWidth{1.5em}
  \chunk{\begin{bundle}
           {(Q var)}
           \chunk{$\forall$ x_1}
         \end{bundle}
        }
  \chunk{
  \begin{bundle}
    {wff}
    \chunk{\begin{bundle}
             {wff}
             \chunk{\begin{bundle}
                      {pred_1}
                      \chunk{P}
                    \end{bundle}
                   }
             \chunk{\begin{bundle}
                     {a}
                     \chunk{
                            \begin{bundle}
                              {const}
                              \chunk{x_1}
                            \end{bundle}
                           }
                    \end{bundle}
                   }
           \end{bundle}
          }
    \chunk{\begin{bundle}
            {conn}
            \chunk{$\rightarrow$}
           \end{bundle}
          }
    \chunk{\begin{bundle}
             {wff}
             \chunk{\begin{bundle}
                      {(Q var)}
                      \chunk{$\exists x_2$}
                    \end{bundle}
                   }
             \chunk{      
               \begin{bundle}
                 {wff}
                 \chunk{\begin{bundle}
                         {neg}
                         \chunk{$\neg$}
                        \end{bundle}
                   }
                \chunk{\begin{bundle}
                        {wff}
                        \chunk{\begin{bundle}
                                 {wff}
                                 \chunk{\begin{bundle}
                                         {pred_2}
                                         \chunk{R}
                                        \end{bundle} 
                                       }
                                 \chunk{\begin{bundle}
                                         {a}
                                         \chunk{\begin{bundle}
                                                  {var}
                                                  \chunk{x_1}
                                                \end{bundle}
                                               }
                                        \end{bundle} 
                                       }
                                 \chunk{\begin{bundle}
                                         {a}
                                         \chunk{\begin{bundle}
                                                  {var}
                                                  \chunk{x_2}
                                                \end{bundle}
                                               }
                                        \end{bundle} 
                                       }
                               \end{bundle}
                              }
                       \end{bundle}
                      }      
               \end{bundle}
                      }
            \end{bundle}           
          }
  \end{bundle}
  }
  \end{bundle}
  }
\end{center}
}

\subsection{C-command}
\frame{\frametitle{Definition of c-command}
\begin{itemize}
  \item<1->Node A \bl{c-commands} (constituent-commands) node B iff
    \begin{itemize}
      \item<2-> \bl{A does not dominate B} and
      \item<3-> and \bl{the first branching node dominating A also dominates B}.
    \end{itemize}
  \item<4-> The definition in CM allows a node to dominate itself.
\end{itemize}
}

\frame{\frametitle{Configurational binding}
\begin{itemize}
  \item<1-> in configurational tree-structures:
  \item<2-> \bl{A variables is bound by the closest c-commanding coindexed quantifier.}
  \item<3-> scope = binding domain
\end{itemize}
}

\frame{\frametitle{A wff with Q's}
 
\begin{center}
{\small
 \begin{bundle}
  {\node{a-2}{wff}}\setlength\GapDepth{1.4ex}\setlength\GapWidth{1.5em}
  \chunk{\begin{bundle}
           {(Q var)}
           \chunk{\node{a-1}{$\forall$ x_1}}
         \end{bundle}
        }
  \chunk{
  \begin{bundle}
    {\node{a-3}{wff}}
    \chunk{\begin{bundle}
             {\node{a-4}{wff}}
             \chunk{\begin{bundle}
                      {pred_1}
                      \chunk{P}
                    \end{bundle}
                   }
             \chunk{\begin{bundle}
                     {\node{a-5}{a}}
                     \chunk{
                            \begin{bundle}
                              {const}
                              \chunk{\node{a-6}{x_1}}
                            \end{bundle}
                           }
                    \end{bundle}
                   }
           \end{bundle}
          }
    \chunk{\begin{bundle}
            {conn}
            \chunk{$\rightarrow$}
           \end{bundle}
          }
    \chunk{\begin{bundle}
             {\node{a-8}{wff}}
             \chunk{\begin{bundle}
                      {(Q var)}
                      \chunk{\node{a-7}{$\exists x_2$}}
                    \end{bundle}
                   }
             \chunk{      
               \begin{bundle}
                 {wff}
                 \chunk{\begin{bundle}
                         {neg}
                         \chunk{$\neg$}
                        \end{bundle}
                   }
                \chunk{\begin{bundle}
                        {wff}
                        \chunk{\begin{bundle}
                                 {wff}
                                 \chunk{\begin{bundle}
                                         {pred_2}
                                         \chunk{R}
                                        \end{bundle} 
                                       }
                                 \chunk{\begin{bundle}
                                         {a}
                                         \chunk{\begin{bundle}
                                                  {var}
                                                  \chunk{\node{a-9}{x_1}}
                                                \end{bundle}
                                               }
                                        \end{bundle} 
                                       }
                                 \chunk{\begin{bundle}
                                         {a}
                                         \chunk{\begin{bundle}
                                                  {var}
                                                  \chunk{\node{a-10}{x_2}}
                                                \end{bundle}
                                               }
                                        \end{bundle} 
                                       }
                               \end{bundle}
                              }
                       \end{bundle}
                      }      
               \end{bundle}
                      }
            \end{bundle}           
          }
  \end{bundle}
  }
  \end{bundle}
  } {\makedash{2pt}
     \anodecurve[t]{a-1}[l]{a-2}{1cm}
     \anodecurve[b]{a-2}[t]{a-6}{1cm}
     \anodecurve[b]{a-2}[t]{a-9}{1cm}
     \anodecurve[t]{a-7}[l]{a-8}{1cm}
     \anodecurve[b]{a-8}[t]{a-10}{1cm}
    }
\end{center}
}

\section{Model theory}
\subsection{Models and valuations}
\frame{\frametitle{Refinement of PC semantics}
\begin{itemize}
  \item<1-> \gr{remember T-sentences: \textbf{S of L is true in v iff p}.}
  \item<2-> \textbf{\bl{$\mathcal{M}$}} is a model of the accessible universe of discourse
    \begin{itemize}
      \item<3-> $\mathcal{M}=\ra{U_n,V_n}$
      \item<4-> \bl{$U_n$} = the set of accessible individuals (\bl{domain})
      \item<5-> \bl{$V_n$} = a \bl{valuation function} which assigns
         \begin{itemize}
           \item<6-> individuals to names
           \item<7-> sets of n-tuples of indivuiduals to pred_n
         \end{itemize}
    \end{itemize}
  \item<8-> \textbf{\bl{g}} is function from variables to individuals in $\mathcal{M}$
  \item<9-> we evaluate: \bl{$\dem{\alpha}^{\mMm_n,g_n}$}
  \item<10-> \emph{the extension of $\alpha$ relative to \mM_n and g_n}
\end{itemize}
}

\subsection{Assignment functions}
\frame{\frametitle{Fixed and context-bound denotation}
\begin{itemize}
  \item<1-> V_n valuates \bl{statically}
  \item<2-> Q's require flexible valuation of pronominal matrices
  \item<3-> \emph{g_n} is like V_n for constants, only flexible
  \item<4-> it can \bl{iterate through U_n}
  \item<5-> initial assignment can be anything:
    \begin{center}
      $g_1 = \left[\begin{array}{l}
                     x_1 \rightarrow Herr\ Webelhuth\\
                     x_2 \rightarrow Frau\ Eckardt\\
                     x_3 \rightarrow Turm-Mensa
                   \end{array}
             \right]$
    \end{center}
\end{itemize}
}

\subsection{Modified assignment functions}
\frame{\frametitle{Iterating through U_n}
\begin{itemize}
  \item<1-> for each Q loop, one modification
  \item<2-> read \bl{$g_n\ekm{d/x_m}$} as\\ `\ldots relative to $g_n$ where $x_m$ is reassigned to $d$'
  \item<3-> $\dem{x_1}^{\mMm_1,g_1\ekm{Eckardt/x_1}}=Frau\ Eckardt$
  \item<4-> $\dem{x_2}^{\mMm_1,g_1\ekm{\ekm{Eckardt/x_1}Mensa/x_2}}=Mensa$
\end{itemize}
}

\frame{\frametitle{Interpreting with g_n}
\begin{itemize}
  \item<1-> \bl{$\Dem{(\forall{}x_1)Px_1}{\mMm_1,g_1}$}
  \item<2-> start with initial assignment: $\Dem{x_1}{\mMm_1,g_1}=Webelhuth$ \\
    check: $\Dem{Px_1}{\mMm_1,g_1}$
  \item<3-> modify: $\Dem{x_1}{\mMm_1,g_1\ekm{Eckardt/x_1}}=Eckardt$ \\
    check: $\Dem{Px_1}{\mMm_1,g_1}$
  \item<4-> modify: $\Dem{x_1}{\mMm_1,g_1\ekm{Mensa/x_1}}=Mensa$ \\
    check: $\Dem{Px_1}{\mMm_1,g_1}$
  \item<5-> iff the answer was never 0, then \bl{$\Dem{(\forall{}x_1)Px_1}{\mMm_1,g_1}=1$}
\end{itemize}
}

\frame{\frametitle{Multiple Q's: subloops}
{\footnotesize
\begin{itemize}
  \item<1-> \bl{$\Dem{(\forall{}x_1)(\exists{}x_2)Px_1x_2}{\mMm_1,g_1}$}
  \item<2-> $\Dem{x_1}{\mMm_1,g_1}=Webelhuth$
    \begin{itemize}
      \item<3->{\footnotesize $\Dem{x_2}{\mMm_1,g_1}=Eckardt$ }
      \item<4->{\footnotesize  $\Dem{x_2}{\mMm_1,g_1\ekm{Webelhuth/x_2}}=Webelhuth$ }
      \item<5->{\footnotesize  $\Dem{x_2}{\mMm_1,g_1\ekm{Mensa/x_2}}=Mensa$ }
    \end{itemize}
  \item<6-> $\Dem{x_1}{\mMm_1,g_1\ekm{Eckardt/x_1}}=Eckardt$
    \begin{itemize}
      \item<7->{\footnotesize  $\Dem{x_2}{\mMm_1,g_1\ekm{Eckardt/x_1}}=Eckardt$ }
      \item<8->{\footnotesize  $\Dem{x_2}{\mMm_1,g_1\ekm{\ekm{Eckardt/x_1}Webelhuth/x2}}=Webelhuth$ }
      \item<9->{\footnotesize  $\Dem{x_2}{\mMm_1,g_1\ekm{\ekm{Eckardt/x_1}Mensa/x2}}=Mensa$ }
    \end{itemize}
  \item<10-> $\Dem{x_1}{\mMm_1,g_1\ekm{Mensa/x_1}}=Mensa$
    \begin{itemize}
      \item<11->{\footnotesize  $\Dem{x_2}{\mMm_1,g_1\ekm{Mensa/x_1}}=Eckardt$ }
      \item<12->{\footnotesize  $\Dem{x_2}{\mMm_1,g_1\ekm{\ekm{Mensa/x_1}Webelhuth/x_2}}=Webelhuth$ }
      \item<13->{\footnotesize  $\Dem{x_2}{\mMm_1,g_1\ekm{\ekm{Mensa/x_1}Mensa/x_2}}=Mensa$ }
    \end{itemize}
\end{itemize}
}
}

\section{Problems with natural language}

\subsection{Restricted quantification}
\frame{\frametitle{Natural weirdness}
\begin{itemize}
  \item<1-> quantifying expressions in NL beyond $\forall$ and $\exists$
  \item<2-> some seem to work differently:
  \item<3-> \emph{\bl{All patients} adore Dr. Rick \underline{D}agless M.D.} \\ \bl{$(\forall x_1)Px_1\rightarrow{}Ax_1d$} (ok)
  \item<4-> but: \emph{\bl{Most patients} adore Dr. Rick \underline{D}agless M.D.}\\ \bl{$(MOST\ x_1)Px_1\rightarrow{}Ax_1d$} (\rot{wrong interpretation})
  \item<5-> domain should be the set of patients, not individuals
  \item<6-> For NL: \bl{Assume that the checking domain for Q is the set denoted by CN.}
\end{itemize}
}

\subsection{Variable binding and scope}
\frame{\frametitle{Scope ambiguities}
\begin{itemize}
  \item<1-> c-command condition on binding/scope fails in NL
  \item<2-> no PNF's in NL
  \item<3-> Q and common noun (CN) usually \bl{in-situ} (e.g., argument position) 
  \item<4-> \bl{ambiguities independent of Q position}
    \begin{itemize}
      \item<5-> \emph{Everybody loves somebody.} (\emph{ELS})
      \item<6-> $(\forall{}x_1)(\exists{}x_2)Lx_1x_2$
      \item<7-> $(\exists{}x_2)(\forall{}x_1)Lx_1x_2$
    \end{itemize}
  \item<8-> \bl{Q ambiguity cannot be structural} (e.g., $\exists$ will never c-command $\forall$)  
\end{itemize}
}

\subsection{Pre-spellout movement}
\frame{\frametitle{Cases of overt movement and traces}
\begin{itemize}
  \item<1-> \bl{wh} movement:
  \item<1-> \emph{\node{c-1}{What_i} will Agent Cooper solve \node{c-2}{t_i}?}
  \item<1->
  \item<2-> \bl{passive} movement:
  \item<2-> \emph{\node{c-3}{(Laura Palmer)_i} was killed \node{c-4}{t_i}.}
  \item<2->
  \item<3-> \bl{raising} verbs:
  \item<3-> \emph{\node{c-5}{(Laura Palmer)_i} seems \node{c-6}{t_i} to be dead.}
  \item<3->
\end{itemize} \anodecurve[b]{c-2}[b]{c-1}{0.8cm}
              \anodecurve[b]{c-4}[b]{c-3}{0.8cm}
              \anodecurve[b]{c-6}[b]{c-5}{0.8cm}
}

\subsection{LF movement}
\frame{\frametitle{Levels of representation}
\begin{itemize}
  \item<1-> construction of an independent representational level LF
  \item<2-> could use movement mechanism as used at surface level
  \item<3-> \bl{All quantifiers adjoin to the left periphery of S at LF.}
  \item<4-> \bl{LF is constructed by syntactic rules!}
\end{itemize}
}

\frame{\frametitle{Ambiguities at LF}
\begin{itemize}
  \item<1-> \emph{[_{S^{\prime\prime}} \node{d-1}{\bl{everybody_i}} [_{S^{\prime}} \node{d-3}{\bl{somebody_j}} [_S \node{d-2}{\bl{t_i}} loves \node{d-4}{\bl{t_j}} ]]]}
  \item<1->
  \item<2-> \emph{[_{S^{\prime\prime}} \node{d-5}{\bl{somebody_j}} [_{S^{\prime}} \node{d-7}{\bl{everybody_i}} [_S \node{d-8}{\bl{t_i}} loves \node{d-6}{\bl{t_j}} ]]]}
  \item<2->
\end{itemize} \anodecurve[b]{d-2}[b]{d-1}{0.6cm}
              \anodecurve[b]{d-4}[b]{d-3}{0.6cm}
              \anodecurve[b]{d-6}[b]{d-5}{0.6cm}
              \anodecurve[b]{d-8}[b]{d-7}{0.6cm}
}

\section{Quantification in English: F2}

\subsection{Movement rules}
\frame{\frametitle{The Q raising rule}

\begin{center}
  {\Large \bl{$[_S\ X\ NP\ Y\ ]\ \Rightarrow\ [_{S^{\prime}}\ NP_i\ [_S\ X\ t_i\ Y\ ]]$}}
\end{center}

\begin{itemize}
  \item<2-> specify a PS as input and output
  \item<3-> QR rule also introduces coindexing of traces
\end{itemize}
}

\subsection{Fragment F2}
\frame{\frametitle{Syntax}
\begin{itemize}
  \item<1-> copies all definitions from F1
  \item<2-> adds appropriate definitions of quantifying determiners etc.
    \begin{itemize}
      \item<3-> \bl{$Det \rightarrow every,\ some$}
      \item<4-> $NP \rightarrow Det N_{common-count}$
    \end{itemize}
  \item<5-> adds the \bl{QR rule}
  \item<6-> assume introduction of reasonable syntactic types/rules without specifying
  \item<7-> assume \bl{admissible (reasonable, possible) models \mM}
\end{itemize}
}

\frame{\frametitle{Semantics for QR output: \emph{every}}
\begin{center}
\bl{$\Dem{\ekm{\ekm{every\ \beta}_i\ S}}{\mMm,g}=1\ iff\ for\ all\ u\in U:$\\
    $if\ u\in\Dem{\beta}{\mMm,g}\ then\ \Dem{S}{\mMm,g\ekm{u/t_i}}$ \\
   }
\end{center}

A sentence containing the trace $t_i$ with an adjoined $NP_i$ (which consists of \emph{every} plus the common noun $\beta$) extend to 1 iff for each individual $u$ in the universe $U$ which is in the set referred to by the common noun $\beta$, $S$ denotes 1 with $u$ assigned to the pronominal trace $t_i$. $g$ is modified iteratively to check that.
}

\frame{\frametitle{Semantics for QR output: \emph{some, a}}
\begin{center}
\bl{$\Dem{\ekm{\ekm{a\ \beta}_i\ S}}{\mMm,g}=1\ iff\ for\ some\ u\in U:$\\
    $u\in\Dem{\beta}{\mMm,g}\ and\ \Dem{S}{\mMm,g\ekm{u/t_i}}$ \\
   }
\end{center}
(similar)
}

\end{document}
