\section{Von Prädikatenlogik zu natürlicher Sprache}

\begin{frame}
  {Zur Erinnerung}
  \onslide<+->
  \onslide<+->
  Semantik von Fragment F1\\
  \Halbzeile
  \begin{itemize}[<+->]
    \item Namen referieren auf \alert{spezifische Individuen}
    \item intransitive Verben referieren auf \alert{Mengen von Individuen}
    \item mehrstellige Verben referieren auf Mengen von \alert{Tupeln von Individuen}
    \item Sätze referieren auf \alert{Wahrheitswerte}!
      \Halbzeile
    \item F2 | Integration von Erkenntnissen aus Prädikatenlogik
  \end{itemize}
  \onslide<+->
  \Zeile
  \centering 
  Alles Wesentliche dieser Sitzung in \citet[Kapitel~3]{ChierchiaMcconnellginet2000}
\end{frame}

\begin{frame}
  {Das Problem mit Pronomina}
  \onslide<+->
  \onslide<+->
  Wie situationsabhängige Namen\\
  \Halbzeile
  \begin{itemize}[<+->]
    \item[ ] \textit{\alert{This} is red.}
    \item Pronomen \alert{\textit{this}} | syntaktisch eine NP
    \item \ldots\ und referiert auf \alert{ein spezifisches Objekt} (wie Namen)\\
      \grau{\footnotesize keine Quantifikation bzw. Mengenreferenz}
      \Halbzeile
    \item Aber \orongsch{nur in gegebener Situation interpretierbar}\\
      \grau{\footnotesize Deixis, im Text auch Anaphorik}
    \item Kein Äquivalent in klassischer Logik
  \end{itemize}
\end{frame}

\begin{frame}
  {Pronomina und Variablen}
  \onslide<+->
  \onslide<+->
  Ähnlichkeit von Variablen und Pronominalausdrücken\\
  \Halbzeile
  \begin{itemize}[<+->]
    \item Rumpf einer quantifizierten Wff | Wff $P(x)$ aus Wff $(\forall x)Px$
    \item Ungebundenes $x$ in $P(x)$ \alert{ähnlich wie Pronominalbedeutung}\\
      \grau{\footnotesize Externe Interpretationsvorschrift erforderlich}
    \Halbzeile
  \item Quantoren | Auswertungsalgorithmus\\
    \grau{\footnotesize Für alle möglichen belegungen von $x$, $P(x)$}
  \item Pronomina | Kontextuelle Auswertung\\
    \grau{\footnotesize Belegung für $x$ im gegebenen Kontext}
  \end{itemize}
\end{frame}

\begin{frame}
  {Prädikatenlogik | Syntax}
  \onslide<+->
  \onslide<+->
  Als Vorüberlegung | Prädikatenlogik als \alert{Phrasenstrukturgrammatik}\\
  \Halbzeile
  \begin{itemize}[<+->]
    \item[ ] $a\ \rightarrow\ const, var$ \grau{| Individuenausdrücke}
    \item[ ] $conn\ \rightarrow\ \wedge,\vee,\rightarrow,\leftrightarrow$ \grau{| Funktoren}
    \item[ ] $neg\ \rightarrow\ \neg$ \grau{| Negation}
    \item[ ] $Q\ \rightarrow\ \exists,\forall$ \grau{| nur zwei Quantoren}
    \item[ ] $pred^1\ \rightarrow\ P, Q$ \grau{| einstellige Prädikate}
    \item[ ] $pred^2\ \rightarrow\ R$ \grau{| zweistellige Prädikate}
    \item[ ] $pred^3\ \rightarrow\ S$ \grau{| dreistellige Prädikate}
    \item[ ] $const\ \rightarrow\ b, c$ \grau{| nur zwei Individenkonstanten}
    \item[ ] $var\ \rightarrow\ x_1,x_2,\cdots x_n$ \grau{| beliebig viele Variablen}
      \Halbzeile
    \item \grau{Die Formalisierung ist äquivalent zur mengenbasierten von letzter Woche!}
  \end{itemize}
\end{frame}

\begin{frame}
  {Prädikatenlogik | PS-Regeln}
  \onslide<+->
  \onslide<+->
  Wir nehmen eine \alert{Prädikatsnotation ohne Klammern} | $Px$ statt $P(x)$ usw.\\
  \Halbzeile
  \begin{itemize}[<+->]
    \item $wff\rightarrow pred^1\ a_1\ldots\ a_n$ \grau{| n-stellige Prädikate und ihre Argumente}
    \item $wff\rightarrow neg\ wff$ \grau{| Applikation von Negation auf Wffs}
    \item $wff\rightarrow wff\ conn\ wff$ \grau{| Applikation von anderen Funktoren auf Wffs}
    \item $wff\rightarrow Q\ var\ wff$ \grau{| Quantifikation}
  \end{itemize}
\end{frame}

\begin{frame}
  {Eine Wff ohne Quantoren}
  \onslide<+->
  \onslide<+->
  Zum Beispiel: \textit{Ben ($b$) paddelt ($P$) und ($\wedge$) Ben rudert ($R$) nicht ($\neg$) mit Chris ($c$).}\\
  In PL: \alert{$Pb\wedge\neg Rbc$}\\
  \onslide<+->
  \Zeile
  \centering
  \scalebox{0.8}{\begin{forest}
    [$wff$, calign=child, calign child=2
      [$wff$
        [$pred^1$
          [$P$]
        ]
        [$a$
          [$const$
            [$b$]
          ]
        ]
      ]
      [$conn$
        [$\wedge$]
      ]
      [$wff$
        [$\neg$]
        [$wff$, calign=child, calign child=2
          [$pred^2$
            [$R$]
          ]
          [$a$
            [$const$
              [$b$]
            ]
          ]
          [$a$
            [$const$
              [$c$]
            ]
          ]
        ]
      ]
    ]
  \end{forest}}
\end{frame}

\begin{frame}
  {Eine Wff mit Quantoren}
  \onslide<+->
  \onslide<+->
  Zum Beispiel: \textit{Als Paddler hat man immer jemanden, mit dem man nicht rudert.}\\
  In PL: \alert{$\forall x_1[Px_1\rightarrow\exists x_2\neg Px_1x_2]$}\\
  \onslide<+->
  \Halbzeile
  \centering
  \scalebox{0.7}{\begin{forest}
    [$wff$
      [$Q\ var$
        [$\forall x_1$]
      ]
      [$wff$, calign=child, calign child=2
        [$wff$
          [$pred^1$
            [$P$]
          ]
          [$a$
            [$var$
              [$x_1$]
            ]
          ]
        ]
        [$conn$
          [$\rightarrow$]
        ]
        [$wff$
          [$Q\ var$
            [$\exists x_2$]
          ]
          [$wff$
            [$\neg$]
            [$wff$, calign=child, calign child=2
              [$pred^2$
                [$R$]
              ]
              [$a$
                [$const$
                  [$x_1$]
                ]
              ]
              [$a$
                [$const$
                  [$x_2$]
                ]
              ]
            ]
          ]
        ]
      ]
    ]
  \end{forest}}
\end{frame}

\begin{frame}
  {Skopus und c-Kommando}
  Skopus in konfigurationaler Logik-Syntax: \alert{c-Kommando}\\
  Variablen als \alert{gebunden vom nächsten c-kommandierenden koindizierten Quantor}\\
  \Halbzeile
  \centering
  \scalebox{0.6}{\begin{forest}
    [$wff$
      [$Q\ var$
        [$\forall x_1$]
      ]
      [$wff$, calign=child, calign child=2, gruentree
        [$wff$
          [$pred^1$
            [$P$]
          ]
          [$a$
            [$var$
              [$x_1$]
            ]
          ]
        ]
        [$conn$
          [$\rightarrow$]
        ]
        [$wff$
          [$Q\ var$
            [$\exists x_2$]
          ]
          [$wff$, bluetree
            [$\neg$]
            [$wff$, calign=child, calign child=2
              [$pred^2$
                [$R$]
              ]
              [$a$
                [$const$
                  [$x_1$]
                ]
              ]
              [$a$
                [$const$
                  [$x_2$]
                ]
              ]
            ]
          ]
        ]
      ]
    ]
  \end{forest}}\\
  \visible<4->{\footnotesize\alert{Skopus\slash c-Kommando-Domäne von $\exists x_2$}}\visible<4->{ | \footnotesize\gruen{Skopus\slash c-Kommando-Domäne von $\forall x_1$} (zgl.\ \alert{derer von $\exists x_2$})}\\
\end{frame}

\section{Modelltheorie}

\begin{frame}
  {Semantik für PL in Vorbereitung auf natürliche Sprache}
  \onslide<+->
  \onslide<+->
  Ziel (zur Erinnerung) | T-Sätze der Form \textit{S aus L ist wahr in v gdw \ldots}\\
  \Halbzeile
  \begin{itemize}[<+->]
    \item \alert{Modell $\Model$} | zugängliches Diskursuniversum (bzw.\ dessen Beschreibung)
    \item \alert{Menge $D_n$} | Zugängliche Individuen (\textit{domain}) in $\Model_n$
    \item \alert{Funktion $V_n$} | Valuation -- Zuweisung von
      \begin{itemize}[<+->]
        \item Namen zu Individuen in $\Model_n$
        \item Predikaten zu Tupeln von Individuen
      \end{itemize}
    \item \alert{$\Model_n=\tuple{D_n,V_n}$}
      \Halbzeile
    \item \alert{Funktion $g_n$} | Zuweisung von Variablen zu Individuen in $\Model_n$ 
      \Halbzeile
    \item Allgemeine Evaluation in $\Model_n$ | $\dem{\alpha}^{\Model_n,g_n}$\\
      \grau{Lies: \textit{Die Extension von Ausdruck $\alpha$ relativ zu $\Model_n$ und $g_n$}}
  \end{itemize}
\end{frame}

\begin{frame}
  {Unterschied zwischen $V_n$ und $g_n$}
  \onslide<+->
  \onslide<+->
  Feste und variable Denotation\\
  \Halbzeile
  \begin{itemize}[<+->]
    \item $V_n$ evaluiert \alert{statisch} im Modell.\\
      \grau{\footnotesize Wenn das Modell einmal feststeht, evaluiert $V_n$ jede Konstante stets gleich.}
      \Halbzeile
    \item Variablen (gebunden durch Quantoren) werden \alert{volatil interpretiert}.\\
    \item \alert{Iteration} durch Universum $D_n$ durch $g_n$
    \item Eine Modifikation der Belegung pro Iteration
      \begin{itemize}[<+->]
        \item Modifizierte \textit{assignment function} \alert{$g_n[d_i/x_m]$}\\
          Lies: \textit{relativ zu $g_n$, wobei die Referenz von Variable $x_m$ auf Individuum $d_i$ gesetzt wird}
      \end{itemize}
  \end{itemize}
\end{frame}

\begin{frame}
  {Evaluation von Variablen}
  \onslide<+->
  \onslide<+->\scriptsize
  $\alert{D_1}=\{Herr\ Webelhuth, Frau\ Klenk, Turm-Mensa\}$ \grau{| Individuen in $\Model_1$}\\
  \onslide<+->
  $\alert{V_1(P)}=\{Herr\ Webelhuth, Frau\ Klenk, Turm-Mensa\}$ \grau{| Prädikat $P$ (\zB \textit{ist ein physikalisches Objekt}) in $\Model_1$}\\
  \onslide<+->
  Evaluiere \alert{$\dem{\forall x_1Px_1}^{\Model_1,g_1}$}\visible<8->{$=\gruen{1}$ weil keiner Belegung $\dem{Px_1}^{\Model_1,g_1}=\orongsch{0}$}\\
  \Halbzeile
  \begin{itemize}[<+->]
    \item Initiale Belegung \alert{$\dem{x_1}^{\Model_1,g_1}=Herr\ Webelhuth$}\\
      \scalebox{0.7}{$g_1 = \left[\begin{array}{l}
          \gruen<5>{x_1 \rightarrow Herr\ Webelhuth}\\
            x_2 \rightarrow Herr\ Webelhuth\\
            x_3 \rightarrow Herr\ Webelhuth
        \end{array}\right]$}\\
        \Viertelzeile
          $\dem{Px_1}^{\Model_1,g_1}=\gruen{1}$
          \Halbzeile
        \item \alert{$\dem{x_1}^{\Model_1,g_1[Klenk/x_1]}=Frau\ Klenk$}\\
         \scalebox{0.7}{$g_1 = \left[\begin{array}{l}
             \gruen<6>{x_1 \rightarrow Frau\ Klenk}\\
            x_2 \rightarrow Herr\ Webelhuth\\
            x_3 \rightarrow Herr\ Webelhuth
        \end{array}\right]$}\\
        \Viertelzeile
          $\dem{Px_1}^{\mMm_1,g_1\ekm{Klenk/x_1}}=\gruen{1}$
          \Halbzeile
        \item \alert{$\dem{x_1}^{\Model_1,g_1[Turm-Mensa/X_1]}=Turm-Mensa$}\\
         \scalebox{0.7}{$g_1 = \left[\begin{array}{l}
             \gruen<7>{x_1 \rightarrow Turm-Mensa}\\
            x_2 \rightarrow Herr\ Webelhuth\\
            x_3 \rightarrow Herr\ Webelhuth
        \end{array}\right]$}\\
        \Viertelzeile
          $\dem{Px_1}^{\mMm_1,g_1\ekm{Mensa/x_1}}=\gruen{1}$
  \end{itemize}
\end{frame}

\begin{frame}
  {Evaluation mit zwei Variablen}
  \onslide<+->
  \onslide<+->\scriptsize
  $\alert{D_1}=\{Herr\ Webelhuth, Frau\ Klenk, Turm-Mensa\}$ \grau{| Individuen in $\Model_1$}\\
  \onslide<+->
  $\alert{V_1(Q)}=\{\tuple{Webelhuth,Klenk},\tuple{Webelhuth,Mensa},\tuple{Klenk,Webelhuth}\}$ \grau{| Prädikat $Q$ (\zB \textit{x besucht y}) in $\Model_1$}\\
  \onslide<+->
  Evaluiere \alert{$\dem{\forall x_1\exists x_2 Qx_1x_2}^{\Model_1,g_1}\visible<18->{=\orongsch{0}}$} \visible<18->{weil nicht für jede Belegung von $x_1$ mindestens einmal \gruen{1}}\\
  \onslide<+->
  \Zeile
  \begin{minipage}{0.5\textwidth}\begin{itemize}[<+->]
        \item Initiale Belegung $\dem{x_1}^{\Model_1,g_1}=Frau\ Klenk$
          \begin{itemize}[<+->]\scriptsize
            \item $\dem{Qx_1x_2}^{\mMm_1,g_1}=\orongsch{0}$
            \item $\dem{Qx_1x_2}^{\mMm_1,g_1[\gruen<8>{Klenk/x_2}]}=\orongsch{0}$
            \item $\dem{Qx_1x_2}^{\mMm_1,g_1[\gruen<9>{Webelhuth/x_2}]}=\gruen{1}$
          \end{itemize}
        \item $\dem{x_1}^{\Model_1,g_1[Turm-Mensa/x_1]}=Turm-Mensa$
          \begin{itemize}[<+->]\scriptsize
            \item $\dem{Qx_1x_2}^{\mMm_1,g_1[\tuerkis<11-13>{Turm-Mensa/x_1}]}=\orongsch{0}$
            \item $\dem{Qx_1x_2}^{\mMm_1,g_1[\tuerkis<11-13>{Turm-Mensa/x_1},\gruen<12>{Klenk/x_2}]}=\orongsch{0}$
            \item $\dem{Qx_1x_2}^{\mMm_1,g_1[\tuerkis<11-13>{Turm-Mensa/x_1},\gruen<13>{Webelhuth/x_2}]}=\orongsch{0}$ \rot{Abbruch!}
          \end{itemize}
        \item $\dem{x_1}^{\Model_1,g_1[Webelhuth/x_1]}=Herr\ Webelhuth$
          \begin{itemize}[<+->]\scriptsize
            \item $\dem{Qx_1x_2}^{\mMm_1,g_1[\tuerkis<15-17>{Webelhuth/x_1}]}=\gruen{1}$
            \item $\dem{Qx_1x_2}^{\mMm_1,g_1[\tuerkis<15-17>{Webelhuth/x_1},\gruen<16>{Klenk/x_2}]}=\gruen{1}$
            \item $\dem{Qx_1x_2}^{\mMm_1,g_1[\tuerkis<15-17>{Webelhuth/x_1},\gruen<17>{Webelhuth/x_2}]}=\orongsch{0}$
          \end{itemize}
    \end{itemize}\end{minipage}%
    \begin{minipage}{0.5\textwidth}%
    \centering 
    \scalebox{1.2}{$g_1 = \left[\begin{array}{l}
        x_1 \rightarrow \tuerkis<11-13,15-17>{%
          \only<5-10,14,18->{Frau\ Klenk}%
          \only<11-13>{Turm-Mensa}%
          \only<15-17>{Herr\ Webelhuth}%
        }\\
        x_2 \rightarrow \gruen<8-9,12-13,16-17>{%
          \only<5-7,10-11,14-15,18->{Turm-Mensa}%
          \only<8,12,16>{Frau\ Klenk}%
          \only<9,13,17>{Herr\ Webelhuth}
        }\\
      x_3 \rightarrow Herr\ Webelhuth\\
    \end{array}\right]$}\\
  \end{minipage}
\end{frame}

\end{document}

\section{Problems with natural language}

\frame{\frametitle{Natural weirdness}
\begin{itemize}
  \item<1-> quantifying expressions in NL beyond $\forall$ and $\exists$
  \item<2-> some seem to work differently:
  \item<3-> \emph{\bl{All patients} adore Dr. Rick \underline{D}agless M.D.} \\ \bl{$(\forall x_1)Px_1\rightarrow{}Ax_1d$} (ok)
  \item<4-> but: \emph{\bl{Most patients} adore Dr. Rick \underline{D}agless M.D.}\\ \bl{$(MOST\ x_1)Px_1\rightarrow{}Ax_1d$} (\rot{wrong interpretation})
  \item<5-> domain should be the set of patients, not individuals
  \item<6-> For NL: \bl{Assume that the checking domain for Q is the set denoted by CN.}
\end{itemize}
}

\frame{\frametitle{Scope ambiguities}
\begin{itemize}
  \item<1-> c-command condition on binding/scope fails in NL
  \item<2-> no PNF's in NL
  \item<3-> Q and common noun (CN) usually \bl{in-situ} (e.g., argument position) 
  \item<4-> \bl{ambiguities independent of Q position}
    \begin{itemize}
      \item<5-> \emph{Everybody loves somebody.} (\emph{ELS})
      \item<6-> $(\forall{}x_1)(\exists{}x_2)Lx_1x_2$
      \item<7-> $(\exists{}x_2)(\forall{}x_1)Lx_1x_2$
    \end{itemize}
  \item<8-> \bl{Q ambiguity cannot be structural} (e.g., $\exists$ will never c-command $\forall$)  
\end{itemize}
}

\frame{\frametitle{Cases of overt movement and traces}
\begin{itemize}
  \item<1-> \bl{wh} movement:
  \item<1-> \emph{\node{c-1}{What\Sub{i}} will Agent Cooper solve \node{c-2}{t\Sub{i}}?}
  \item<1->
  \item<2-> \bl{passive} movement:
  \item<2-> \emph{\node{c-3}{(Laura Palmer)\Sub{i}} was killed \node{c-4}{t\Sub{i}}.}
  \item<2->
  \item<3-> \bl{raising} verbs:
  \item<3-> \emph{\node{c-5}{(Laura Palmer)\Sub{i}} seems \node{c-6}{t\Sub{i}} to be dead.}
  \item<3->
\end{itemize} \anodecurve[b]{c-2}[b]{c-1}{0.8cm}
              \anodecurve[b]{c-4}[b]{c-3}{0.8cm}
              \anodecurve[b]{c-6}[b]{c-5}{0.8cm}
}

\frame{\frametitle{Levels of representation}
\begin{itemize}
  \item<1-> construction of an independent representational level LF
  \item<2-> could use movement mechanism as used at surface level
  \item<3-> \bl{All quantifiers adjoin to the left periphery of S at LF.}
  \item<4-> \bl{LF is constructed by syntactic rules!}
\end{itemize}
}

\frame{\frametitle{Ambiguities at LF}
\begin{itemize}
  \item<1-> \emph{[\Sub{S\Up{$\prime\prime$}} \node{d-1}{\bl{everybody\Sub{i}}} [\Sub{S\Up{$\prime$}} \node{d-3}{\bl{somebody\Sub{j}}} [\Sub{S} \node{d-2}{\bl{t\Sub{i}}} loves \node{d-4}{\bl{t\Sub{j}}} ]]]}
  \item<1->
  \item<2-> \emph{[\Sub{S\Up{$\prime\prime$}} \node{d-5}{\bl{somebody\Sub{j}}} [\Sub{S\Up{$\prime$}} \node{d-7}{\bl{everybody\Sub{i}}} [\Sub{S} \node{d-8}{\bl{t\Sub{i}}} loves \node{d-6}{\bl{t\Sub{j}}} ]]]}
  \item<2->
\end{itemize} \anodecurve[b]{d-2}[b]{d-1}{0.6cm}
              \anodecurve[b]{d-4}[b]{d-3}{0.6cm}
              \anodecurve[b]{d-6}[b]{d-5}{0.6cm}
              \anodecurve[b]{d-8}[b]{d-7}{0.6cm}
}

\section{Quantification in English: F2}

\frame{\frametitle{The Q raising rule}

\begin{center}
  {\Large \bl{$[_S\ X\ NP\ Y\ ]\ \Rightarrow\ [_{S^{\prime}}\ NP_i\ [_S\ X\ t_i\ Y\ ]]$}}
\end{center}

\begin{itemize}
  \item<2-> specify a PS as input and output
  \item<3-> QR rule also introduces coindexing of traces
\end{itemize}
}

\frame{\frametitle{Syntax}
\begin{itemize}
  \item<1-> copies all definitions from F1
  \item<2-> adds appropriate definitions of quantifying determiners etc.
    \begin{itemize}
      \item<3-> \bl{$Det \rightarrow every,\ some$}
      \item<4-> $NP \rightarrow Det N_{common-count}$
    \end{itemize}
  \item<5-> adds the \bl{QR rule}
  \item<6-> assume introduction of reasonable syntactic types/rules without specifying
  \item<7-> assume \bl{admissible (reasonable, possible) models \mM}
\end{itemize}
}

\frame{\frametitle{Semantics for QR output: \emph{every}}
\begin{center}
\bl{$\Dem{\ekm{\ekm{every\ \beta}_i\ S}}{\mMm,g}=1\ iff\ for\ all\ u\in U:$\\
    $if\ u\in\Dem{\beta}{\mMm,g}\ then\ \Dem{S}{\mMm,g\ekm{u/t_i}}$ \\
   }
\end{center}

A sentence containing the trace $t_i$ with an adjoined $NP_i$ (which consists of \emph{every} plus the common noun $\beta$) extend to 1 iff for each individual $u$ in the universe $U$ which is in the set referred to by the common noun $\beta$, $S$ denotes 1 with $u$ assigned to the pronominal trace $t_i$. $g$ is modified iteratively to check that.
}

\frame{\frametitle{Semantics for QR output: \emph{some, a}}
\begin{center}
\bl{$\Dem{\ekm{\ekm{a\ \beta}_i\ S}}{\mMm,g}=1\ iff\ for\ some\ u\in U:$\\
    $u\in\Dem{\beta}{\mMm,g}\ and\ \Dem{S}{\mMm,g\ekm{u/t_i}}$ \\
   }
\end{center}
(similar)
}

