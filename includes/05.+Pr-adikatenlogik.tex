
\section{Why predicate calculus?}

\frame{\frametitle{Weak compositionality in SL}
\begin{itemize}
  \item<1-> properties/relations vs. individuals
  \item<2-> \emph{Martin is an \underline{e}xpert on inversion and Martin is a good \underline{c}limber.}
  \item<3-> \ldots becomes $E\wedge C$
  \item<4-> compositionality resticted to level of connected propositional atoms
\end{itemize}
}

\frame{\frametitle{Some desirable deductions}
\begin{itemize}
  \item<1-> important generalizations about all and some individuals (which have property P)
  \item<2-> \emph{`all P $\rightarrow$ some P'}
  \item<3-> \emph{`Martin P $\rightarrow$ some P'}
\end{itemize}
}

\section{The construction of PC}

\subsection{Atoms and syntax}

\frame{\frametitle{Atoms of PC}
\begin{itemize}
  \item<1-> individual \bl{variables}: $x,y,z,x_1,x_2\ldots$
  \item<2-> individual \bl{constants}: $a,b,c,\ldots$
  \item<3-> variables and constants: \bl{terms}
  \item<4-> \bl{predicate symbols} (taking individual symbols or tuples of them): $A,B,C,\ldots$
  \item<5-> \bl{quantifiers}: existential $\exists$ (or $\bigvee$) and universal $\forall$ (or $\bigwedge$)
  \item<6-> plus the connectives of SL
\end{itemize}
}

\frame{\frametitle{Some syntax}
\begin{itemize}
  \item<1-> for an $n$-ary predicate P and terms $t_1\ldots t_n$,\\ \bl{$P(t_1\ldots t_n)$} or $Pt_1\ldots t_n$ is a wff.
  \item<2-> possible prefix, function (bracket) and infix notation:\\ $Pxy$, $P(x,y)$, $xPy$
  \item<3-> syntax for connectives from SL
  \item<4-> for any wff $\phi$ and any variable $x$, \bl{$(\exists x)\phi$} and \bl{$(\forall x)\phi$} are wff's
\end{itemize}
}

\subsection{Semantics}

\frame{\frametitle{Semantic for individual constants}
\begin{itemize}
  \item<1-> \bl{denote} individuals
  \item<2-> a \bl{model $\mathcal M$} contains a set of individuals $D$
  \item<3-> the \bl{valuation function $V$} (or F): from constants to individuals in D
  \item<4-> for some ${\mathcal M}_1$: $D=\{Martin,Kilroy,Scully\}$
  \item<5-> $V_{{\mathcal M}_1}(m)=Martin$
  \item<6-> $V_{{\mathcal M}_1}(k)=Kilroy$, $V_{{\mathcal M}_1}(s)=Scully$
\end{itemize}
}

\frame{\frametitle{Semantics for predicate symbols}
\begin{itemize}
  \item<1-> \bl{denote relations (sets of n-tuples)}
  \item<2-> $\dem{P}^{{\mathcal M}_1}=\{Martin,Kilroy\}$ or $V_{{\mathcal M}_1}(P)=\{Martin,Kilroy\}$
  \item<3-> $V_{{\mathcal M}_1}(Q)=\{\langle Martin,Kilroy\rangle, \langle Martin,Scully\rangle, \langle Kilroy,Kilroy\rangle,$\\ $\langle Scully,Scully\rangle\}$
  \item<4-> s.t. $\llbracket P(m)\rrbracket^{\mathcal{M}_1}=\llbracket P\rrbracket^{\mathcal{M}_1}(\llbracket m)\rrbracket^{\mathcal{M}_1})=1$ iff $\llbracket m\rrbracket^{\mathcal{M}_1}\in\llbracket P\rrbracket^{\mathcal{M}_1}$
\end{itemize}
}

\frame{\frametitle{Semantics for connectives and quantifiers}
\begin{itemize}
  \item<1-> \bl{connectives}: `apply to' formulas (semantically truth-valued), semantics as in SL
  \item<2-> \bl{$(\forall x)\phi$} = 1 iff $\phi$ is true for every $d\in D$\\ assigned to every occurence of $x$ in $\phi$
  \item<3-> \bl{$(\exists x)\phi$} = 1 iff $\phi$ is true for at least one $d\in D$\\ assigned to every occurence of $x$ in $\phi$
  \item<4-> algorithmic instruction to check wff's containing Q's
  \item<5-> check outside-in (unambiguous scoping)
\end{itemize}
}

\frame{\frametitle{Dependencies}
\begin{itemize}
  \item<1-> universal quantifiers can be swapped:\\ $(\forall x)(\forall y)\phi \Leftrightarrow (\forall y)(\forall x)\phi$
  \item<2-> same for existential quantifiers:\\ $(\exists x)(\exists y)\phi \Leftrightarrow (\exists y)(\exists x)\phi$
  \item<3-> whereas: \bl{$(\exists x)(\forall y)\phi \Rightarrow (\forall y)(\exists x)\phi$}
  \item<4-> example in $\mathcal{M}_1$:
    \begin{itemize}
      \item<4-> $\dem{\underline{(\forall x)\underline{(\exists y)Qxy}}}^{\mathcal{M}_1}=1$
      \item<5-> but: $\dem{\underline{(\exists y)\underline{(\forall x)Qxy}}}^{\mathcal{M}_1}=0$
      \item<6-> direct consequence of algorithmic definition
      \item<7-> if $\exists\forall$ is true, $\forall\exists$ follows
    \end{itemize}
\end{itemize}
}

\frame{\frametitle{Hints on quantifiers}
\begin{itemize}
  \item<1-> domain of quantifiers: D (universe of discourse)
  \item<2-> $\forall x$ checks for truth of some predication for all individuals
  \item<3-> $\exists x(Px\wedge\neg Px)$ is a contradiction
  \item<4-> $\forall x(Wx\wedge \neg Wx)$ is a contradiciton,\\ $\forall x$ `checks' for an empty set by def.
  \item<5-> standard form of NL quantification:\\ \bl{$\forall x(Wx\rightarrow Bx)$} `All women are beautiful.'
  \item<6-> standard form of NL existential quantification:\\ \bl{$\exists x(Wx\wedge Bx)$} `Some woman is beautiful.'
\end{itemize}
}

\subsection{More rules}

\frame{\frametitle{Functor/quantifier practice}
\begin{itemize}
  \item<1-> by def., functors take formulas, not terms:
    \begin{itemize}
      \item<2-> $\neg Wm$ `Mary doesn't weep.'
      \item<3-> $(\exists x)(Gx \wedge Wx)$ `Some girl weeps.'
      \item<4-> \textcolor{red}{$^{\ast}W\neg x$}
      \item<5-> \textcolor{red}{$^{\ast}(\exists\neg x)(Gx)$}
    \end{itemize}
  \item<6-> quantifiers take variables, not constants:
    \begin{itemize}
      \item<7-> $(\forall x)(Ox\rightarrow Wx)$ `All ozelots are wildcats.'
      \item<8-> \textcolor{red}{$^{\ast}(\forall o)(Wo)$}
    \end{itemize}
  \item<9-> $\neg$ negates the wff, not the q:\\ \textcolor{red}{$^{\ast}(\neg\forall x)Px$} but $\neg(\forall x)Px$
\end{itemize}
}

\frame{\frametitle{Scope}
\begin{itemize}
  \item<1-> quantifiers \bl{bind} variables
  \item<2-> free variables (constants) are unbound
  \item<3-> \bl{no double binding} \textcolor{red}{$^{\ast}(\forall x\exists x)Px$}
  \item<4-> \bl{Q scope}: only the first wff to its right:
    \begin{itemize}
      \item<5-> $\underline{(\forall x)Px}\vee Qx$
      \item<6-> $\underline{(\forall x)(Px\vee Qx)} = \underline{(\forall x)Px} \vee \underline{(\forall x)Qx}$
      \item<7-> $\underline{(\exists x)Px} \rightarrow \underline{(\forall y)(Qy\wedge Ry)}$
      \item<8-> $\underline{(\exists x)Px}\wedge Qx$ (second $x$ is a unbound)
    \end{itemize}
  \item<9->\bl{no double-naming}
\end{itemize}
}

\section{Laws of PC}

\subsection{Negation and distribution}

\frame{\frametitle{Universal $\vee$ and $\wedge$}
\begin{itemize}
  \item<1-> $\exists$ and $\forall$ `or' and `and' over the universe of discourse (hence: $\bigvee$ and $\bigwedge$)
  \item<2-> \bl{$(\forall x)Px$ $\Leftrightarrow$ $Px_1\wedge Px_2\wedge \ldots\wedge Px_n$} for all $x_n$ assigned to $d_n\in D$
  \item<3-> \bl{$(\exists x)Px$ $\Leftrightarrow$ $Px_1\vee Px_2\vee\ldots\vee Px_n$} for all $x_n$ assigned to $d_n\in D$
  \item<4-> hence: \bl{$\neg(\forall x)Px$ $\Leftrightarrow$ $\neg(Px_1\wedge Px_2\wedge \ldots\wedge Px_n)$}
  \item<5-> with DeM: \bl{$\overline{Px_1\wedge Px_2\wedge \ldots\wedge Px_n}$}
  \item<6-> $\Leftrightarrow$ \bl{$\overline{Px_1}\vee \overline{Px_2}\vee \ldots\vee \overline{Px_n}$}
  \item<7-> $\Leftrightarrow$ \bl{$(\exists x)\neg Px$}
\end{itemize}
}

\frame{\frametitle{Quantifier negation (QN)}
\begin{itemize}
  \item<1-> $\neg(\forall x)Px \Leftrightarrow (\exists x)\neg Px$
  \item<2-> $\neg(\exists x)Px \Leftrightarrow (\forall x)\neg Px$
  \item<3-> $\neg(\forall x)\neg Px \Leftrightarrow (\exists x)Px$
  \item<4-> $\neg(\exists x)\neg Px \Leftrightarrow (\forall x)Px$
\end{itemize}
}

\frame{\frametitle{The distribution laws}
\begin{itemize}
  \item<1-> the conjunction of universally quantified formulas:\\
    \bl{$\underline{(\forall x)(Px\wedge Qx)} \Leftrightarrow \underline{(\forall x)Px}\wedge\underline{(\forall x)Qx}$}
  \item<2-> the disjunction of existentially quantified formulas:\\
    \bl{$\underline{(\exists x)(Px\vee Qx)} \Leftrightarrow \underline{(\exists x)Px}\vee\underline{(\exists x)Qx}$}
   \item<3-> not v.v.: $(\forall x)Px\vee(\forall x)Qx$ $\bl{\Rightarrow}$ $(\forall x)(Px\vee Qx)$ 
   \item<4-> why?
\end{itemize}
}

\subsection{Movement}

\frame{\frametitle{Quantifier movement (QM)}
\begin{itemize}
  \item<1-> desirable format: \bl{prefix + matrix}
  \item<2-> Movement Laws for antecedents of conditionals:\\ \bl{$(\exists x)Px \rightarrow \phi \Leftrightarrow (\forall x)(Px \rightarrow \phi)$}\\
  \bl{$(\forall x)Px \rightarrow \phi \Leftrightarrow (\exists x)(Px \rightarrow \phi)$}
  \item<3-> Movement Laws for Q's in disjunction, conjunction, and the
	  consequent of conditionals: \bl{Just move them to the prefix!}
  \item<4-> condition: \bl{$x$ must not be free in $\phi$.}
  \item<5-> i.e.: Watch your variables!
\end{itemize}
}

\subsection{Some in-class practice}

\frame{\frametitle{Let's formalize:}
\begin{itemize}
  \item<1-> \emph{Paul \underline{K}alkbrenner is a \underline{m}usician and \underline{s}igned on \underline{b}pitchcontrol.}
  \item<2->	\emph{Herr \underline{S}. \underline{i}nstalled \underline{R}edHat and not every \underline{L}inux distribution is \underline{e}asy to install.	}
  \item<3->	\emph{All \underline{t}alkmasters are \underline{h}uman and Harald
	\underline{S}chmidt is a talkmaster.}
  \item<4->	\emph{Some \underline{t}alkmasters are not \underline{m}usicians.}
  \item<5->	\emph{Heiko \underline{L}aux \underline{o}wns \underline{K}anzleramt records and
	does not \underline{l}ike any \underline{G}igolo artist.}
  \item<6-> \emph{Some \underline{h}umans are neither \underline{t}alkmasters nor
	do they \underline{o}wn \underline{K}anzleramt records.}
 \end{itemize}
}

\section{Natural deduction in PC}

\subsection{Quantifier elimination}

\frame{\frametitle{Universal instantiation ($-\forall$) and generalization ($+\forall$)}
\begin{itemize}
  \item<1-> \bl{$(\forall x)Px\rightarrow Pa$}
  \item<2-> always applies
  \item<4-> can use any variable/constant
  \item<5-> \bl{$Pa\rightarrow (\forall x)Px$}
  \item<6-> iff $Pa$ was instantiated by $-\forall$
\end{itemize}
}

\frame{\frametitle{Existential generalization ($+\exists$) and instantiation ($-\exists$)}
\begin{itemize}
  \item<1-> \bl{$Pa \rightarrow (\exists x)Px$} for any individual constant a
  \item<2-> always applies
  \item<3-> \bl{$(\exists x)Px \rightarrow Pa$} for some indiv. const.
  \item<4-> always applies (there is a minimal individual for $\exists x$)
  \item<5-> for some $(\exists x)Px$ and $(\exists x)Qx$ the minimal individual might be different
  \item<6-> hence: \bl{When you apply EI, always use fresh constants!}
\end{itemize}
}

\subsection{An example}

\frame{\frametitle{One sample task}
\begin{itemize}
  \item<1-> (1) Herr \underline{K}eydana \underline{d}rives a Golf. (2) Anything that drives a golf is \underline{h}uman or a complex \underline{p}rogram simulating an artificial neural net. (3) There are no programs s.a.a.n.n. which are complex enough to drive a Golf.
  \item<2-> Formalize and prove: \bl{At least one human exists.}
  \item<3-> (1) $Dk$
  \item<4-> (2) $(\forall x)(Dx\rightarrow Hx \vee Px)$
  \item<5-> (3) $\neg(\exists x)(Px\wedge Dx)$
  \item<6-> \bl{$(\exists x)Hx$}
\end{itemize}
}

\frame{\frametitle{The proof}
\begin{tabular}{lll}
 (1) & $Dk$ &  \\
 (2) & $(\forall x)(Dx\rightarrow Hx \vee Px)$ & \\
 (3) & $\neg(\exists x)(Px\wedge Dx)$ & \\
 \hline
 (4) & $(\forall x)\neg(Px\wedge Dx)$ & 3,QN\\
 (5) & $(\forall x)(\neg Px\vee\neg Dx)$ & 4,DeM\\
 (6) & $(\forall x)(Dx\rightarrow\neg Px)$ & 5,Comm,Impl \\
 (7) & $Dk \rightarrow \neg Pk$ & 6,$-\forall$(1) \\
 (8) & $\neg Pk$ & 1,7,MP \\
 (9) & $Dk\rightarrow Hk \vee Pk$ & 2,$-\forall$(1)\\
 (10) & $Hk\vee Pk$ & 1,9,MP \\
 (11) & $Hk$ & 8,10,DS \\
 $\therefore$ & \bl{$(\exists x)Hx$} & 10,$+\exists$ \\
\end{tabular} 
}

