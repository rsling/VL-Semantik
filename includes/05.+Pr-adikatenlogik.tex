\section{Warum Prädikatenlogik?}

\begin{frame}
  {Kompositionalität}
  \onslide<+->
  \onslide<+->
  Kaum Kompositionalität in der Aussagenlogik\\
  \Halbzeile
  \begin{itemize}[<+->]
    \item Kompositionalität beschärnkt auf Ebene des \alert{Satzes\slash der Proposition} (und größer)
    \item keine Generalisierungen über \alert{Individuen}, \alert{Eigenschaften} und \alert{Quantifikation}
    \item offensichtlicher Informationsverlust
      \begin{itemize}[<+->]
        \item \textit{Martin is an \underline{e}xpert on inversion and Martin is a good \underline{c}limber.}
        \item wird zu $e\wedge c$
      \end{itemize}
  \end{itemize}
\end{frame}

\begin{frame}
  {Erwünschte Schlussfolgerungen}
  \onslide<+->
  \onslide<+->
  Deduktion mit Quantifikation\\
  \Halbzeile
  \begin{itemize}[<+->]
    \item alle, einige\slash mindestens ein, zwei, die meisten, \ldots\ Individuen
    \item einige zu formalisierende Schlüsse
      \begin{itemize}[<+->]
        \item \alert{alle x} haben eine Eigenschaft $\vdash$ \alert{einige x} haben diese Eigenschaft
        \item \alert{Martin} hat eine Eigenschaft $\vdash$ \alert{mindestens ein x} hat diese Eigenschafgt
      \end{itemize}
  \end{itemize}
\end{frame}

\section{Syntax und Semantik}

\begin{frame}
  {Syntax von PL | Atome ($\approx$ Lexikon)}
  \onslide<+->
  \onslide<+->
  Atome sind nicht mehr automatisch Wffs.\\
  \Halbzeile
  \begin{itemize}[<+->]
    \item \alert{(Individuen)-Variablen}: $x_1, x_2, y, z, \cdots$ \grau{(Menge V)}
    \item \alert{(Individuen-)Konstanten}: $a, b, c, \cdots$ \grau{(Menge C)}
    \item \alert{Terme}: Variablen und Konstanten \grau{($T=V\cup C$)}
    \item \alert{Prädikatensymbole}: $A, B, C, \cdots$
      \begin{itemize}[<+->]
        \item jedes Prädikat mit $n$ Argumenten \grau{($n$-stellige Prädikate)}
        \item \grau{Mengen $P_1,\ldots,P_n\subset P$}
      \end{itemize}
    \item \alert{Quantoren}
      \begin{itemize}[<+->]
        \item $\exists$ \textit{es gibt mindestens ein \_\_}
        \item $\forall$ \textit{für alle \_\_\ gilt}
      \end{itemize}
    \item plus alle \alert{Funktoren} der Aussagenlogik
  \end{itemize}
\end{frame}

\begin{frame}
  {Syntax von PL | Komposition}
  \onslide<+->
  \onslide<+->
  Wffs aus Prädikaten, Termen und Quantoren\\
  \Halbzeile
  \begin{itemize}[<+->]
    \item \alert{$P(t_1,\cdots,t_n)$} ist eine Wff\\
      wenn \alert{$P\in P_n$} und \alert{$t_1,\cdots,t_n\in T$}
      \Halbzeile
    \item \alert{$(\exists x)\phi$} und \alert{$(\forall x)\phi$} sind Wffs\\
      wenn \alert{$x\in V$} und \alert{$\phi$ eine Wff} ist
      \Halbzeile
    \item Wffs lassen sich mit den Funktoren der Aussagenlogik kombinieren
      \Halbzeile
    \item Nichts anderes ist eine Wff in PL.\\
      \grau{Hinweis | Eigentlich ist Wff auch als Menge aller Wffs definiert.}
  \end{itemize}
\end{frame}

\begin{frame}
  {Semantik | Modelle und Individuen}
  \onslide<+->
  \onslide<+->
  \alert{Modell} | Die (Beschreibung der) Welt, relativ zu derer Wffs ausgewertet werden\\
  \Halbzeile
  \begin{itemize}[<+->]
    \item \alert{Model} \Model | enthält \alert{Diskursuniversum} $D$ = Menge aller Individuen
    \item Beispiel | ${\mathcal M}_1$ enthält $D=\{Martin,Kilroy,Scully\}$
      \Halbzeile
    \item \alert{Valuationsfunktion} | $\dem{\cdot}^{\Model}$ enthält eine Funktion in $T\times D$
      \Viertelzeile
    \item Beispiel für ein Model \Model\Sub{1}
      \begin{itemize}[<+->]
        \item für $m,k,s\in C$ und $Martin, Kilroy, Scully\in D$
        \item $\dem{m}^{\Model_1}=Martin$
        \item $\dem{k}^{\Model_1}=Kilroy$
        \item $\dem{s}^{\Model_1}=Scully$
      \end{itemize}
  \end{itemize}
\end{frame}

\begin{frame}
  {Semantik | Prädikate}
  \onslide<+->
  \onslide<+->
  Bedeutung von Prädikaten | \alert{Relationen} (Mengen von Tupeln)\\
  \Halbzeile
  \begin{itemize}[<+->]
    \item Erinnerung | n-stellige Relationen als \alert{n-Tupel}, hier Teil des Modells
      \begin{itemize}[<+->]
        \item \gruen<10>{\textit{schläft} | $R_1=\{Martin,Kilroy,Biden,\cdots\}$}
        \item \textit{Staatsoberhaupt von} | $R_2=\{\tuple{Biden,USA},\tuple{Xi,China},\tuple{Carl\ Gustaf,Schweden},\cdots\}$
        \item \gruen<11>{\textit{jagt} | $R_3=\{\tuple{Kilroy,Martin}, \tuple{Scully,Kilroy}\}$}
      \end{itemize}
      \Halbzeile
    \item Menge $R$ der Relationen \grau{($R^n$ für n-stellige)}
    \item Semantik eines Prädikats: Relation | $\dem{\cdot}^{\Model}$ enthält eine Funktion in $P\times \{0,1\}$
    \item \alert{$\dem{P(t_1)}^{\Model}=\dem{P}^{\Model}(\dem{t_1}^{\Model})=1\ iff\ \dem{t_1}^{\Model}\in\dem{P}^{\Model}$}\\
      \grau{\footnotesize Äquivalente Formulierung mit n-Tupeln für n-stellige Prädikate}
      \Viertelzeile
      \begin{itemize}[<+->]
        \item \gruen<10>{\textit{Martin ($m$) schläft ($R_1$)}: $\dem{R_1(m)}^{\Model_1}=1$ weil $\dem{m}^{\Model_1}=Martin$ und $Martin\in\dem{R_1}^{\Model_1}$}
        \item \gruen<11>{\textit{Martin ($m$) jagt ($R_3$) Kilroy ($k$)}: $\dem{R_3(m,k)}^{\Model_1}=0$ weil\\
          $\dem{m}^{\Model_1}=Martin$ und $\dem{k}^{\Model_1}=Kilroy$ und $\tuple{Martin,Kilroy}\not\in\dem{R_3}^{\Model_1}$}
      \end{itemize}
  \end{itemize}
\end{frame}

\frame{\frametitle{Semantics for connectives and quantifiers}
\begin{itemize}
  \item<1-> \bl{connectives}: `apply to' formulas (semantically truth-valued), semantics as in SL
  \item<2-> \bl{$(\forall x)\phi$} = 1 iff $\phi$ is true for every $d\in D$\\ assigned to every occurence of $x$ in $\phi$
  \item<3-> \bl{$(\exists x)\phi$} = 1 iff $\phi$ is true for at least one $d\in D$\\ assigned to every occurence of $x$ in $\phi$
  \item<4-> algorithmic instruction to check wff's containing Q's
  \item<5-> check outside-in (unambiguous scoping)
\end{itemize}
}

\frame{\frametitle{Dependencies}
\begin{itemize}
  \item<1-> universal quantifiers can be swapped:\\ $(\forall x)(\forall y)\phi \Leftrightarrow (\forall y)(\forall x)\phi$
  \item<2-> same for existential quantifiers:\\ $(\exists x)(\exists y)\phi \Leftrightarrow (\exists y)(\exists x)\phi$
  \item<3-> whereas: \bl{$(\exists x)(\forall y)\phi \Rightarrow (\forall y)(\exists x)\phi$}
  \item<4-> example in $\mathcal{M}_1$:
    \begin{itemize}
      \item<4-> $\dem{\underline{(\forall x)\underline{(\exists y)Qxy}}}^{\mathcal{M}_1}=1$
      \item<5-> but: $\dem{\underline{(\exists y)\underline{(\forall x)Qxy}}}^{\mathcal{M}_1}=0$
      \item<6-> direct consequence of algorithmic definition
      \item<7-> if $\exists\forall$ is true, $\forall\exists$ follows
    \end{itemize}
\end{itemize}
}

\frame{\frametitle{Hints on quantifiers}
\begin{itemize}
  \item<1-> domain of quantifiers: D (universe of discourse)
  \item<2-> $\forall x$ checks for truth of some predication for all individuals
  \item<3-> $\exists x(Px\wedge\neg Px)$ is a contradiction
  \item<4-> $\forall x(Wx\wedge \neg Wx)$ is a contradiciton,\\ $\forall x$ `checks' for an empty set by def.
  \item<5-> standard form of NL quantification:\\ \bl{$\forall x(Wx\rightarrow Bx)$} `All women are beautiful.'
  \item<6-> standard form of NL existential quantification:\\ \bl{$\exists x(Wx\wedge Bx)$} `Some woman is beautiful.'
\end{itemize}
}


\frame{\frametitle{Functor/quantifier practice}
\begin{itemize}
  \item<1-> by def., functors take formulas, not terms:
    \begin{itemize}
      \item<2-> $\neg Wm$ `Mary doesn't weep.'
      \item<3-> $(\exists x)(Gx \wedge Wx)$ `Some girl weeps.'
      \item<4-> \textcolor{red}{$^{\ast}W\neg x$}
      \item<5-> \textcolor{red}{$^{\ast}(\exists\neg x)(Gx)$}
    \end{itemize}
  \item<6-> quantifiers take variables, not constants:
    \begin{itemize}
      \item<7-> $(\forall x)(Ox\rightarrow Wx)$ `All ozelots are wildcats.'
      \item<8-> \textcolor{red}{$^{\ast}(\forall o)(Wo)$}
    \end{itemize}
  \item<9-> $\neg$ negates the wff, not the q:\\ \textcolor{red}{$^{\ast}(\neg\forall x)Px$} but $\neg(\forall x)Px$
\end{itemize}
}

\frame{\frametitle{Scope}
\begin{itemize}
  \item<1-> quantifiers \bl{bind} variables
  \item<2-> free variables (constants) are unbound
  \item<3-> \bl{no double binding} \textcolor{red}{$^{\ast}(\forall x\exists x)Px$}
  \item<4-> \bl{Q scope}: only the first wff to its right:
    \begin{itemize}
      \item<5-> $\underline{(\forall x)Px}\vee Qx$
      \item<6-> $\underline{(\forall x)(Px\vee Qx)} = \underline{(\forall x)Px} \vee \underline{(\forall x)Qx}$
      \item<7-> $\underline{(\exists x)Px} \rightarrow \underline{(\forall y)(Qy\wedge Ry)}$
      \item<8-> $\underline{(\exists x)Px}\wedge Qx$ (second $x$ is a unbound)
    \end{itemize}
  \item<9->\bl{no double-naming}
\end{itemize}
}

\section{Äquivalenzen}

\frame{\frametitle{Universal $\vee$ and $\wedge$}
\begin{itemize}
  \item<1-> $\exists$ and $\forall$ `or' and `and' over the universe of discourse (hence: $\bigvee$ and $\bigwedge$)
  \item<2-> \bl{$(\forall x)Px$ $\Leftrightarrow$ $Px_1\wedge Px_2\wedge \ldots\wedge Px_n$} for all $x_n$ assigned to $d_n\in D$
  \item<3-> \bl{$(\exists x)Px$ $\Leftrightarrow$ $Px_1\vee Px_2\vee\ldots\vee Px_n$} for all $x_n$ assigned to $d_n\in D$
  \item<4-> hence: \bl{$\neg(\forall x)Px$ $\Leftrightarrow$ $\neg(Px_1\wedge Px_2\wedge \ldots\wedge Px_n)$}
  \item<5-> with DeM: \bl{$\overline{Px_1\wedge Px_2\wedge \ldots\wedge Px_n}$}
  \item<6-> $\Leftrightarrow$ \bl{$\overline{Px_1}\vee \overline{Px_2}\vee \ldots\vee \overline{Px_n}$}
  \item<7-> $\Leftrightarrow$ \bl{$(\exists x)\neg Px$}
\end{itemize}
}

\frame{\frametitle{Quantifier negation (QN)}
\begin{itemize}
  \item<1-> $\neg(\forall x)Px \Leftrightarrow (\exists x)\neg Px$
  \item<2-> $\neg(\exists x)Px \Leftrightarrow (\forall x)\neg Px$
  \item<3-> $\neg(\forall x)\neg Px \Leftrightarrow (\exists x)Px$
  \item<4-> $\neg(\exists x)\neg Px \Leftrightarrow (\forall x)Px$
\end{itemize}
}

\frame{\frametitle{The distribution laws}
\begin{itemize}
  \item<1-> the conjunction of universally quantified formulas:\\
    \bl{$\underline{(\forall x)(Px\wedge Qx)} \Leftrightarrow \underline{(\forall x)Px}\wedge\underline{(\forall x)Qx}$}
  \item<2-> the disjunction of existentially quantified formulas:\\
    \bl{$\underline{(\exists x)(Px\vee Qx)} \Leftrightarrow \underline{(\exists x)Px}\vee\underline{(\exists x)Qx}$}
   \item<3-> not v.v.: $(\forall x)Px\vee(\forall x)Qx$ $\bl{\Rightarrow}$ $(\forall x)(Px\vee Qx)$ 
   \item<4-> why?
\end{itemize}
}

\frame{\frametitle{Quantifier movement (QM)}
\begin{itemize}
  \item<1-> desirable format: \bl{prefix + matrix}
  \item<2-> Movement Laws for antecedents of conditionals:\\ \bl{$(\exists x)Px \rightarrow \phi \Leftrightarrow (\forall x)(Px \rightarrow \phi)$}\\
  \bl{$(\forall x)Px \rightarrow \phi \Leftrightarrow (\exists x)(Px \rightarrow \phi)$}
  \item<3-> Movement Laws for Q's in disjunction, conjunction, and the
	  consequent of conditionals: \bl{Just move them to the prefix!}
  \item<4-> condition: \bl{$x$ must not be free in $\phi$.}
  \item<5-> i.e.: Watch your variables!
\end{itemize}
}

\frame{\frametitle{Let's formalize:}
\begin{itemize}
  \item<1-> \emph{Paul \underline{K}alkbrenner is a \underline{m}usician and \underline{s}igned on \underline{b}pitchcontrol.}
  \item<2->	\emph{Herr \underline{S}. \underline{i}nstalled \underline{R}edHat and not every \underline{L}inux distribution is \underline{e}asy to install.	}
  \item<3->	\emph{All \underline{t}alkmasters are \underline{h}uman and Harald
	\underline{S}chmidt is a talkmaster.}
  \item<4->	\emph{Some \underline{t}alkmasters are not \underline{m}usicians.}
  \item<5->	\emph{Heiko \underline{L}aux \underline{o}wns \underline{K}anzleramt records and
	does not \underline{l}ike any \underline{G}igolo artist.}
  \item<6-> \emph{Some \underline{h}umans are neither \underline{t}alkmasters nor
	do they \underline{o}wn \underline{K}anzleramt records.}
 \end{itemize}
}

\section{Schlussregeln}

\frame{\frametitle{Universal instantiation ($-\forall$) and generalization ($+\forall$)}
\begin{itemize}
  \item<1-> \bl{$(\forall x)Px\rightarrow Pa$}
  \item<2-> always applies
  \item<4-> can use any variable/constant
  \item<5-> \bl{$Pa\rightarrow (\forall x)Px$}
  \item<6-> iff $Pa$ was instantiated by $-\forall$
\end{itemize}
}

\frame{\frametitle{Existential generalization ($+\exists$) and instantiation ($-\exists$)}
\begin{itemize}
  \item<1-> \bl{$Pa \rightarrow (\exists x)Px$} for any individual constant a
  \item<2-> always applies
  \item<3-> \bl{$(\exists x)Px \rightarrow Pa$} for some indiv. const.
  \item<4-> always applies (there is a minimal individual for $\exists x$)
  \item<5-> for some $(\exists x)Px$ and $(\exists x)Qx$ the minimal individual might be different
  \item<6-> hence: \bl{When you apply EI, always use fresh constants!}
\end{itemize}
}

\frame{\frametitle{One sample task}
\begin{itemize}
  \item<1-> (1) Herr \underline{K}eydana \underline{d}rives a Golf. (2) Anything that drives a golf is \underline{h}uman or a complex \underline{p}rogram simulating an artificial neural net. (3) There are no programs s.a.a.n.n. which are complex enough to drive a Golf.
  \item<2-> Formalize and prove: \bl{At least one human exists.}
  \item<3-> (1) $Dk$
  \item<4-> (2) $(\forall x)(Dx\rightarrow Hx \vee Px)$
  \item<5-> (3) $\neg(\exists x)(Px\wedge Dx)$
  \item<6-> \bl{$(\exists x)Hx$}
\end{itemize}
}

\frame{\frametitle{The proof}
\begin{tabular}{lll}
 (1) & $Dk$ &  \\
 (2) & $(\forall x)(Dx\rightarrow Hx \vee Px)$ & \\
 (3) & $\neg(\exists x)(Px\wedge Dx)$ & \\
 \hline
 (4) & $(\forall x)\neg(Px\wedge Dx)$ & 3,QN\\
 (5) & $(\forall x)(\neg Px\vee\neg Dx)$ & 4,DeM\\
 (6) & $(\forall x)(Dx\rightarrow\neg Px)$ & 5,Comm,Impl \\
 (7) & $Dk \rightarrow \neg Pk$ & 6,$-\forall$(1) \\
 (8) & $\neg Pk$ & 1,7,MP \\
 (9) & $Dk\rightarrow Hk \vee Pk$ & 2,$-\forall$(1)\\
 (10) & $Hk\vee Pk$ & 1,9,MP \\
 (11) & $Hk$ & 8,10,DS \\
 $\therefore$ & \bl{$(\exists x)Hx$} & 10,$+\exists$ \\
\end{tabular} 
}

